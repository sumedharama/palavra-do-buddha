\documentclass[
  final,
  babelLanguage=portuguese,
  % desktopVersion,
  %showtrims,
  %overleaf,
]{anecdote}

%\graphicspath{{./assets/photos/300dpi/}}
\graphicspath{{./assets/photos/92dpi/}}

% Page size: 6x9 inch
% Body text: 10.5 / 15 pt

\usepackage{local}

%% Details of the book
%% ===================

\title{A Palavra do Buddha}
\subtitle{}
\author{Nyanatiloka Mahāthera}
\publisher{Publicações Sumedhārāma}
\date{2024-02-10}
% FIXME: Terceira edição?
\editionInfo{\textit{Terceira edição portuguesa}, 2024}
\ISBN{000-000-0000-00-0}% TODO update ISBN

% === Metadata ===

\hypersetup{
  pdftitle={\thetitle},
  pdfauthor={\theauthor},
  pdfcopyright={Copyright (C) 2024, \thePublisher},
  pdfsubject={},% TODO subject
  pdfkeywords={},% TODO keywords
  pdflicenseurl={https://creativecommons.org/licenses/by-nc-nd/4.0/},
  pdfcontacturl={},
  pdflang={en},
}

% \pdfinfo{%
%   /Title (\thetitle)%
%   /Author (\theauthor)
%   /Subject (subject)% TODO subject
%   /Keywords (keywords)% TODO keywords
%   /GTS_PDFXVersion (PDF/X-1:2001)%
%   /GTS_PDFXConformance (PDF/X-1a:2001)%
% }

%% === Load further packages ===

%% === Hyphenation exceptions and corrections ===

\hyphenation{London akusala}

\begin{document}

\frontmatter

\ifdesktopversion
\desktopCover{\includegraphics[height=\paperheight]{./desktop-cover.png}}
\fi

\cleartorecto
\thispagestyle{empty}
\vspace*{5em}

% FIXME: ? title style with large bold text

{\centering

\settowidth{\titleLength}{%
  {\Large\chapterTitleFont\scshape\MakeLowercase{\thetitle}}%
}

{\Large\chapterTitleFont\scshape\MakeLowercase{\thetitle}}\\[0.3\baselineskip]
\setlength{\xheight}{\heightof{X}}
\raisebox{0.5\xheight}{\color[gray]{0.4}\rule{\titleLength}{0.25pt}}\\[0.3\baselineskip]

Uma Síntese do Ensinamento do Buddha\\
baseada no Cânone Pāḷi

\vfill

Compilado, traduzido e comentado por

Nyanatiloka Mahathera

% Tradução Portuguesa de Bhikkhu Dhammiko

\vspace*{5em}

}



\cleartoverso
\thispagestyle{empty}

{\copyrightsize
\centering
\setlength{\parindent}{0pt}%
\setlength{\parskip}{0.8\baselineskip}%

\thetitle\ -- \thesubtitle\\
por \theauthor

Publicações Sumedhārāma\\
\href{https://sumedharama.pt}{www.sumedharama.pt}

Para distribuição gratuita\\
\textit{Sabbadānaṁ dhammadānaṁ jinati}\\
‘A oferta de Dhamma é superior a qualquer outra oferta.’

Este livro encontra-se disponível para distribuição gratuita em:\\
\href{https://sumedharama.pt}{www.sumedharama.pt}

ISBN \theISBN

Copyright \copyright\ Publicações Sumedhārāma 2024

Tradução: Dhammiko Bhikkhu\\
Formatação: Gambhīro Bhikkhu

Traduzido do original `The Word of the Buddha'\\
Publicado por Buddhist Publication Society\\
Sangharaja Mawatha -- Kandy, Sri Lanka

16ª Edição 1981\\
17ª Edição 2001\\
1ª Edição Portuguesa 2010\\
2ª Edição Portuguesa 2013\\
3ª Edição Portuguesa 2024

\vfill

{\footnotesize

Este trabalho está licenciado com uma Licença Creative Commons\\
Atribuição-NãoComercial-SemDerivações 4.0 Internacional.

Veja página \pageref{copyright-details} para mais detalhes sobre direitos e restrições desta licença.\\
Produzido com o sistema tipográfico \LaTeX. Fonte utilizada:\\
Gentium, Butler e Accanthis.

% FIXME: typefaces

\theEditionInfo

}}


\cleartorecto
\tableofcontents*

\chapter{Imagem da Capa}

``Pegadas do Buddha'' (\emph{Buddhapada}) é uma das representações mais antigas
da arte e da simbologia budista na fase anti iconográfica (a ausência de
estátuas). O \emph{Buddhapada} é altamente reverenciado em países budistas,
especialmente no Sri Lanka e na Tailândia. Na Índia, os pés têm sido objecto de
respeito muito antes do Budismo, como arquétipo de ligação do ``transcendente''
à Terra.

De acordo com a lenda, o Buddha depois da sua iluminação, deixou a impressão dos
seus pés numa pedra onde caminhara em Kusinara, na Índia. As pegadas simbolizam
a \emph{presença do Buddha}, no contacto com a Terra e paradoxalmente, também a
\emph{ausência do Buddha}, aquando da sua entrada no Nirvāna, daí a memória ao
ideal budista do desapego.

As pegadas do Buddha são normalmente representadas com todos os dedos dos pés no
mesmo comprimento e com um \emph{Dharma-chakra} (Roda do Dharma) ao
centro. Outros símbolos budistas antigos aparecem também nos calcanhares e
dedos, tais como o Lótus, a \emph{Swastika} e as \emph{Triratna} (Três Jóias).

Resgatando o verdadeiro significado ancestral da cruz \emph{swastika},
independentemente das atrocidades cometidas com a sua imagem pelos nazis, a
palavra deriva do Sânscrito \emph{svastika} (em Devanagari \devanagari{स्वस्तिक}),
significando fortuna e bem-estar, um símbolo utilizado para dar boa sorte. A
palavra é composta por su-significando ``bom'', ``bem'' e \emph{asti} ``ser''
\emph{svasti} significando ``bem-estar''. O sufixo - \emph{ka} ora forma um
diminutivo ora intensifica o significado verbal, e~\emph{svastika} pode então
traduzir-se literalmente como ``aquilo que está associado com bem-estar'',
correspondendo a ``boa fortuna'' ou ``algo auspicioso''. Historicamente,
tornou-se um símbolo sagrado no Hinduísmo, Jainismo, Mitraismo e Xamanismo,
ganhando importância no Budismo durante o Império Máuria. Com a propagação do
Budismo, alcançou o Tibete e a China. Pensa-se que o seu uso pela fé indígena do
Tibete, bem como de religiões sincréticas como a \emph{Cao Dai} do Vietnam e a
\emph{Falun Gong} da China, também se originou do Budismo. O símbolo pode também
ser encontrado por toda a Coreia.

Hoje em dia é usado na arte e nas escrituras budistas, representando o
Dharma, a harmonia universal e o equilíbrio dos opostos. Pode observar-se
a \emph{swastika} nos pilares de Ashoka (304 A.C.), onde simboliza a dança
cósmica em torno de um centro fixo, funcionando como protecção contra o mal.

\chapter[Prefácio à Décima Primeira Edição Inglesa]{Prefácio\\ à Décima Primeira Edição Inglesa}

\emph{A Palavra do Buddha}, cuja primeira edição foi publicada em língua alemã,
constituiu a primeira explanação sistemática das linhas mestras do Ensinamento
do Buddha, apresentada pelas palavras do próprio Mestre, tal como encontradas no
\emph{Sutta Piṭaka} do Cânone Pāli Budista.

Embora possa servir como primeira introdução para o principiante, o objectivo
principal deste livro é oferecer ao leitor que já se encontra mais ou menos
familiarizado com as ideias fundamentais do Budismo, uma síntese clara,
autêntica e concisa dos seus diversos ensinamentos, no enquadramento das
``Quatro Nobres Verdades'', respectivamente as verdades do sofrimento (inerente
a toda a existência), da origem do sofrimento, da extinção do sofrimento e do
caminho que conduz à extinção do sofrimento. Verifica"-se pelo próprio conteúdo
do livro, como os ensinamentos do Buddha, em última análise, convergem todos
para uma realização final: a Libertação do Sofrimento. Por essa razão se
encontrava impressa na capa da primeira edição em alemão, a seguinte passagem do
\emph{Aṅguttara Nikāya}, que diz:

\begin{verse}
  \emph{``Eu ensino não só a verdade do sofrimento, como também a libertação
    desse sofrimento''.}
\end{verse}

Os textos, traduzidos do Pāli original, foram seleccionados de entre as cinco
grandes colecções de discursos que formam o \emph{Sutta Piṭaka}. Foram agrupados
e explicados de modo a formarem um todo interligado. Assim, a colecção,
originalmente compilada de entre os inúmeros e volumosos livros do \emph{Sutta
  Piṭaka} para orientação do próprio autor, revela"-se um guia fidedigno para o
estudante do Budismo. Facilita o trabalho, no sentido de consultar todas as
demais secções das escrituras Pāli, permitindo obter uma visão clara no seu
todo; poderá ajudar a relacionar a parte principal da doutrina com os inúmeros
pormenores encontrados em estudos subsequentes.

Como o livro contém muitas definições e explanações de termos importantes da
doutrina, com respectiva equivalência Pāli, pode, com a ajuda da pronúncia Pāli
(ver p.\pageref{pron-pali}), servir como uma referência útil para o estudo individual
da doutrina do Buddha.

Depois da primeira edição em língua alemã em 1906, a primeira versão em língua
inglesa foi publicada em 1907 e, desde então, já se fizeram mais dez, incluindo
uma edição abreviada para estudantes (Colombo, 1948, Y.M.B.A.) e outra americana
(Santa Bárbara, Cal., 1950, J. F. Rowny Press). A obra foi também incluída na
Bíblia Budista de Dwight Goddard, publicada nos Estados Unidos da América.

Para além das edições subsequentes alemãs, já foram também editadas em francês,
italiano, checo, finlandês, russo, japonês, hindu, bengali e cingalês. O Pāli
original das passagens traduzidas foi publicado em caracteres ceilonenses
(edição do autor, sob o título \emph{Sacca"-Sangaha}, Colombo, 1914) e em escrita
devanagárica na Índia.

A 11ª edição foi totalmente revista. Foram feitas algumas adições à Introdução e
às notas explicativas, bem como acrescentados alguns textos.

\bigskip

{\raggedleft
  Nyanatiloka
\par}

\chapter[Prefácio à Décima Quarta Edição Inglesa (1967)]{Prefácio\\ à Décima Quarta Edição Inglesa\\ (1967)}

O venerável autor desta pequena obra emblemática da literatura budista, faleceu
em 28 de Maio de 1957, com a idade de 79 anos. A presente edição comemora o
décimo aniversário da sua morte.

Antes da sua partida, foi incluída uma reedição revista deste livro como 12ª
edição, em \emph{“The Path of Buddhism”}, publicação do Buddhist Council of
Ceylon (Lanka Bauddha Mandalaya). O texto das reimpressões seguintes foi baseado
nessa 12ª edição, apenas com algumas emendas menores. A partir da 13ª edição
(1959) e com a gentil permissão dos primeiros editores “Sāsanadhāra Kantha
Samitiya”, o livro é publicado agora pela Buddhist Publication Society
(Sociedade de Publicações Budistas).

Paralelamente a esta edição, a Sociedade publica também, em letra romana, com o
título de \emph{Buddha Vacanaṁ}, os textos Pāli originais que estão traduzidos
no presente livro. Esta edição Pāli tem o propósito de servir como leitura para
estudantes de língua Pāli e como manual de referência, bem como de breviário de
apreciação contemplativa para aqueles que, de certa forma, já conhecem a
linguagem das escrituras Budistas.

\bigskip

{\raggedleft
  Buddhist Publication Society\\
  Kandy, Ceylon, Dezembro 1967
\par}

\chapter[Prefácio à Primeira Edição Portuguesa (2010)]{Prefácio\\ à Primeira Edição Portuguesa\\ (2010)}

A compilação dos ensinamentos básicos dos Suttas do Tripitaka, pelo Venerável
Nyanatiloka, tem sido a minha referência e guia de meditação ao longo de 44 anos
de prática monástica.

No primeiro ano da minha vida monástica, em 1966, usei ``\emph{A~\mbox{Palavra} do
  Buddha}'' como único suporte durante esse longo e intenso ano de retiro de
meditação em Wat Nern Panow, Nong Khai, Tailândia. Esse ano transformou a minha
vida e deu"-me a inabalável fé na prática e no ensinamento do Buddha.

Estou muito feliz por termos este livro tão importante traduzido para a língua
portuguesa. Que seja de grande benefício para todos aqueles que estão
interessados nos ensinamentos essenciais do Senhor Buddha.

\bigskip

{\raggedleft
  Ven. Ajahn Sumedho\\
  Amaravati Buddhist Monastery, 2010
\par}

\chapter{PREFÁCIO à Terceira Edição Portuguesa}

Uma tradução é sempre delicada no que respeita a preservar o sentido e significado original que o autor quis transmitir, principalmente quando envolve um Ensinamento milenar, como o do Buddha. Este livro em particular, inclui transcrições directas das Escrituras Budistas Theravada, sendo esta uma tradução para a língua portuguesa da versão inglesa anteriormente traduzida do original alemão. Por conseguinte, exigiu um cuidado extra com relação aos princípios fundamentais do Dhamma, no sentido de evitar deturpações do seu significado original, uma vez que contém precisamente transcrições directas parciais de alguns dos principais \emph{Suttas} do Cânone Pāḷi.

Após várias considerações, procedeu-se ao ajuste de alguns dos termos chave do Budismo, como o caso mais pertinente do termo \emph{Sati} na língua Pāḷi, traduzido normalmente para o inglês como ``mindfulness'' e para o português directamente do inglês como ``plena atenção'', não fazendo a justiça devida ao significado original mais amplo que o termo \emph{Sati} comporta. Neste trabalho decidiu-se então, para traduzir melhor esse significado mais amplo de \emph{Sati}, adoptar-se a palavra ``consciência'' em vez de meramente ``plena atenção'', evitando assim a deturpação e o reducionismo do seu significado. De uma forma mais abrangente, o termo português ``consciência'', abarca simultaneamente os quatro termos em inglês, ``consciousness'', ``conscience'', ``mindfulness'' e ``awareness'', que traduzem mais completamente as noções ausentes nos termos ``plena atenção'' de uma consciência já espiritual, não só de atenção mesmo que plena no seu limite cognitivo em termos de memória, inteligência e consciência - a presença e lembrança correcta do que é importante e se deve fazer e como se deve fazer, saudável e correctamente, diligentemente, com responsabilidade humana e espiritual. Significa a forma cuidada com que se usa, aplica e presta atenção, mais do que a atenção propriamente dita, que por si só, na óptica da Palavra do Buddha, pode ser aplicada e usada tanto sábia como inconscientemente e até de uma forma criminosa. Daí os termos \emph{yoniso manasikara} (atenção correcta, sabiamente aplicada) versus \emph{ayoniso manasikara} (atenção incorrecta, mal aplicada).

Por sua vez na cultura e língua inglesa, o uso dos termos ``Consciousness'' versus ``Conscience'' apresentam uma distinção de significado em que o significado o primeiro se circunscreve mais aos sentidos físicos humanos, a função cognitiva física e sensorial do corpo, que em Pāḷi se traduz por \emph{Viññāṇa}, termo que neste livro se traduziu para o português como ``cognição'', fazendo aqui também jus à raiz sânscrita do Pāli \emph{Viññāṇa}. ``Conscience'' na cultura e língua inglesa, traduz mais o aspecto ético que se aproxima mais do significado de \emph{sati} no que respeita ao princípio do cuidado, da diligência e da consciência, num todo de princípios éticos e qualidades fundamentais que no Ensinamento do Buddha ajudam a conduzir-nos ao caminho da purificação e realização espiritual.

Sendo este só um exemplo, entre vários termos, tentou-se assim ajustar da melhor forma as diferentes expressões usadas, no sentido de respeitar tanto o Pāḷi original, como o próprio autor, o Venerável Nyanatiloka Mahāthera.

Os trechos em itálico encontrados ao longo do livro são os comentários do autor, à excepção dos termos Pāḷi também em itálico.

A fundamentação dos significados Pāḷi, foi apoiada com a colaboração de outros monásticos versados em Pāḷi e com o suporte adicional dos dicionários `Pāḷi-English Dictionary' de T. W. Rhys Davids e William Stede, bem como `Buddhist Dictionary' de Nyanatiloka Mahāthera.

\emph{Dhammiko Bhikkhu}

\chapter{Abreviaturas}

DN -- \emph{Dīgha Nikāya}

MN -- \emph{Majjhima-Nikāya}

AN -- \emph{Aṅguttara-Nikāya}

SN -- \emph{Saṁyutta-Nikāya}

Dhp -- \emph{Dhammapada}

Ud -- \emph{Udāna}

Snp -- \emph{Sutta-Nipāta}

VisM. -- \emph{Visuddhi-Magga (A Senda da Purificação)}

B. Dict. -- \emph{Buddhist Dictionary (Dicionário Budista)},\\
por Nyanatiloka Mahāthera

Fund. -- \emph{Fundamentals of Buddhism (Fundamentos do Budismo)},\\
por Nyanatiloka Mahāthera

\chapter{A Pronúncia Pāli}
\label{pron-pali}

\subsection{As Vogais}

\begin{tabular}{@{} L{5mm} L{\linewidth-5mm}}
\emph{a} & Em Português é pronunciado como \emph{a} mudo em \emph{par\prul{a}} ou \emph{c\prul{a}nec\prul{a}}.
\end{tabular}

\bigskip

Em Português do Brasil é melhor exemplificar em inglês: O \emph{a} funciona como o \emph{u} na palavra inglesa \emph{shut}; nunca aberto como em \emph{cat}, e nunca como em \emph{take}.

\bigskip

\begin{tabular}{@{} L{5mm} L{\linewidth-5mm}}
\emph{ā} & Como \emph{á} de \emph{já} ou \emph{a} de \emph{tomar}.\\

\emph{e} & É pronunciado como \emph{ê} longo.\\

\emph{i} & Como \emph{i}.\\

\emph{ī} & Como \emph{i} longo.\\

\emph{o} & Como \emph{ô} longo.\\

\emph{u} & Como \emph{u}.\\

\emph{ū} & Como \emph{u} longo.
\end{tabular}

\subsection{As Consoantes}

\begin{tabular}{@{} L{5mm} L{\linewidth-5mm}}
\emph{c} & É pronunciado como \emph{tch}, assim como o \emph{ch} inglês em \emph{chair}; nunca como \emph{c} em \emph{cavalo} ou \emph{ch} em \emph{cheirar}, \emph{chover}.\\

\emph{g} & Como em \emph{gamo}.\\
\end{tabular}

\clearpage

\begin{tabular}{@{} L{5mm} L{\linewidth-5mm}}
\emph{h} & Mesmo que colocado imediatamente a seguir às consoantes ou consoantes duplas, o \emph{h} é sempre aspirado como sopro em suspiro gutural, típico na língua inglesa; exemplo como no inglês:
\end{tabular}

\begin{tabular}{@{} L{5mm} L{5mm} L{\linewidth-10mm}}
& \emph{bh} & Como \emph{bh} em \emph{cabhorse}.\\

& \emph{ch} & Como \emph{chh} em \emph{ranch-house}.\\

& \emph{dh} & Como \emph{dh} em \emph{handhold}.\\

& \emph{gh} & Como \emph{gh} em \emph{bag-handle}.\\

& \emph{jh} & Como \emph{dgeh} em \emph{sledgehammer}, etc.\\
\end{tabular}

\bigskip

\begin{tabular}{@{} L{5mm} L{\linewidth-5mm}}
\emph{j} & Não como em \emph{jarra}, mas como \emph{dj} de \emph{djarra}; como na palavra inglesa \emph{joy}.\\

\emph{ṁ} & O chamado ‘nasal’ é como o \emph{m} em \emph{amparo} ou, \emph{ambiente}.\\

\emph{s} & Sempre como em \emph{sublimar} ou em \emph{se}; nunca como \emph{z}, ex: em \emph{causar} ou em \emph{físico}.\\

\emph{ñ} & Como o \emph{nh} normal na língua portuguesa; ex: \emph{manhã}, \emph{minha}, \emph{apanhar}, etc.\\

\emph{ph} & Como \emph{f} seguido de suspiro gutural como no inglês, assim como o \emph{ph} da palavra inglesa \emph{haphazard}.\\

\emph{th} & Como \emph{t} seguido de suspiro gutural típico do \emph{h} em inglês.\\

\emph{y} & Como o \emph{i} normal.\\
\end{tabular}

\bigskip

\emph{ṭ, ṭh, ḍ, ḍh:} São sons de língua, ditos cerebrais; ao pronunciá-los deve-se pressionar a língua contra o céu da boca.

\emph{Consoantes duplas:} Cada uma deve ser pronunciada, como \emph{bb} em \emph{subbase}.

\chapter{INTRODUÇÃO}

I. O BUDDHA

\begin{quote}
O BUDDHA ou Iluminado -- lit. Aquele que sabe ou o Desperto -- é o nome honorífico conferido ao Sábio indiano, Gotama, que desvendou e proclamou ao mundo a lei da libertação, conhecida no Ocidente pelo nome de Budismo.

Nasceu no Século VI a.C., em Kapilavatthu, filho do rei que na época regia o País Sakya, um principado situado na zona de fronteira com o actual Nepal. O seu nome próprio era Siddhattha e seu nome de clã Gotama (Sânscrito: \emph{Siddhārtha Gautama}). Aos 29 anos de idade, renunciou ao esplendor da sua vida principesca como herdeiro real, e tornou-se um asceta mendicante, com o propósito de descobrir uma solução para aquilo que antes havia reconhecido como um mundo de sofrimento. Depois de uma busca de seis anos sob a orientação de vários instrutores religiosos e de um período de auto-mortificação infrutífera, Siddhattha finalmente alcançou a Iluminação Perfeita (\emph{sammāsambodhi}), debaixo da árvore \emph{Bodhi} em Gayā (actualmente Boddh-Gayā). Seguiram-se quarenta e cinco anos de incansável ensinamento e pregação, e finalmente, no seu octogésimo ano de vida, morre em Kusinara ``aquele ser não iludido que surgiu para a bênção e alegria do mundo''.

O Buddha não é nem um deus nem um profeta, nem a encarnação de um deus, mas um ser humano supremo que, através do seu próprio empenho, alcançou a redenção final, a sabedoria perfeita, tornando-se ``o mestre sem par de deuses e homens''. É um ``Salvador'' unicamente no sentido em que mostra aos homens como se salvarem a si próprios, seguindo até ao fim, na prática, o caminho percorrido e mostrado por ele. O Buddha, na sua consumada harmonia de sabedoria e compaixão, encarna o ideal universal e intemporal do homem Aperfeiçoado.
\end{quote}

II. O DHAMMA

\begin{quote}
O DHAMMA -- é o Ensinamento da Libertação total, tal como foi desvendado, realizado e proclamado pelo Buddha. Tem sido transmitido na antiga língua Pāḷi e preservado em três grandes colecções de livros, chamados \emph{Ti-Piṭaka}, os ``Três Cestos'', nomeadamente: (I) o \emph{Vinaya-piṭaka}, ou a Colecção da Disciplina, contendo as regras da ordem monástica; (II) o \emph{Sutta- piṭaka}, ou a Colecção dos Discursos, consistindo em vários livros de discursos, diálogos, versos, histórias, etc., tratando da doutrina em si, tal como foi resumida nas ``Quatro Nobres Verdades''; (III) o \emph{Abhidhamma-piṭaka}, ou a Colecção Filosófica, apresentando os ensinamentos do \emph{Sutta-piṭaka} de uma forma sistemática e filosófica.

O \emph{Dhamma} não é uma doutrina de revelação, mas o ensinamento da Iluminação baseado na compreensão lúcida da realidade. É o ensinamento da \emph{Quádrupla Verdade} que trata dos factos fundamentais da vida e da libertação realizada através do próprio esforço do homem, em direcção à introspecção e purificação. O \emph{Dhamma} oferece um sistema ético superior, mas realista, uma análise penetrante da vida, uma filosofia profunda, métodos práticos para o treino da mente -- resumidamente, uma orientação no seu todo, perfeita e acessível no Caminho para a Libertação. Ao responder ao clamor tanto do coração como da razão, e ao mostrar o libertador ``Caminho do Meio'' que nos conduz para além de todos os extremos fúteis e destruidores da mente e da conduta individual, o \emph{Dhamma} tem e terá sempre um apelo intemporal e universal onde quer que existam corações e mentes suficientemente maduras para valorizar a sua Mensagem.
\end{quote}

III. O SANGHA

\begin{quote}
O SANGHA -- lit. a assembleia, ou comunidade -- é a Ordem dos \emph{Bhikkhus} ou Monges Mendicantes, fundada pelo Buddha e ainda existente na sua forma original em Myanmar (Birmânia), Tailândia, Sri Lanka (Ceilão), Camboja, Laos e Chittagong (Bengala). Juntamente com a Ordem dos monges Jainas, é uma das ordens monásticas mais antigas do mundo. Entre os mais famosos discípulos no tempo do Buddha, encontravam-se: Sāriputta que, a seguir ao próprio Mestre, tinha a mais profunda compreensão no \emph{Dhamma}; Moggallāna, dotado com os maiores poderes sobrenaturais; Ānanda, o devotado discípulo e constante companheiro do Buddha; Mahā-Kassapa, o Presidente do Conselho que se reuniu em Rājagaha imediatamente a seguir à morte do Buddha; Anuruddha, o mestre de visão divina e da consciência pura e Rāhula, o filho do próprio Buddha.

O \emph{Sangha} providencia o veículo externo e as condições favoráveis para todos aqueles que, livres das distracções mundanas, desejem seriamente devotar toda a sua vida à realização do mais elevado objectivo que é a libertação. Assim, o \emph{Sangha} também possui um significado universal e intemporal, onde quer que o desenvolvimento religioso alcance a maturidade.
\end{quote}

O TRIPLO REFÚGIO

\begin{quote}
O Buddha, o \emph{Dhamma} e o \emph{Sangha}, são designados ``As Três Jóias'' (\emph{tiratana}) pela sua pureza inigualável e por serem, para o budista, aquilo que há de mais precioso no mundo. Estas ``Três Jóias'' constituem também o ``Triplo Refúgio'' (\emph{ti-saraṇa}) que o praticante assume, ao proferir as palavras com as quais o declara ou reafirma, ao adoptá-las como guias da sua vida e do seu pensamento.

A fórmula Pāḷi do Refúgio é ainda a mesma aquando do tempo do Buddha:

\emph{Buddhaṃ saraṇaṃ gacchāmi}. \emph{Dhammaṃ saraṇaṃ gacchāmi. Sanghaṃ saraṇaṃ gacchāmi.}

Eu busco o refúgio no \emph{Buddha} Eu busco o refúgio no \emph{Dhamma} Eu busco o refúgio no \emph{Sangha}

É através do simples acto de recitar esta fórmula três vezes, que uma pessoa se considera budista. Na segunda e terceira repetições, acrescentam-se as palavras correspondentes \emph{Dutiyampi} e \emph{Tatiyampi} no início das frases:

\emph{Dutiyampi Buddhaṃ saraṇaṃ gacchāmi}.

\emph{Dutiyampi Dhammaṃ saraṇaṃ gacchāmi.}

\emph{Dutiyampi Sanghaṃ saraṇaṃ gacchāmi.}

Pela 2ª vez eu busco o refúgio no \emph{Buddha}

Pela 2ª vez eu busco o refúgio no \emph{Dhamma}

Pela 2ª vez eu busco o refúgio no \emph{Sangha}

\emph{Tatiyampi Buddhaṃ saraṇaṃ gacchāmi}.

\emph{Tatiyampi Dhammaṃ saraṇaṃ gacchāmi.}

\emph{Tatiyampi Sanghaṃ saraṇaṃ gacchāmi.}

Pela 2ª vez eu busco o refúgio no \emph{Buddha}

Pela 2ª vez eu busco o refúgio no \emph{Dhamma}

Pela 2ª vez eu busco o refúgio no \emph{Sangha}

OS CINCO PRECEITOS

A seguir à fórmula do Triplo Refúgio, normalmente assumem-se os Cinco Preceitos Morais \emph{(pañca-sīla)}. A sua obser- vância é o requisito de base para uma vida íntegra e consequente progresso em direcção à Libertação.

1. \emph{Pāṇātipātā veramaṇī-sikkhāpadaṃ samādiyāmi.}

Eu assumo o preceito de me abster de matar seres vivos.

2. \emph{Adinnādānā veramaṇī-sikkhāpadaṃ samādiyāmi.}

Eu assumo o preceito de me abster de tirar o que não me é oferecido.

3. \emph{Kāmesu micchācārā veramaṇī-sikkhāpadaṃ samādiyāmi.}

Eu assumo o preceito de me abster de sexualidade imprópria.

4. \emph{Musāvādā veramaṇī-sikkhāpadaṃ samādiyāmi.}

Eu assumo o preceito de me abster de discurso desonesto.

5. \emph{Surāmeraya - majja - pamādaṭṭhānā- veramaṇī - sikkhāpadaṃ samādiyāmi.}

Eu assumo o preceito de me abster de bebidas e

drogas intoxicantes que conduzem à falta de consciência.
\end{quote}

\chapter{As Quatro Nobres Verdades}

Assim foi dito pelo Buddha, o Iluminado:

Foi por não compreender, por não realizar quatro coisas, que eu, discípulos, tal
como vós, tive de vaguear tanto tempo neste ciclo de renascimentos. E quais são
essas quatro coisas? São:

\begin{enumerate}
  \item A nobre verdade do sofrimento \emph{(dukkha)};

  \item A nobre verdade da origem do sofrimento (\emph{dukkha"-samudaya});

  \item A nobre verdade da extinção do sofrimento (\emph{dukkha"-nirodha});

  \item A nobre verdade do caminho que conduz à extinção do sofrimento (\emph{dukkha"-nirodha"-gāminī"-paṭipadā}).
\end{enumerate}

\quoteRef{DN 16}

Enquanto a visão introspectiva e o conhecimento perfeitamente verazes,
respeitando estas “Quatro Nobres Verdades”, não se clarificaram em mim de
forma alguma, durante esse tempo, não tive a certeza de ter conquistado a
suprema iluminação, insuperável em todo o mundo com os seus seres celestiais,
espíritos malignos e deuses, entre todas as hostes de ascetas, sacerdotes e
homens. Mas, assim que a visão introspectiva e o conhecimento perfeitamente
verazes, com respeito a estas “Quatro Nobres Verdades”, se clarificaram em
mim, surgiu"-me interiormente a certeza de ter conquistado aquela suprema e
insuperável iluminação.

\quoteRef{SN 56.11}

E eu descobri aquela verdade profunda, tão difícil de perceber, difícil de
compreender, tranquilizadora e sublime, que não é conquistada por mero
raciocínio e só é visível aos sábios.

\quoteRef{MN 26}

O mundo, está, no entanto, votado ao prazer, deleitado no prazer, enfeitiçado
com o prazer. Na verdade, tais seres dificilmente compreenderão a lei da
condicionalidade, a génese dependente (\emph{paṭicca"-samuppāda}) de tudo;
incompreensível para eles será também o fim de todas as formações, o abandono de
tudo o que subjaz a cada renascimento, o desaparecimento da cobiça, o desapego
-- a libertação, o \emph{Nibbāna}.

No entanto, existem seres cujos olhos estão só ligeiramente cobertos de poeira:
estes compreenderão a verdade.

\quoteRef{MN 26}


% Page 1 is the first page of the first chapter.
\mainmatter

\chapterNote{A Nobre Verdade do Sofrimento}

\chapter{A Primeira Nobre Verdade}

\tocChapterNote{A Nobre Verdade do Sofrimento}

O que é afinal a nobre verdade do sofrimento?

Nascer é sofrimento; envelhecer é sofrimento; morrer é sofrimento; a tristeza, a
lamentação, a dor, a angústia e o desespero são sofrimento; não conseguir o que
se deseja, é sofrimento; resumindo: os cinco agregados da existência são
sofrimento.

E afinal, o que é nascer? É o aparecer de seres pertencentes a determinada
ordem, a sua concepção e o acto de nascerem, o virem à existência, a
manifestação dos agregados da existência, o começo da actividade sensitiva -- a
isto chama-se nascer.

E o que é envelhecer? É a degradação de seres pertencentes a determinada ordem,
o acumular de mais idade, o debilitar, o ficar grisalho, o enrugar; a diminuição
da força vital, a exaustão dos sentidos -- a isto chama-se envelhecer.

E o que é morrer? É a partida e o desfalecer de seres de determinada ordem, a
sua destruição, o desaparecimento, o término do seu período de vida, a
dissolução dos agregados da existência, o descartar do corpo -- a isto chama-se
morte.

E o que é a tristeza? A tristeza surge por qualquer tipo de perda ou infortúnio
com que a pessoa se depara, pela preocupação, susto, aflição e lamento -- a isto
chama-se tristeza.

E o que é o lamento? É toda a lamúria e queixume por qualquer tipo de perda,
infortúnio sofrido, o facto de se lamentar e recriminar, o estado de aflição e
deploração -- a isto chama-se lamentação.

E o que é a dor? É a sensação dolorosa e desagradável produzida pela impressão
física -- a isto chama-se dor.

E o que é a angústia? É a dor e o desagrado mental, o sentimento doloroso e
desagradável produzido pela impressão mental -- a isto chama-se angústia.

E o que é o desespero? É o estado aflitivo e angustiante que surge de qualquer
tipo de perda ou infortúnio com que a pessoa se depara, a desolação e a
exasperação -- a isto chama-se desespero.

E o que é o sofrimento por não se conseguir o que se deseja? Aos seres que estão
sujeitos a nascer, surge o desejo: ``Ah, pudéssemos não estar sujeitos a nascer!
Pudéssemos não ter pela frente mais nenhum nascimento!''. Sujeitos ao
envelhecer, à doença, à morte, à tristeza, à lamentação, à dor, à angústia e ao
desespero, surge-lhes o desejo: ``Ah, pudéssemos não estar sujeitos a estas
coisas! Pudéssemos não ter de nos sujeitar a isto de novo!'' Mas tal não se
obtém por mero desejo; e não obter o que se deseja, é sofrimento.

\quoteRef{\href{https://suttacentral.net/dn22/en/sujato}{DN 22}}

\clearpage

\section{Os Cinco Khandhas, ou Agregados da Existência}

\sectionSubtitle{(pañcupādānakkhandhā)}

E o que são os cinco agregados da existência? Eles são a corporalidade, a
sensação, a percepção, as formações mentais e a cognição.

\quoteRef{\href{https://suttacentral.net/dn22/en/sujato}{DN 22}}

Todos os fenómenos físicos, sejam eles do passado, do presente ou futuro, de uma
pessoa ou exteriores a ela, grosseiros ou subtis, superiores ou inferiores,
distantes ou próximos, pertencem ao agregado da corporalidade; todas as
sensações pertencem ao agregado da sensação; todas as percepções pertencem ao
agregado da percepção; todas as formações mentais pertencem ao agregado das
formações mentais; toda a cognição pertence ao agregado da cognição.

\quoteRef{\href{https://suttacentral.net/mn109/en/sujato}{MN 109}}

\begin{quote}
  Estes agregados estão reunidos numa classificação quíntupla, na qual o Buddha
  resumiu todos os fenómenos físicos e mentais da existência, principalmente
  aqueles que parecem ao homem ignorante como sendo o seu ego e sua
  personalidade. Assim, o nascimento, o envelhecimento, a morte, etc., estão
  também incluídos nestes cinco agregados que na realidade englobam o mundo
  inteiro.
\end{quote}

\clearpage

\subsection{O Agregado da Corporalidade}

\sectionSubtitle{(rūpa-khandha)}

Afinal, o que é o agregado da corporalidade? São os quatro elementos primários e
a corporalidade que deles deriva.

\subsubsection{Os Quatro Elementos}

E o que são os quatro elementos primários? São o elemento sólido, o elemento
fluido, o elemento térmico e o elemento vibrante (ventoso).

\quoteRef{\href{https://suttacentral.net/mn28/en/bodhi}{MN 28}}

\begin{quote}
  Os quatro elementos (\emph{dhātu} ou \emph{mahā-bhūta}), popularmente chamados
  de Terra, Água, Fogo e Ar, deverão ser entendidos como qualidades elementares
  da matéria. São chamados em Pāli, \emph{pa\d{t}havī-dhātu}, \emph{āpo-dhātu},
  \emph{tejo-dhātu}, \emph{vāyo-dhātu}, e podem ser traduzidos como inércia,
  coesão, radiação e vibração. Todos os quatro estão presentes em qualquer
  objecto material, variando, no entanto, em grau de força. Se, por exemplo, o
  elemento Terra predomina, o objecto material é chamado ``sólido'', etc.

  A ``corporalidade derivada dos quatro elementos'' (\emph{upādāya rūpa} ou
  \emph{upādā rūpa}) consiste, conforme o Abhidhamma, nos seguintes vinte e
  quatro fenómenos e qualidades materiais: olho, ouvido, nariz, língua, corpo,
  forma visível, som, cheiro, sabor, masculinidade, feminilidade, vitalidade, a
  base física da mente (\emph{hadaya-vatthu}; ver B. Dic.), gesto, fala, espaço
  (cavidades do ouvido, nariz, etc.), envelhecimento, mudança e nutrição.

  As impressões corporais (\emph{pho\d{t}\d{t}habba} -- o tacto) não são propriamente
  mencionadas entre estes vinte e quatro fenómenos, uma vez que são idênticas
  aos elementos, sólido, térmico e vibrante, que são apreendidas pelas sensações
  de pressão, frio, calor, dor, etc.
\end{quote}

(1.) O que é afinal o elemento sólido (\emph{paṭhavī-dhātu})? O elemento sólido
pode ser do próprio indivíduo, ou pode ser exterior. E o que é o elemento sólido
do indivíduo? O que quer que exista na própria pessoa ou no corpo de dureza e
firmeza adquiridas karmicamente, tais como os cabelos da cabeça e do corpo,
unhas, dentes, pele, carne, tendões, ossos, medula, rins, coração, fígado,
diafragma, baço, pulmões, estômago, intestinos, mesentério, excremento e por aí
fora -- a isto chama-se elemento sólido do indivíduo. Quer o elemento sólido
seja do indivíduo ou do exterior, é meramente o elemento sólido.

E há que compreender que, de acordo com a realidade e a verdadeira sabedoria --
``Isto não me pertence; eu não sou isto; isto não é o que eu sou''.

(2.) O que é afinal o elemento fluido (\emph{āpo-dhātu})? O elemento fluido pode
ser do indivíduo ou do exterior. E o que é o elemento fluido do indivíduo? O que
quer que exista na própria pessoa ou no corpo, de liquidez ou fluidez adquiridas
karmicamente, tais como a bílis, mucosidades, pus, sangue, suor, gordura,
lágrimas, gordura da pele, saliva, muco nasal, líquido nas articulações, urina e
por aí fora -- a isto chama-se elemento fluido do indivíduo. Quer o elemento
fluido seja do indivíduo ou do exterior, é mera- mente o elemento fluido.

E uma pessoa devia compreender, conforme a realidade e a verdadeira sabedoria,
que: - ``Isto não me pertence; isto não sou eu; isto não é o que eu sou''.

(3.) O que é afinal o elemento térmico (\emph{tejo-dhātu})? O elemento térmico
pode ser do indivíduo, ou do exterior. E o que é o elemento térmico do
indivíduo? O que quer que exista na própria pessoa ou no corpo, de calor ou
ardor adquiridos karmicamente, tal como tudo com que um indivíduo se aquece,
consome, queima, e tudo através do qual digere o que foi ingerido, bebido,
mastigado ou degustado e por aí adiante -- a isto chama-se o elemento térmico do
indivíduo. Quer o elemento térmico seja do indivíduo ou do exterior, é meramente
o elemento térmico.

Há que compreender, que de acordo com a realidade e a verdadeira sabedoria --
``Isto não me pertence; eu não sou isto; isto não é o que eu sou''.

\enlargethispage{\baselineskip}

(4.) O que é afinal o elemento vibrante (ventoso) \emph{(vāyo-dhātu)}? O
elemento vibrante pode ser do indivíduo, ou do exterior. E o que é o elemento
vibrante do indivíduo? O que quer que exista na própria pessoa ou no corpo, de
vento ou ventosidade adquiridos karmicamente, tal como os ventos que sobem e
descem, os ventos do estômago e intestinos, o vento que permeia todos os
membros, a inspiração e a expiração, etc. -- a isto chama-se o elemento vibrante
do indivíduo. Quer o elemento vibrante seja do indivíduo ou do exterior, é
meramente o elemento vibrante.

E há que compreender, conforme a realidade e a verdadeira sabedoria -- ``Isto
não me pertence; eu não sou isto; isto não é o que eu sou''.

Assim como se chama ``cabana'' ao espaço circunscrito formado com madeira e
juncos, canas e barro, da mesma forma se chama ``corpo'' ao espaço circunscrito
formado com ossos, tendões, carne e pele.

\quoteRef{\href{https://suttacentral.net/mn28/en/bodhi}{MN 28}}

\subsection{O Agregado da Sensação}

\sectionSubtitle{(vedanā-khandha)}

Existem três tipos de sensação: agradável, desagradável e nem agradável nem
desagradável.

\quoteRef{SN 36.1}

\subsection{O Agregado da Percepção}

\sectionSubtitle{(saññā-khandha)}

O que é afinal a percepção? Existem seis classes de percepção: percepção das
formas, dos sons, odores, sabores, sensações físicas, e dos objectos mentais.

\subsection{O Agregado das Formações Mentais}

\sectionSubtitle{(sa\.{n}khāra-khandha)}

O que são afinal as formações mentais? Existem seis classes de volições
(\emph{cetanā}): vontade projectada nas formas (\emph{rūpa-sañcetanā}), nos
sons, odores, sabores, nas sensações físicas, e nos objectos mentais.

\quoteRef{SN 22.56}

\clearpage

\begin{quote}
  O ``agregado das formações mentais'' (\emph{sa\.{n}khāra-khandha}), é um termo
  colectivo para representar inúmeras funções ou aspectos da actividade mental
  que, acrescidos à sensação e à percepção, estão presentes num só momento da
  consciência. No Abhidhamma, são distinguidas cinquenta formações mentais, sete
  das quais são factores constantes da mente. O número e a composição do
  restante, varia consoante o carácter da respectiva classe de cognição (ver
  quadro no \emph{B. Dic.}). No Discurso sobre a Visão Correcta (\href{https://suttacentral.net/mn9/en/bodhi}{MN 9}),
  são mencionados três factores principais representativos do agregado das
  formações mentais: volição (\emph{cetanā}), impressão sensual (\emph{phassa})
  e atenção (\emph{manasikāra}). Destes, uma vez mais, é a volição que sendo um
  factor principal ``formativo'', é particularmente característico do agregado
  das formações, tendo sido assim utilizado para exemplificá-lo na passagem
  acima referida.

  Para outras aplicações do termo \emph{sa\.{n}khāra}, ver \emph{B. Dic.}
\end{quote}

\subsection{O Agregado da Cognição}

\sectionSubtitle{(viññā\d{n}a-khandha)}

O que é afinal a cognição? Há seis classes de cognição: cognição das formas, dos
sons, dos odores, sabores, sensações físicas, e dos objectos mentais (lit.:
cognição-visão, cognição-audição, etc.).

\quoteRef{SN 22.56}

\section{A Génese Dependente da Cognição}

Mesmo que uma pessoa veja bem, se no entanto, as formas externas não estiverem
dentro do seu campo de visão, e não suceder qualquer ligação correspondente (de
vista e formas), não ocorrerá nesse caso a respectiva génese do aspecto da
cognição. Ou, mesmo que uma pessoa tenha boa visão e as formas externas estejam
dentro do seu campo de visão, mas mesmo assim não suceder qualquer ligação
correspondente, igualmente aí não ocorrerá a respectiva génese do aspecto da
cognição. Se, porém, a pessoa tiver uma boa visão, as formas externas estiverem
dentro do seu campo de visão e a ligação correspondente suceder, nesse caso
ocorrerá a respectiva génese do aspecto da cognição.

\quoteRef{\href{https://suttacentral.net/mn28/en/bodhi}{MN 28}}

Por isso afirmo: a génese da cognição depende das condições e, sem estas
condições, não se gera cognição alguma. E sejam quais forem as condições de que
a cognição depende, esta denomina-se segundo as respectivas condições.

Quando a génese da cognição depende da vista e das formas, denomina-se
``cognição visual'' (\emph{cakkhu-viññāṇa}).

Quando a génese da cognição depende do ouvido e dos sons, denomina-se ``cognição
auditiva'' (\emph{sota-viññāṇa}).

Quando a génese da cognição depende do órgão do olfacto e dos odores,
denomina-se ``cognição olfactiva'' (\emph{ghāna-viññāṇa}).

Quando a génese da cognição depende da língua e dos sabores, denomina-se
``cognição-palatal'' (\emph{jivhā-viññāṇa}).

Quando a génese da cognição depende do corpo e das sensações físicas,
denomina-se ``cognição-corporal'' (\emph{kāya-viññāṇa}).

Quando a génese da cognição depende da mente e dos objectos da mente,
denomina-se ``cognição-mental'' (\emph{mano-viññāṇa}).

\quoteRef{MN 38}

O que quer que exista de corporalidade (\emph{rūpa}), nesse momento pertence ao
agregado da corporalidade. O que quer que exista de ``sensação''
(\emph{vedanā}), pertence ao agregado da sensação. O que quer que exista de
``percepção'' (\emph{saññā}), pertence ao agregado da percepção. O que quer que
exista de ``formações mentais'' (\emph{saṅkhāra}), pertence ao agregado das
formações men- tais. O que quer que exista de ``cognição'' (\emph{viññāṇa})
pertence ao agregado da cognição.

\quoteRef{\href{https://suttacentral.net/mn28/en/bodhi}{MN 28}}

\section{A Dependência da Cognição\\ dos Outros Quatro Khandhas}

Também é impossível explicar o que é o término de uma existência e a entrada
noutra, ou o crescimento, ou o aumento e o desenvolvimento da cognição,
independentemente da corporalidade, sensação, percepção e formações mentais.

\quoteRef{SN 22.53}

\clearpage

\section{As Três Características da Existência}

\sectionSubtitle{(ti-lakkha\d{n}a)}

Todas as formações são ``transitórias'' (\emph{anicca}); todas as formações
estão sujeitas ao sofrimento (\emph{dukkha}); todas as coisas são desprovidas de
um eu (\emph{anattā}).

\quoteRef{AN 3.134}

A corporalidade é transitória, a sensação é transitória, a percepção é
transitória, as formações mentais são transitórias, a cognição é transitória.

E o que é transitório, está sujeito ao sofrimento; e é incorrecto dizer --
``Isto pertence-me; isto sou eu; isto é o que eu sou'' - daquilo que é
transitório e sujeito ao sofrimento e à mudança.

Assim, o que quer que exista de corporalidade, de sensação, percepção, formações
mentais, cognição, seja do passado, presente ou do futuro, do nosso interior ou
exterior, grosseiro ou subtil, elevado ou inferior, distante ou próximo, deve-se
compreender segundo a realidade e a verdadeira sabedoria -- ``Isto não me
pertence; eu não sou isto; isto não é o que eu sou''.

\quoteRef{SN 22.59}

\section{A Doutrina Anattā}

\begin{quote}
  A existência individual, bem como a de todo o mundo, não é, na realidade, mais
  do que um processo de fenómenos em constante mutação, todos incluídos nos
  cinco agregados da existência. Este processo tem decorrido desde antes do
  nosso nascimento, há tempos imemoriais, e assim continuará também depois da
  nossa morte, por tempos sem fim, enquanto e até onde existirem condições para
  tal. Como referido nos textos anteriores, os cinco agregados da existência --
  sejam eles considerados separadamente ou combinados -- de forma alguma
  constituem uma verdadeira entidade ego ou personalidade subsistente, e da
  mesma forma nenhum ``eu'', alma ou substância se poderá encontrar como seu
  proprietário fora destes agregados. Por outras palavras, os cinco agregados da
  existência são ``não-eu'' (\emph{anattā}), nem tão pouco pertencem a um ``eu''
  (\emph{anattaniya}). Tendo em conta a impermanência e condicionalidade de toda
  a existência, a crença em qualquer ``forma'' de ``eu'' deverá ser vista como
  uma ilusão.

  Tal como o que designamos de ``carruagem'' não tem existência separada dos
  eixos, das rodas, veios, corpo e por aí adiante, assim bem como a palavra
  ``casa'', que sendo apenas uma designação apropriada para indicar vários
  materiais reunidos, encerrando determinado espaço, não existe na realidade
  como entidade-casa separada, precisamente da mesma forma, aquilo que nós
  chamamos de ``ser'', ``indivíduo'', ``pessoa'', ou ``eu'', não é senão uma
  combinação transitória de fenómenos físicos e mentais, sem existência real
  própria.

  Isto é resumidamente, a doutrina anattā do Buddha, o ensinamento de que toda a
  existência é vazia (\emph{suñña}) de um ``eu'' ou substância permanente.

  É a doutrina fundamental budista, que não se encontra em nenhum outro
  ensinamento religioso ou sistema filosófico. Percebê-la plenamente, não só
  apenas de uma forma abstracta e intelectual, mas com referência constante à
  experiência real, é condição indispensável para a verdadeira compreensão do
  Buddha-Dhamma e para a realização do seu objectivo. A doutrina-anattā é o
  resultado indispensável da análise minuciosa da realidade, efectuada por
  exemplo na doutrina dos cinco khandhas, da qual só pode ser feita uma ligeira
  referência com os textos aqui incluídos.

  Para uma análise pormenorizada sobre os khandhas, ver \emph{B. Dic}.
\end{quote}

Imagine-se um homem que não sendo cego, contempla as inúmeras bolhas no Ganges,
observando-as e examinando-as à medida que passam; após tê-las examinado
cuidadosamente, parecem-lhe vazias, irreais e insubstanciais.

Precisamente da mesma maneira, o monge contempla todos os fenómenos corporais,
sensações, percepções, formações mentais, e estados de cognição -- sejam eles do
passado, do presente ou do futuro, distantes ou próximos. Observa-os e
examina-os cuidadosamente; e após examiná-los com cuidado, eles parecem-lhe
vazios, sem nada e sem um eu.

\quoteRef{SN 22.95}

Quem quer que se deleite na corporalidade, ou na sensação, ou na percepção, ou
nas formações mentais, ou na consciência, deleita-se no sofrimento; e quem se
deleita no sofrimento, não se libertará do sofrimento.

\quoteRef{SN 22.29}

\clearpage

Assim digo,

\begin{verse}
Que delícia e regozijo poderás encontrar\\
Quando tudo arde incessantemente?\\
Estás fechado na mais profunda escuridão!\\
Porque não procuras a luz?

Olha aqui para este fantoche, bem ataviado,\\
Acumulando mazelas,\\
Doente, e cheio de gula,\\
Instável, e impermanente!

Esta forma é devorada pela idade avançada,\\
Presa da doença, fraca e frágil;\\
Em pedaços se partirá este corpo pútrido,\\
A vida acabando na morte.
\end{verse}

\quoteRef{Dhp 146-148}

\section{Os Três Avisos}

Nunca viste um homem ou mulher neste mundo, com oitenta, noventa, ou cem anos de
idade, frágil, quebrado como um telhado velho, curvado, apoiado em muletas, com
passos inseguros, sem firmeza, a juventude há muito perdida, os dentes
estragados, cabelo ralo e branco ou nenhum, cheio de rugas, os seus membros com
manchas? E nunca pensaste que também tu estás sujeito à velhice e que não
conseguirás escapar-lhe?

Nunca viste um homem ou mulher neste mundo que, estando fatigados, aflitos, e
gravemente doentes, revolvendo-se na sua própria impureza, foram ajudados por
uns para se levantarem e postos na cama por outros? E nunca pensaste que também
tu estás sujeito a adoecer, que a isso não conseguirás escapar? Nunca viste o
corpo de um homem ou mulher neste mundo, um, dois ou três dias depois da sua
morte, inchado, de cor azul-escuro, em pleno apodrecimento? E nunca pensaste que
também tu estás sujeito à morte, que não lhe conseguirás escapar?

\quoteRef{AN 3.35}

\section{Sa\.{m}sāra}

O começo deste \emph{Saṁsāra} é inconcebível; difícil é de se conhecer qualquer
princípio dos seres que, obstruídos pela ignorância e enredados na cobiça,
correm apressadamente através deste ciclo de renascimentos.

\quoteRef{SN 15.3}

\begin{quote}
  \emph{Sa\.{m}sāra} -- a roda da existência, lit. o ``ciclo perpétuo'' - é o
  nome dado, nas escrituras Pāli, ao mar da vida que se agita constantemente
  para cima e para baixo, símbolo deste contínuo processo de nascer, uma e outra
  vez, de envelhecer, sofrer e morrer. Mais precisamente: \emph{Sa\.{m}sāra} é a
  sequência ininterrupta das combinações entre os cinco khandhas que, em
  constante mutação, a cada momento se sucedem continuamente ao longo de
  períodos inconcebíveis. Deste saṁsāra, o período de uma vida constitui apenas
  uma minúscula fracção. Assim, de modo a compreender a Primeira Nobre Verdade,
  dever-se-á meditar no saṁsāra, nesta terrível sequência de renascimentos, e
  não meramente numa só vida, a qual, como é evidente, poderá, por vezes, não
  ser assim tão dolorosa.

  Assim, o termo ``sofrimento'' (\emph{dukkha}), na primeira nobre verdade,
  refere-se não só às sensações dolorosas do corpo e da mente provocadas pelas
  impressões desagradáveis, mas inclui também tudo o que produz sofrimento ou
  que seja responsável por este. A verdade do sofrimento ensina que, devido à
  lei universal da impermanência, até os estados sublimes e elevados de
  felicidade estão sujeitos à mudança e a acabar, e que todos os estados desta
  existência são assim insatisfatórios, carregando em si, sem excepção, as
  sementes do sofrimento.
\end{quote}

O que pensais ser maior: a inundação das lágrimas que, em choro e lamento haveis
derramado neste longo caminho -- nesta corrida desenfreada ao longo deste ciclo
de renascimentos unido ao que é indesejável e separado do que é desejável -- ou
as águas dos quatro oceanos?

Durante muito tempo sofrestes a morte de pai, mãe, filhos, filhas, irmãos e
irmãs. E nesse sofrimento, haveis derramado na realidade mais lágrimas neste
longo caminho do que a água dos quatro oceanos.

O que pensais ser maior: os rios de sangue que foram derramados pela vossa
decapitação, neste longo caminho\ldots{} ou as águas dos quatro oceanos?

Por eras sem fim, tendes sido apanhados como ladrões, bandidos ou adúlteros e,
na verdade, pela vossa decapitação, correu muito mais sangue neste longo caminho
do que a água dos quatro oceanos.

Mas como é isto possível?

É inconcebível o começo deste \emph{Saṁsāra}. Difícil será de conhecer qualquer
início dos seres que, obstruídos pela ignorância e enredados pela cobiça, correm
apressadamente por este ciclo de renascimentos.

\quoteRef{SN 15.13}

E assim, há muito que vós tendes vindo a sofrer, vivendo tormento, vivendo o
infortúnio, enchendo os cemitérios; na verdade, já há muito que tendes vivido o
suficiente para vos sentirdes insatisfeitos com todas as formas de existência, o
bastante para partir e libertar-vos de todas elas.

\quoteRef{SN 15.1}

\chapterNote{A Nobre Verdade da Origem do Sofrimento}

\chapter{A Segunda Nobre Verdade}

\tocChapterNote{A Nobre Verdade da Origem do Sofrimento}

Afinal, o que significa a nobre verdade da origem do sofrimento? É o anseio que enlaçado pelo prazer e pela sensualidade, provoca o renascimento, logo encontrando renovado deleite, ora aqui, ora acolá.

\section{O Triplo Anseio}

Existe o anseio sensual (\emph{kāma"-taṇhā}); o anseio pela existência (eterna) (\emph{bhava"-taṇhā}); o anseio pela auto"-aniquilação (\emph{vibhava"-taṇhā}).

\quoteRef{DN 22}

\begin{quote}
Anseio sensual (\emph{kāma"-ta\d{n}hā}) é o anseio por desfrutar do
prazer dos objectos dos cinco sentidos.

Anseio pela existência (\emph{bhava"-ta\d{n}hā}) é o anseio pela eternidade ou 
continuidade da vida, mais particularmente a vida naqueles mundos superiores chamados
existências de matéria subtil e existências imateriais (\emph{rūpa"-bhava}, e
\emph{arūpa"-bhava}). Está estreitamente relacionada com a chamada crença na
eternidade (\emph{bhava"-di\d{t}\d{t}hi} ou \emph{sassata"-di\d{t}\d{t}hi}), i.e.,
a crença num eu pessoal absoluto e eterno, que persiste independentemente do corpo.

O anseio pela auto"-aniquilação (lit., “pela não existência”,
\emph{vibhava"-ta\d{n}hā}) é o resultado do crer na aniquilação
(\emph{vibhava"-di\d{t}\d{t}hi} ou \emph{uccheda"-di\d{t}\d{t}hi}), i.e., a noção
materialista ilusória de um “eu” mais ou menos real que se aniquila no momento
da morte, não permanecendo nenhuma relação casual com o tempo, antes e depois da
morte.
\end{quote}

\section{A Origem do Anseio}

Mas onde nasce e ganha raiz este anseio? Onde quer que no mundo existam coisas
adoráveis e agradáveis, este anseio surge e ganha raiz. Os olhos, os ouvidos, o
nariz, a língua, o corpo e a mente, transmitem prazer e agrado: aí este anseio
surge e ganha raiz.

Os objectos visuais, os sons, os cheiros, os sabores, as impressões corporais e
os objectos da mente são belos e agradáveis: aí este anseio surge e ganha raiz.

A cognição, a impressão sensorial e a sensação nascida da impressão
sensorial, da percepção, da vontade, do anseio, do pensamento e da reflexão, são
belas e agradáveis: aí este anseio surge e ganha raiz.

\enlargethispage{\baselineskip}

Esta é chamada a nobre verdade da origem do sofrimento.

\quoteRef{DN 22}

\section{A Génese Dependente de Todos os Fenómenos}

Sempre que alguém percepcione um objecto visual, som, odor, sabor, impressão
corporal, ou um objecto mental, se o objecto for agradável, sentirá atracção; se
o objecto for desagradável, sentirá repulsa.

Assim, qualquer tipo de sensação (\emph{vedanā}) que seja experimentada por
alguém -- agradável, desagradável ou indiferente -- se a pessoa aprovar,
acalentar e se apegar a essa sensação, ao fazê"-lo, surge o anseio; mas, ansear
por sensações significa apego (\emph{upādāna}) e é do apego que depende o
(presente) processo de retorno; por sua vez é do processo de retorno
(\emph{bhava}; neste caso \emph{kamma"-bhava}, processo kármico) que depende o
(futuro) nascimento (\emph{jāti}); e a decadência, a morte, a tristeza, a
lamentação, a dor, a angústia e o desespero, assentam no nascimento. Desta forma
surge toda esta carga de sofrimento.

\quoteRef{MN 38}

\begin{quote}
  A fórmula da génese dependente (\emph{pa\d{t}icca"-samuppāda}) da qual só algumas das
  doze correspondências foram mencionadas na passagem anterior, pode ser
  entendida como uma explicação minuciosa da segunda nobre verdade.
\end{quote}

\section{Os Resultados-Kármicos Presentes}

Na realidade, devido ao anseio dos sentidos, condicionados pelo anseio dos
sentidos, impelidos pelo anseio dos sentidos, completamente movidos pelo anseio
dos sentidos, reis lutam contra reis, príncipes contra príncipes, padres contra
padres, cidadãos contra cidadãos; a mãe discute com o filho, o filho com o pai;
o irmão discute com o irmão, o irmão com a irmã, amigo com amigo.

Assim, entregues à dissensão, à implicância e ao desacato, atiram"-se uns aos
outros com punhos cerrados, paus e armas. E, por conseguinte, acabam por sofrer
dor mortal ou morte.

E mais ainda, devido ao anseio dos sentidos, condicionadas pelo anseio dos
sentidos, impelidas pelo anseio dos sentidos, completamente movidas pelo anseio
dos sentidos, as pessoas arrombam casas, roubam, saqueiam e cometem sérios
assaltos na rua e na estrada e seduzem as mulheres do alheio. Então, os
governantes mandam prender estas pessoas, infligindo"-lhes várias medidas de
punição. E, por isso, acabam por incorrer na dor mortal e na morte. Ora, isto é
a miséria do anseio dos sentidos, o acumular do sofrimento nesta vida presente
devido ao anseio dos sentidos, condicionado pelo anseio dos sentidos, gerado
pelo anseio dos sentidos, totalmente dependente do anseio dos sentidos.

\quoteRef{MN 13}

\section{Os Resultados-Kármicos Futuros}

E nesta sequência, as pessoas seguem o caminho do mal por acções, palavras e
pensamentos; e ao irem pelo caminho do mal por acções, palavras e pensamentos,
no momento da desintegração do corpo, após a morte, caem num estado involutivo
de existência, num estado de sofrimento, num destino infeliz, nos abismos do
inferno. Mas esta é a miséria do anseio dos sentidos, o acumular de sofrimento
na vida futura devido ao anseio dos sentidos, condicionado pelo anseio dos
sentidos, gerado pelo anseio dos sentidos, totalmente dependente do anseio dos
sentidos.

\quoteRef{MN 13}

\begin{verse}
  Nem no ar, nem no meio do oceano,\\
  Nem escondido nas frestas da montanha,\\
  Em sítio algum se encontra um lugar na Terra,\\
  Onde o homem esteja livre de más acções.

  \quoteRef{Dhp 127}
\end{verse}

\section{O Karma Como Volição}

É à volição (\emph{cetanā}) que chamo \emph{“Kamma”} (acção -- Skr: \emph{karma}). Por se ter
desejado, age"-se com o corpo, com a fala, com a mente.

Há acções (\emph{kamma}) a amadurecer nos infernos\ldots{} a amadurecer no reino
animal\ldots{} a amadurecer no domínio dos espíritos\ldots{} a amadurecer entre
os homens\ldots{} a amadurecer em mundos celestiais.

O resultado das acções (\emph{vipāka}) é de três tipos: amadurecimento na vida
presente; na próxima; ou em vidas futuras.

\quoteRef{AN 6.63}

\section{A Herança das Acções}

Todos os seres são os responsáveis pelas suas acções (\emph{kamma, Skr: karma}),
herdeiros das suas acções: as suas acções são o útero de onde brotam, eles
aprisionam"-se com as suas acções, as suas acções são o seu refúgio.

Quaisquer acções que façam -- boas ou más -- eles serão os seus herdeiros.

\quoteRef{AN 10.206}

E onde quer que surjam os seres na existência, é aí que as suas acções
amadurecerão; e onde quer que amadureçam as suas acções, é aí que ganharão os
frutos dessas acções, nesta vida e nas futuras.

\quoteRef{AN 3.33}

Virá um tempo, em que o poderoso oceano secará, desaparecerá, e não mais
existirá. Virá um tempo em que a poderosa Terra será devorada pelo fogo,
perecerá, e não mais existirá. Mas mesmo assim, não haverá fim para o sofrimento
dos seres que, obstruídos pela ignorância e enganados pelo anseio, se apressam
e precipitam ao longo deste ciclo de renascimentos.

\quoteRef{SN 22.99}

\clearpage

\begin{quote}
  O anseio (\emph{ta\d{n}hā}) não é, no entanto, a única causa da má acção, nem por
  conseguinte, de todo o sofrimento e miséria produzidos desta forma, nesta e na
  próxima vida; mas onde quer que haja anseio, é aí que, na dependência desse
  anseio, poderá surgir inveja, raiva, ódio, e muitos outros males que geram a
  infelicidade e a miséria. E todos estes impulsos e acções egoístas de
  afirmação da vida, juntamente com os diversos tipos de miséria gerados, agora
  ou posteriormente, e mesmo todos os cinco agregados de fenómenos que
  constituem a vida -- está tudo basicamente enraizado na cegueira e na
  ignorância (\emph{avijjā}).
\end{quote}

\section{Karma}

\begin{quote}
  A segunda nobre verdade também ajuda a explicar as causas das aparentes
  injustiças na natureza, ensinando que nada no mundo se pode manifestar sem
  razão ou causa e que, não só as nossas tendências latentes, mas todo o nosso
  destino, toda a boa e má sorte, provêm de causas que devemos procurar, em
  parte nesta vida, em parte em vidas passadas.

  Estas causas são as actividades de afirmação da vida (\emph{kamma, Skr:
    karma}) produzidas pelo corpo, pela fala e pela mente. O carácter e o
  destino de todos os seres é assim determinado por esta tripla acção.

  O \emph{karma}, definido com exactidão manifesta essas volições boas e más
  (\emph{kusala"-akusala"-cetanā}), causando o renascer. Assim sendo, a
  existência, ou melhor, o proceder do retorno (\emph{bhava}) consiste num
  processo kármico activo e condicionante (\emph{kamma"-bhava}) e no seu
  resultado, o processo do renascer (\emph{upapatti"-bhava}).

  Igualmente, ao considerarmos o karma, não nos devemos esquecer da natureza
  impessoal (\emph{anattatā}) da existência. No caso de um maremoto por exemplo,
  não é a mesma onda que se apressa à superfície do oceano, mas sim o movimento
  de diferentes massas de água consideráveis. Da mesma forma se deve compreender
  que não existem entidades"-ego reais precipitando"-se através do oceano do
  renascimento, mas simplesmente ondas"-vida, que, de acordo com a sua natureza e
  actividades (boas ou más), se manifestam ora aqui como seres humanos, ora
  acolá como animais e noutros lugares como seres invisíveis.

  De novo se deve enfatizar o facto de que, correctamente falando, o termo
  “\emph{karma}” significa simplesmente os tipos de acção em si, já
  anteriormente referidos e não significa nem inclui os seus resultados.

  Para mais pormenores sobre o \emph{karma} ver \emph{Fund.} e \emph{B. Dict}.
\end{quote}

\chapterNote{A Nobre Verdade da Extinção do Sofrimento}

\chapter{A Terceira Nobre Verdade}

\tocChapterNote{A Nobre Verdade da Extinção do Sofrimento}

Afinal, o que significa a nobre verdade da extinção do sofrimento? É o total
desvanecimento e fim deste anseio, a sua renúncia e abandono, o seu desapego e a
sua libertação.

Mas onde pode este anseio desaparecer, onde é que poderá ser extinto?

Onde quer que existam coisas belas e agradáveis no mundo, aí se poderá desvanecer este anseio, aí poderá ser extinto.

\quoteRef{DN 22}

Seja no passado, presente ou futuro, quem de entre os monges ou sacerdotes
encarar as coisas belas e agradáveis como impermanentes (\emph{anicca}),
portadoras de infelicidade (\emph{dukkha}) e vazias de um eu (\emph{anattā}),
como doenças ou cancros, esses são os que superam o anseio.

\quoteRef{SN 12.66}

\section{A Dependência da Extinção\\ de Todos os Fenómenos}

É através do total desvanecimento e extinção do anseio (\emph{taṇhā}), que se
extingue o apego (\emph{upādāna}); através da extinção do apego, extingue"-se o
processo de retorno (\emph{bhava}); através da extinção do processo (kármico) do
retorno, extingue"-se o renascer (\emph{jāti}); e através da extinção do
renascer, extingue"-se a decadência e a morte, a tristeza, a lamentação, o
sofrimento, a angústia e o desespero. Assim se realiza a extinção de toda esta
carga de sofrimento.

\quoteRef{SN 12.43}

Daí o aniquilar, o cessar e o superar da corporalidade, da sensação, da
percepção, das formações mentais e da cognição -- isto é o fim do sofrimento, o
fim da doença, a vitória sobre a idade avançada e a morte.

\quoteRef{SN 22.30}

\begin{quote}
  O movimento ondulatório a que chamamos onda -- e que no observador ignorante
  gera a ilusão de uma e mesma massa de água movendo"-se à superfície do lago --
  é produzido e insuflado pelo vento e mantido pelas energias acumuladas. Ora,
  depois do vento parar e não mais agitar a água do lago, as energias acumuladas
  serão gradualmente consumidas e, consequentemente, todo o movimento
  ondulatório chegará ao fim. Da mesma forma, se não for adicionado ao fogo novo
  combustível, este extinguir"-se-á, depois de consumir todo o combustível
  existente.

  Assim também, este processo dos cinco khandhas -- que cria a ilusão de uma
  entidade"-ego na pessoa mundana ignorante -- é gerado e insuflado pelo anseio
  (\emph{ta\d{n}hā}) de afirmação da vida e mantido durante certo tempo, pelas energias de
  vida acumuladas. Ora, após o combustível (\emph{upādāna}), i.e., o anseio e
  apego à vida, cessar, se nenhum anseio impulsionar de novo este processo dos
  cinco khandhas, a vida continuará enquanto ainda houver energias de vida
  acumuladas, mas com a sua destruição pela morte, o processo dos cinco khandhas
  alcançará então a extinção final.

  Assim, Nibbāna, ou “extinção” (Sânscrito: \emph{nirvā\d{n}a}; derivado de nir +
  vā -- parar de soprar, apagar"-se) poderá ser considerado sob dois aspectos a
  citar:

  \begin{enumerate}

    \item “Extinção das Impurezas” (\emph{kilesa"-parinibbāna}), que se
          alcança ao realizar o nível de Arahant, ou Purificação Nobre, o que
          geralmente ocorre durante o período de vida; nos Suttas é referido
          como \emph{saupādisesa"-nibbāna}, i.e., “Nibbāna com os agregados da
          existência ainda remanescentes”.

    \item “Extinção do processo dos cinco khandhas”
          (\emph{khandha"-parinibbāna}), que ocorre à morte do Arahant, referida
          nos Suttas como: \emph{anupādisesa"-nibbāna}, i.e., “Nibbāna já sem os
          agregados da existência remanescentes”.

  \end{enumerate}
\end{quote}

\section{Nibbāna}

Isto é na verdade, a paz, o mais elevado, nomeadamente o fim de todas as
formações kármicas, o renunciar de toda a forma de renascimento, o
desvanecimento do anseio e do apego, a extinção, \emph{Nibbāna}.

\quoteRef{AN 3.32}

Extasiado na sensualidade, enfurecido pela raiva, cego pela ilusão, avassalado,
com a mente enredada, dirige"-se o homem à sua própria ruína, à ruína dos outros,
à ruína de ambos, acabando por experimentar a dor e a angústia mental. Mas, se
abandonar a sensualidade, a raiva e a ilusão, o homem não se dirige à sua
própria ruína, nem à ruína dos outros, nem à ruína de ambos, e acaba então por
não experimentar nem dor, nem angústia mental. Assim é o \emph{Nibbāna}
imediato, visível nesta vida, convidativo, cativante e compreensível aos olhos
dos sábios.

\quoteRef{AN 3.55}

A extinção da cobiça, a extinção do ódio, a extinção da ilusão, isto é na
verdade chamado de \emph{Nibbāna}.

\quoteRef{SN 38.1}

\clearpage

\section{O Arahant, o Puro (Santo)}

E para um discípulo assim liberto, em cujo coração mora a paz, nada mais há a
acrescentar ao que já foi feito, e nada mais resta ser feito. Tal como uma rocha
sólida e inabalável ao vento, assim também nem formas, nem sons, nem
odores, nem sabores, nem contactos de género algum, nem o desejável ou o
indesejável, conseguirão perturbar tal discípulo. Firme na sua mente, ele
conquista a libertação.

\quoteRef{AN 6.55}

E aquele que reflectiu sobre todos os contrastes nesta terra, que já não se
deixa perturbar por mais nada no mundo, “O Pacífico”, livre da raiva, da
tristeza e da saudade, esse transcendeu o nascimento e a decadência.

\quoteRef{Snp 1048}

\section{O Incondicionado}

Na verdade, existe uma dimensão, onde nem sequer existe o sólido, nem o fluido,
nem calor, nem movimento, nem este nem qualquer outro mundo, nem sol, nem lua.

A isto eu chamo nem surgir, nem passar, nem permanecer quieto, nem nascer, nem
morrer. Não existe sequer um ponto de apoio, nem desenvolvimento, nem qualquer
base. Isto é o fim do sofrimento.

\quoteRef{Ud 8.1}

Existe um Não"-nascido, Não"-originado, Não"-creado, Não"-formado. Se não existisse
este Não"-nascido, Não"-originado, Não"-creado, Não"-formado, então a saída do mundo
do nascido, do originado, do creado e do formado, não seria possível.

Mas uma vez que existe este Não"-nascido, Não"-originado, Não"-creado, Não"-formado,
é possível sair do mundo do nascido, do originado, do creado e do formado.

\quoteRef{Ud 8.3}

\chapter{A Quarta Nobre Verdade}

\textbf{A Nobre Verdade do Caminho que}

\textbf{Conduz à Extinção do Sofrimento}

\begin{quote}
\textbf{OS DOIS EXTREMOS E O CAMINHO DO MEIO}

Entregar-se à indulgência do prazer sensual, ordinário, comum, vulgar, mundano, inútil; ou entregar-se à auto-mortificação dolorosa, mundana, inútil; estes dois extremos foram evitados pelo ``Ser Perfeito'' que encontrou o ``Caminho do Meio'', que permite tanto ver como saber, que conduz à paz, ao discerni- mento, à iluminação, ao \emph{Nibbāna}.

O caminho que conduz à extinção do sofrimento é o ``Nobre

Óctuplo Caminho'', a citar:

1. Entendimento Correcto

\emph{Sammā-diṭṭhi}

2. Pensamento Correcto

\emph{Sammā-saṅkappa}

3. Palavra Correcta

\emph{Sammā-vācā}

4. Acção Correcta

\emph{Sammā-kammanta}

5. Sustento Correcto

\emph{Sammā-ājiva}

6. Empenho Correcto

\emph{Sammā-vāyāma}

7. Consciência Correcta

\emph{Sammā-sati}

8. Concentração Correcta

\emph{Sammā-samādhi}

III. Sabedoria

\emph{Paññā}
\end{quote}

I. Moralidade

\emph{Sīla}

II. Concentração

\emph{Samādhi}

\begin{quote}
Este é o ``Caminho do Meio'' que foi encontrado pelo ``Ser

Perfeito'', que permite tanto ver como saber, que conduz à paz, ao discernimento, à iluminação, ao \emph{Nibbāna}.

S. 56:11
\end{quote}

\subsubsection{O NOBRE ÓCTUPLO CAMINHO}\label{o-nobre-uxf3ctuplo-caminho}

\emph{(Arya-aṭṭhangikamagga)}

\begin{quote}
\emph{A expressão figurativa ``Caminho'' ou ``Via'' tem sido por vezes mal compreendida, como se os factores singulares desse Caminho tivessem que ser praticados pela ordem referida, um após o outro. Nesse caso, o Entendimento Correcto, i.e. a completa penetração da Verdade, ter-se-ia que realizar primeiro, ainda antes de se poder pensar em desenvolver o Pensamento Correcto, ou de praticar a Pala- vra Correcta, etc. Mas na realidade, os três factores (3-5) que constituem a secção da Moralidade (sīla) têm de ser aperfeiçoados primeiro; depois disso, deve-se dar atenção ao treino sistemático da mente praticando os três factores (6-8) que constituem a secção da Concentração (samādhi); só então depois desta preparação é que a mente e o carácter do ser humano permitirão alcançar a perfeição nos dois primeiros factores (1-2) que constituem a secção da Sabedoria (paññā).}

\emph{No entanto, logo de início, é indispensável um mínimo de Entendimento Correcto, fundamental para a compreensão dos factos acerca do sofrimento, etc., para oferecer razões convincentes e incentivo na direcção de uma prática diligente no Caminho. Algum Entendimento Correcto é também indispensável para ajudar os outros factores do Caminho, de modo a cumprir inteligente e eficazmente as suas funções na tarefa comum para a libertação. Por essa razão, e para enfatizar a importância deste factor, foi dado ao Entendi- mento Correcto o primeiro lugar no ``Nobre Óctuplo Caminho''.}

\emph{Este Entendimento inicial do Dhamma, no entanto, tem de ser desenvolvido gradualmente com a ajuda dos outros factores do Caminho, até se atingir finalmente aquela elevada lucidez introspectiva (vipassanā), que é a condição imediata para entrar nos quatro Estágios Sagrados (ver p. 64 f.) e alcançar o Nibbāna.}

\emph{O Entendimento Correcto é assim o princípio, bem como o culminar do ``Nobre Óctuplo Caminho''.}

Livre de dor e de tortura é o caminho, livre de pranto e de sofrimento: este é o caminho perfeito.

M. 139

Na verdade, não existe outro caminho como este para a pureza introspectiva. Se seguirdes este caminho, acabareis com o sofrimento.

Dhp. 274-75

Mas cada um tem que lutar por si próprio, os ``Seres Perfeitos'' somente apontaram o caminho.

Dhp. 276

Prestai atenção, pois a imortalidade descobre-se. Eu a revelo, eu exponho a Verdade. Tal como eu vos revelo, agi de acor- do! E esse supremo objectivo da vida sagrada, que por devoção inspira os filhos de boas famílias a deixar, acertadamente, a vida de casa pela vida mendicante: isto, vós tendes de descobrir, de realizar e de integrar em vós, sem demora, ainda nesta vida.

M. 26

\textbf{ENTENDIMENTO CORRECTO}

\emph{(Sammā-diṭṭhi)}

O Primeiro Factor

Afinal, o que significa o Entendimento Correcto?

ENTENDER AS QUATRO NOBRES VERDADES

1. Entender o sofrimento; 2. Entender a origem do sofrimento; 3. Entender a extinção do sofrimento; 4. Entender o caminho que conduz à extinção do sofrimento. A isto chama-se Entendi- mento Correcto.

D. 24

ENTENDER O QUE É BENÉFICO E PREJUDICIAL

Mais uma vez, quando o nobre discípulo entende o que é karmicamente saudável e qual a raiz do \emph{karma} saudável, o que é karmicamente prejudicial e qual a raiz do \emph{karma} prejudicial, ele então possui Entendimento Correcto.

E o que é que se apresenta como karmicamente prejudicial

(\emph{akusala})?

1. Destruição de seres vivos

2. Roubo

3. Relação sexual imprópria

4. Mentira

5. Intriga

6. Palavra rude

7. Conversa fútil

8. Cobiça

9. Má-fé

10. Juízos incorrectos
\end{quote}

Acção Corporal

\emph{(kāya-kamma)}

Acção Verbal

\emph{(vacī-kamma)}

Acção Mental

\emph{(mano-kamma)}

\begin{quote}
M. 9

\emph{Estas são as chamadas dez ``Vias de Acção Prejudicial''} \emph{(akusalakammapatha).}

E quais são as raízes do \emph{karma} prejudicial? A cobiça (\emph{lobha}) é uma raiz de karma prejudicial; o ódio (\emph{dosa}) é uma raiz de \emph{karma} prejudicial; a ilusão (\emph{moha}) é uma raiz de \emph{karma} prejudicial.

Por isso afirmo, estas acções prejudiciais são de três tipos: sejam devido à cobiça, ao ódio, ou à ilusão.

M. 9

\emph{Qualquer acto volitivo de corpo, palavra ou mente, enraizado na cobiça, no ódio ou na} ilusão\emph{, é considerado karmicamente prejudicial (a-kusala). É visto como akusala, i.e. prejudicial ou descuidado, uma vez que gera resultados negativos e dolorosos, nesta ou em qualquer existência futura. Aquilo que realmente conta como acção (kamma) é o estado da vontade, ou volição, que se pode manifestar exte- riormente como acção de corpo ou palavra; mas se não se manifestar exteriormente, conta como acção mental.}

\emph{O estado de cobiça (lobha), tal como o de ódio (dosa), é sempre acompanhado por ignorância (ou ilusão - moha), sendo esta a raiz principal de todo o mal. A cobiça e o ódio, não coexistem, no entanto, num mesmo e único momento de consciência.}

E o que é que se apresenta como karmicamente saudável (\emph{kusala})?

M. 9

1. Abster-se de matar

2. Abster-se de roubar

3. Abster-se de relação sexual imprópria

4. Abster-se de mentir

5. Abster-se de intriga

6. Abster-se de palavra rude

7. Abster-se de conversa fútil

8. Abster-se de cobiça

9. Abster-se de má-fé

10. Abster-se de juízos incorrectos
\end{quote}

Acção Corporal

\emph{(kāya-kamma)}

Acção Verbal

\emph{(vacī-kamma)}

Acção Mental

\emph{(mano-kamma)}

\begin{quote}
\emph{Estas são as chamadas dez ``Vias de Acção Saudável'' (kusalakammapatha).}

E quais são as raízes do \emph{karma} saudável? A ausência de cobiça (\emph{a-lobha} = altruísmo) é uma raiz de \emph{karma} saudável; a ausência de ódio (\emph{a-dosa} = bondade) é uma raiz de \emph{karma} saudável; a ausência de ilusão (\emph{a-moha} = sabedoria) é uma raiz de \emph{karma} saudável.

M. 9

COMPREENDER AS TRÊS CARACTERÍSTICAS

\emph{(ti-lakkhaṇa)}

Mais uma vez, quando se compreende que a corporalidade, o sentimento, a percepção, as formações mentais e a cognição são transitórias (sujeitas ao sofrimento, e sem um ``eu'') nessa condição existe Entendimento Correcto.
\end{quote}

S. 22:51

\begin{quote}
QUESTÕES INÚTEIS

Se alguém disser que não quer viver a vida pura sob a orientação do Abençoado, a não ser que o Abençoado lhe diga primeiro se o mundo é eterno ou temporal, finito ou infinito, se o princípio (causa primária) da vida é idêntico ao corpo ou se é algo diferente, se o ``Ser Perfeito'' (o Buddha) continua depois da morte, etc., -- tal pessoa morreria antes ainda de o ``Ser Perfeito'' lhe conseguir dizer tudo isso.

É como se um homem, atingido por uma seta envenenada, impedisse os seus amigos, companheiros e relações próximas de chamar um cirurgião, ao dizer ``não quero retirar esta seta enquanto não souber quem foi o homem que me feriu: se é um nobre, um padre, um comerciante, ou um criado''; ou, ``qual é o nome dele e a que família pertence''; ou, ``se é alto, baixo, ou de média estatura''. Na verdade, tal homem morreria ainda antes de conseguir saber tudo isso concretamente.

M. 63

Assim, a pessoa que procura o seu próprio bem, deveria retirar esta seta -- esta seta de lamentação, de dor e de tristeza. Snp. 592

Porque, quer exista ou não a teoria sobre o mundo ser eterno ou temporal, finito ou infinito -- o que é certo é que existe o nascer, o envelhecer, a morte, a tristeza, a lamentação, a dor, a angústia e o desespero, a extinção dos quais, ainda nesta vida

presente, eu vos dou a conhecer.

M. 63

AS CINCO PRISÕES

(Saṁyojana)

Imagine-se, por exemplo, uma pessoa inculta, sem consideração pelos homens santos, ignorante do ensinamento dos homens santos, sem qualquer treino da nobre doutrina. O seu coração encontrando-se possuído e dominado pela ilusão-do-ego, pelo cepticismo, pelo apego a meras regras e rituais, por luxúria e má-fé; esta pessoa não sabe na realidade, como se livrar destas coisas.

M. 64

\emph{A Ilusão-do-Ego (sakkāya-diṭṭhi) pode revelar-se como:}

\emph{1. ``Teoria do Eterno'' (bhava -- ou sassata-diṭṭhi, lit. ``convicção de eternidade'') i.e. a convicção de que um ego, um ``eu'' ou uma alma, existe eternamente, independentemente do corpo físico, continuando mesmo depois da desintegração deste.}

\emph{2. ``Teoria da Aniquilação'' (vibhava -- ou uccheda-diṭṭhi, lit. ``convicção de aniquilação'') i.e., a convicção materialista de que esta vida presente constitui o ``eu'', e que o ``eu'' por sua vez termina na morte do corpo físico.}

\emph{Para as dez ``prisões'' (}saṁyojana\emph{), ver na página nº 61}

CONSIDERAÇÕES IMPRÓPRIAS

Ao não saber o que é digno de consideração e o que é indigno de consideração, uma pessoa acaba por considerar o que é indigno e não considerar o que é digno.

E então, de forma insensata, considera: ``Será que já existi no passado? Ou será que não existi? O que é que eu fui no passado? Como é que eu fui? De que estado e para que estado eu mudei? -- Será que existirei no futuro? Ou, será que não existirei? O que é que serei? Como é que eu serei? De que estado e para que estado mudarei?'' -- E o presente também o enche de dúvida: ``Sou? Ou não sou? O que sou? Como é que sou? De onde veio este ser? Para onde vai este ser?''

AS SEIS TEORIAS ACERCA DO EU

E com tais considerações impróprias, adopta uma das seis teorias, tornando-se sua convicção e crença firme: «eu tenho um ``eu'\,'; ou eu não tenho um ``eu''; ou com o ``eu'' eu distingo o ``eu'\,'; ou com o não-eu, eu distingo o ``eu'\,'; ou com o ``eu'', eu distingo o não-eu»; Ou, adopta a seguinte teoria: «este meu ``eu'', capaz de pensar e sentir, que ora aqui, ora acolá, vive o fruto das boas e más acções -- este meu ``eu'' é permanente, estável, eterno, não sujeito à mudança, e permanecerá assim eternamente o mesmo».

M. 2

Se o ``eu'' realmente existisse, também existiria algo pertencente ao ``eu''. Como na verdade, não se consegue encontrar realmente, nem o ``eu'' nem nada pertencente ao ``eu'', não será deveras uma doutrina de imbecis proclamar: ``Isto é o mundo, isto sou eu; depois de morrer continuarei permanente e eterno''?

M. 22

Estas são chamadas meras teorias vulgares, um matagal de teorias, uma fantochada de teorias, uma trabalheira de teorias, uma armadilha de teorias; e enredado na prisão das teorias, o ser humano ignorante não se libertará do renascer, do envelhecer e da morte, do sofrimento, da dor, da angústia e do desespero; eu

vos digo, ele não se libertará do sofrimento.

CONSIDERAÇÕES SÁBIAS

No entanto, o discípulo nobre e culto, que tem grande consideração pelos homens santos, que conhece o ensinamento dos homens santos e tem treino da nobre doutrina, compreende o que é digno de consideração e o que é indigno.

Tendo este conhecimento, ele considera o que é digno, e não o indigno. Ele considera sabiamente o que é o sofrimento; ele considera sabiamente o que é a génese do sofrimento; considera sabiamente o que é a extinção do sofrimento; considera sabiamente o que é o caminho que conduz à extinção do sofrimento.

O \emph{SOTĀPANNA} OU ``AQUELE QUE ENTRA NA CORRENTE''

E assim considerando, três prisões se desvanecem, nomeadamente: a ilusão do ego, o cepticismo e o apego a meras regras e rituais.

M. 2

E aqueles discípulos em que as três prisões se desvaneceram, todos eles ``entraram na corrente'' (\emph{sotāpanna}).
\end{quote}

M. 22

\begin{quote}
Melhor do que qualquer poder mundano, Melhor do que todas as alegrias do céu, Melhor do que reinar sobre o mundo inteiro É a Entrada na Corrente''.
\end{quote}

Dhp. 178

\begin{quote}
AS DEZ PRISÕES

\emph{(Saṁyojana)}

\emph{Existem dez prisões (saṁyojana) pelas quais os seres ficam presos à roda da existência. Elas são: 1. a ilusão do ego (sakkāya-diṭṭhi); 2. o cepticismo (vicikicchā); 3. o apego a mera regra e ritual (sīlabbataparāmāsa); 4. a volúpia (kāmarāga); 5. a má-fé (vyāpāda); 6. o apego à existência na esfera material-subtil (rūpa-rāga); 7. o apego à existência imaterial (arūpa-rāga); 8. a soberba (māna); 9. a inquietação (uddhacca) e; 10. a ignorância (avijjā).}

OS NOBRES

\emph{(Ariya-puggala)}

\emph{Aquele que se liberta das três primeiras prisões é chamado ``Aquele que entra na Corrente'' (em Pāḷi: Sotāpanna), i.e., aquele que entrou na corrente que conduz ao Nibbāna. Ele tem uma fé inabalável no Buddha, no Dhamma e no Sangha, e é incapaz de quebrar os cinco Preceitos Morais.}

\emph{Renascerá mais sete vezes, no máximo, e não num estado inferior ao humano.}

\emph{Aquele que já transcendeu a quarta e a quinta prisões na sua forma grosseira, é chamado de Sakadāgāmi, lit. ``Aquele que Regressará só mais uma vez'', i.e., renascerá só mais uma vez na esfera sensual (kāma-loka) e, consequentemente, aí realizará a Pureza.}

\emph{Um Anāgāmi, lit. ``Aquele que já não regressa'', está totalmente livre das primeiras cinco prisões que sujeitam uma pessoa a renascer na esfera sensual; depois da morte, quando já estiver a viver na esfera material-subtil (rūpa-loka), realizará o objectivo.}

\emph{Um Arahant, i.e., o perfeitamente santo, está livre das}

\emph{dez prisões.}

\emph{Cada um dos quatro estágios da nobre Purificação previamente mencionados, consiste no ``Caminho'' (magga) e na ``Fruição'', e.g. ``O Caminho da Entrada na Corrente'' (sotāpatti-magga) e ``Fruição da Entrada na Corrente'' (sotāpatti-phala). Respectivamente, existem oito tipos, ou quatro pares, de ``Indivíduos Nobres'' (arya-puggala).}

\emph{O ``Caminho'' consiste no momento singular de entrada na respectiva realização. Por ``Fruição'' entendem-se aqueles momentos de consciência que se seguem, como resultado imediato do ``Caminho'', e que sob determinadas circunstâncias, se podem repetir inúmeras vezes no período de uma vida.}

\emph{Para mais pormenores, ver} B. Dict.: \emph{ariya-puggala, sotā-}

\emph{panna, etc.}

ENTENDIMENTO CORRECTO MUNDANO E SUPRA MUNDANO

Assim, afirmo que o Entendimento Correcto é de dois tipos:

1. O entendimento de que esmolas e oferendas não são inúteis; que existe fruto e resultado, ambos de boas e más acções; que existem tais coisas como esta e a próxima vida; que pai e mãe, igualmente como seres espontaneamente nascidos (nas esferas celestes), não são vulgares palavras; que existem no mundo, monges e homens santos, puros e perfeitos, que possam explicar esta e a próxima vida, a qual eles próprios viram -- isto é chamado o ``Entendimento Mundano Correcto'' (\emph{lokiya-sammā-diṭṭhi}), que providencia frutos mundanos e traz bons resultados.

2. Mas o que quer que exista de sabedoria, de discernimento, de entendimento correcto em conjunção com o ``Caminho'' (do \emph{Sotāpanna, Sakadāgāmi, Anāgāmi, ou Arahant}) -- a mente tendo-se retirado do mundo e unido ao caminho, seguindo o caminho puro -- isto é chamado o ``Entendimento Supramundano Correcto'' (\emph{lokuttara-sammā-diṭṭhi}), que não é do mundo, mas supra mundano e conjunto ao caminho.
\end{quote}

M. 117

\begin{quote}
\emph{Assim, há dois tipos do óctuplo caminho: (1) o mundano (lokiya), praticado pela pessoa mundana (puthujjana), i.e., por todos os que ainda não alcançaram o primeiro estágio de purificação; e (2) o supramundano (lokuttara) praticado pelos nobres de bom exemplo (ariya-puggala).}

EM CONJUNÇÃO COM OUTROS FACTORES

Ora, ao compreender-se o entendimento errado como errado e o entendimento correcto como correcto, pratica-se o \emph{entendimento correcto} (1º factor); ao esforçar-se por vencer o entendimento errado, desenvolvendo o entendimento correcto, pratica-se o \emph{empenho correcto} (6º factor); e ao vencer-se o entendimento errado com consciência correcta, mantendo a consciência correcta na posse de entendimento correcto, pratica-se a \emph{consciência correcta} (7º factor). Assim, existem três coisas que seguem e acompanham o entendimento correcto, respectivamente: entendimento correcto, empenho correcto, e consciência correcta.
\end{quote}

M. 117

\begin{quote}
LIVRE DE TODAS AS TEORIAS

Ora, se alguém me perguntasse se admito qualquer teoria, a resposta devia ser:

O ``Ser Perfeito'' é livre de qualquer teoria, pois o ``Ser Perfeito'' compreendeu o que é a corporalidade, como isso começa e acaba. Compreendeu o que é a sensação, como isso começa e acaba. Compreendeu o que é a percepção, como isso começa e acaba. Compreendeu o que são as formações mentais, como começam e acabam. Compreendeu o que é a consciência senso- rial, como começa e acaba. Portanto, afirmo, o ``Ser Perfeito'' conquistou a total libertação através do desvanecimento, do desaparecimento, da rejeição, da libertação e da extinção de todas as opiniões e conjecturas, de toda a inclinação para se vangloriar do ``eu'' e do ``meu''.
\end{quote}

M 72

\begin{quote}
AS TRÊS CARACTERÍSTICAS

Quer os ``Seres Perfeitos'' (\emph{Buddhas}) apareçam no mundo ou não, ainda assim permanece uma condição firme, um facto imutável e lei fixa: que todas as formações são impermanentes (\emph{anicca}); que todas as formações estão sujeitas ao sofrimento (\emph{dukkha}); que tudo é ``não eu'' (\emph{anattā}).
\end{quote}

A 3:134

\begin{quote}
\emph{Em Pāḷi: sabbe saṅkhārā aniccā, sabbe saṅkhārā dukkhā, sabbe saṅkhārā anattā.}

\emph{O termo ``saṅkhārā'' (formações) abarca aqui todas as coisas que são condicionadas ou ``formadas'' (saṅkhāta- dhamma), i.e., todos os possíveis constituintes físicos e mentais da existência. O termo ``Dhamma'', no entanto, comporta um significado ainda mais amplo e é todo abrangente, uma vez que também abarca o chamado Incondicionado (``não-formado'', asaṅkhata), i.e., Nibbāna.}

\emph{Por esta razão, seria errado dizer que todos os dhammas são impermanentes e sujeitos à mudança, porque, Nibbāna-dhamma é permanente e livre de mudança. Pela mesma razão, será correcto dizer-se que não só todos os saṅkhārās (=saṅkhata-dhamma) mas também todos os dhammas (incluindo o saṅkhata-dhamma) são desprovidos de um ``eu'' (anattā).}

Um fenómeno corpóreo, uma sensação, uma percepção, uma formação mental, uma cognição, que seja permanente e duradoira, eterna e não sujeita à mudança, tal coisa, um sábio não reconhece neste mundo; da mesma forma eu afir- mo que tal coisa não existe.

S 22:94

E é impossível que um ser imbuído de entendimento correcto identifique algo como sendo o ``eu''.

A 1:15

OPINIÕES E DISCUSSÕES ACERCA DO ``EU''

Ora, se alguém afirmar que a sensação é o seu ``eu'', dever-se-ia responder da seguinte maneira: «existem três tipos de sensação: agradável, dolorosa e nem agradável nem dolorosa. Qual destas três sensações considerais ser o vosso ``eu''?» Porque quando se experimenta uma destas sensações, não se experimentam as outras duas. Estes três tipos de sensação são impermanentes, de génese dependente, estão sujeitas à degeneração e à dissolução, a desvanecerem-se e a acabarem. Quem quer que, ao experimentar uma destas sensações, assuma como sendo o seu ``eu'', deve, depois de tal sensação acabar, admitir que o seu ``eu'' se dissolveu. E então considerará o seu ``eu'', já nesta vida presente, impermanente, misturado com prazer e dor, sujeito a começar e a acabar.

Se alguém afirmar que a sensação não é o seu ``eu'', e que o seu ``eu'' é inacessível a sensação, dever-se-ia perguntar: «Ora, será então possível dizer-se ``eu sou isto'' onde não existe a sensação?»

Ou, outrem poderá dizer: «a sensação, na verdade, não é o meu ``eu'', mas também não é verdade que o meu ``eu'' esteja inacessível à sensação, porque é o meu ``eu'' que sente, é o meu ``eu'' que possui a faculdade de sentir».

A essa pessoa dever-se-ia então responder: «Suponhamos que a sensação se extinguia completamente; ora, após a extinção de toda a sensação, se nenhuma sensação subsiste, será então possível dizer: ``eu sou isto''?»

D 15

Afirmar que a mente, ou os objectos-mente, ou a mente- consciência constituem o ``eu'' -- tal asserção é infundada. Uma vez que aí se observa o começo e o fim e vendo o começo e o fim destas coisas, poder-se-ia concluir então que o ``eu'' pessoal começa e finda.

M 148

Seria melhor que o ser mundano sem formação, considerasse o seu corpo, que é constituído pelos quatro elementos, como sendo o seu ``eu'', mais do que a sua mente. Porque está visto que o corpo pode durar por um ano, ou dois anos, ou três, quatro, cinco, ou dez anos, ou até por cem anos ou mais; mas aquilo a que se chama pensamento, mente, ou cognição, surge continuamente dia e noite como uma coisa, e termina como outra.

S 12:61

Assim, haja o que houver de corporalidade, sensação, percepção, formações mentais, ou de cognição, seja do passado, presente, ou futuro, individual ou externo, grosseiro ou subtil, elevado ou inferior, distante ou próximo, daqui se deveria depreender, de acordo com a realidade e a sabedoria perfeita, que: «Isto não me pertence; isto não sou eu; isto não é o meu ``eu''».

S 22:59

\emph{Em Vism. XVI, 90 encontra-se o seguinte verso aludindo à impersonalidade e ao vazio manifesto da existência:}

Mero sofrimento existe, mas nenhum sofredor se encontra; A acção existe, mas não alguém que age.

O \emph{Nirvāna} existe, mas não o ser humano que entra.

O caminho existe, mas não o viajante que o percorre.

PASSADO, PRESENTE, FUTURO

Ora, se alguém te perguntasse: ``Não exististe já no passado e não será mentira dizer que não exististe? Não existirás no futuro e não será mentira dizer que não existirás? Não existes agora e não será mentira dizer que não existes?'' -- poderás responder que já exististe no passado, e que é mentira dizer que não exististe; que existirás no futuro, e que é mentira dizer que não existirás; que existes, e que é mentira dizer que não existes.

No passado só a existência passada foi real, irreais a futura e a presente existências. No futuro só a existência futura será real, enquanto irreais a existência passada e presente. No agora, só a presente existência é real, irreais a existência passada e futura.

D 9

Quem discerne a génese dependente (\emph{paṭicca-samuppāda}), discerne a verdade; e quem discerne a verdade, discerne a génese dependente.

M 28

Pois, tal como o leite provém da vaca, a coalhada do leite, a manteiga da coalhada, o ghee (manteiga fina) da manteiga, a nata do ghee do próprio ghee; e quando é leite, este não é coalhada, nem manteiga, nem ghee, nem nata do ghee, mas somente leite, e quando é coalhada, esta é somente coalhada; assim também foi a minha existência passada, real nesse passado, mas irreal no futuro e no presente; e a minha futura existência será real nesse futuro, mas irreal no passado e no presente; e a minha presente existência é real agora, mas irreal no passado e no futuro. Tudo isto são designações e expressões meramente populares, termos de linguagem meramente convencionais, noções meramente populares. O ``Ser Perfeito'' sem dúvida faz uso delas, sem, no entanto, se apegar a elas.

D 9

Assim, quem não compreende a corporalidade, a sensação, a percepção, as formações mentais, a cognição de acordo com a realidade (i.e., como sendo vazios de um ``eu'') nem o seu começo, a sua extinção, nem o caminho para a sua extinção, sujeita-se assim a acreditar que o ``Ser Perfeito'' continua para além da morte, ou a acreditar que Ele não continua para além da morte, e por aí adiante.

S 44:4

OS DOIS EXTREMOS E A DOUTRINA DO MEIO

Na verdade, se uma pessoa está convencida de que o princípio vital (\emph{jiva} -- ``alma'') é idêntico ao corpo, nesse caso não é possível realizar uma vida santa; e se uma pessoa está convencida de que o princípio vital é algo bastante diferente do corpo, nesse caso também não é possível uma vida santa. O ``Ser Perfeito'' evitou estes dois extremos e demonstrou a Doutrina do Meio da génese dependente.
\end{quote}

S 12:25

\begin{quote}
A GÉNESE DEPENDENTE

\emph{(paṭicca-samuppāda)}

Dependentes da ignorância (\emph{avijjā}) estão as formações- kármicas (\emph{saṅkhārā}). Dependente das formações-kármicas está a cognição (\emph{viññāṇa}, começando com cognição-renascimento no útero da mãe). Dependente da cognição está a existência mental e física (\emph{nāmarūpa}). Dependentes da existência mental e física estão os seis órgãos dos sentidos (\emph{saḷāyatana})

Dependente dos seis órgãos dos sentidos\footnote{\begin{quote}
  Os seis órgãos dos sentidos e os seis objectos -- olho(s), ouvido(s), nariz, língua, corpo, e mente; formas, sons, odores, sabores, coisas tangíveis, ideias; ocupa o quarto lugar no \emph{Paṭiccasamuppāda}.
  \end{quote}} está a impressão sensorial (\emph{phassa}). Dependente da impressão sensorial está a sensação (\emph{vedanā}). Dependente da sensação está o anseio (\emph{taṇhā}). Dependente do anseio está o apego (\emph{upādāna}). Dependente do apego está o processo de retornar a ser (\emph{bhava}). Dependente do processo de retornar a ser (aqui: \emph{kamma-bhava} / processo-\emph{karma}) está o renascer (\emph{jāti}). Dependente do renascer está a degradação e a morte (\emph{jāra-maraṇa}), a tristeza, a lamentação, a dor, a angústia, e o desespero. E assim surge toda esta carga de sofrimento. A isto se chama a nobre verdade da génese do sofrimento.

S 12:1

Nenhum deus, nenhum Brahmā, pode ser designado

O criador desta roda da vida; Fenómenos vazios desenrolam-se, Dependentes das condições na sua totalidade.
\end{quote}

Vism. XIX, 20

\begin{quote}
No entanto, um discípulo no qual tenha desaparecido a ignorância e aparecido a sabedoria, tal discípulo nem acumula formações-kármicas meritórias, nem demeritórias, nem imperturbáveis.

S 12:51

\emph{O termo ``saṅkhārā'' foi aqui traduzido como ``formações-kármicas'' porque, no contexto da génese dependente, refere-se à volição kármica digna e indigna (cetanā), ou à actividade volitiva, resumindo, ao karma.}

\emph{Esta tripla classificação, na passagem anterior, compreende a actividade kármica em todas as esferas da existência ou planos de consciência. As ``formações-kármicas meritórias'' estendem-se inclusivamente à esfera material-subtil (rūpāvacara), enquanto as ``formações-kármicas imperturbáveis'' (āneñjābhisaṅkhārā) referem-se somente à dimensão imaterial (arūpāvacara).}

Assim, através do completo desaparecimento e extinção desta ignorância, terminam as formações-kármicas. Ao extinguirem-se as formações kármicas, termina a cognição (renascimento). Ao extinguir-se a cognição, termina a existência física e mental. Ao extinguir-se a existência física e mental, terminam os seis órgãos dos sentidos. Ao extinguirem-se os seis órgãos dos sentidos, termina a impressão sensorial. Ao extinguir-se a impressão sensorial, termina a sensação. Ao extinguir-se a sensação, termina o anseio. Ao extinguir-se o anseio, termina o apego. Ao extinguir-se o apego, termina o processo de vir a existir. Ao extinguir-se o processo de vir a existir, termina o renascer. Ao extinguir-se o renascer, terminam a degradação e a morte, a tristeza, o lamento, a dor, a angústia e o desespero. Assim se processa a extinção de toda esta carga de sofrimento. A isto chama-se a nobre verdade da extinção do sofrimento.

S 12:1

RENASCER -- PRODUZINDO \emph{KARMA}

Na verdade, os seres, por estarem obstruídos pela ignorância (\emph{avijjā}) e aprisionados pelo anseio (\emph{taṇhā}), procuram continuamente renovado prazer, ora aqui ora ali, e assim novo renascer se processa continuamente.
\end{quote}

M 43

\begin{quote}
E a acção (\emph{kamma}) que é realizada por cobiça, ódio, ilusão (\emph{lobha, dosa, moha}), tem sua fonte e génese nestas características: esta acção amadurece onde quer que o ser renasça, e onde quer que esta acção amadureça é aí que se vivem os frutos dessa acção, seja nesta vida ou numa outra vida futura qualquer.

A 3:33

A EXTINÇÃO DO \emph{KARMA}

No entanto, através do desaparecimento da ignorância, através do despontar da sabedoria, através do extinguir do anseio, jamais ocorrerá algum renascer.

M 43

Pois as acções cometidas por cobiça, ódio e ilusão, que despontam destas características, que se originam e resultam delas: tais acções, pela ausência de cobiça, ódio e ilusão, são abandonadas, desenraizadas, como uma palmeira que retirada da terra, é destruída e impedida de crescer novamente.

A 3:33

A este respeito pode dizer-se de mim correctamente: que eu ensino a aniquilação, que eu apresento a minha doutrina com o propósito da aniquilação, e que nesse propósito eu treino os meus discípulos; que na verdade ensino o aniquilamento -- respectivamente o aniquilamento da cobiça, do ódio, da ilusão, bem como de toda a malícia e coisas que não prestam.

A 8:12
\end{quote}

\textbf{A GÉNESE DEPENDENTE}

\begin{quote}
\emph{O diagrama seguinte dá um panorama de como as doze ligações da fórmula se estendem sobre três existências consecutivas - passado, presente e futuro.}
\end{quote}

% FIXME diagram

% \begin{longtable}[]{@{}
%   >{\raggedright\arraybackslash}p{(\columnwidth - 6\tabcolsep) * \real{0.2084}}
%   >{\raggedright\arraybackslash}p{(\columnwidth - 6\tabcolsep) * \real{0.2986}}
%   >{\raggedright\arraybackslash}p{(\columnwidth - 6\tabcolsep) * \real{0.2861}}
%   >{\raggedright\arraybackslash}p{(\columnwidth - 6\tabcolsep) * \real{0.2069}}@{}}
% \toprule\noalign{}
% \endhead
% \bottomrule\noalign{}
% \endlastfoot
% \begin{minipage}[t]{\linewidth}\raggedright
% \begin{quote}
% \emph{\textbf{3 períodos de tempo}}
% \end{quote}
% \end{minipage} & \begin{minipage}[t]{\linewidth}\raggedright
% \begin{quote}
% \emph{\textbf{12 factores ou nidānas}}
% \end{quote}
% \end{minipage} & \begin{minipage}[t]{\linewidth}\raggedright
% \begin{quote}
% \emph{\textbf{4 grupos / 5 modos cada}}
% \end{quote}
% \end{minipage} & \begin{minipage}[t]{\linewidth}\raggedright
% \begin{quote}
% \emph{\textbf{20 modos}}
% \end{quote}
% \end{minipage} \\
% \begin{minipage}[t]{\linewidth}\raggedright
% \begin{quote}
% \emph{\ul{Passado}}
% \end{quote}
% \end{minipage} & \begin{minipage}[t]{\linewidth}\raggedright
% \begin{quote}
% \emph{1. Ignorância}

% \emph{2. Formações --}

% \emph{kármicas}
% \end{quote}
% \end{minipage} & \begin{minipage}[t]{\linewidth}\raggedright
% \begin{quote}
% \emph{Karma -- processo}

% \emph{(}kamma-bhava\emph{)}

% \emph{5 causas do karma:}

% \emph{1,2,8,9,10}
% \end{quote}
% \end{minipage} & \begin{minipage}[t]{\linewidth}\raggedright
% \begin{quote}
% \emph{Cinco causas no passado,}
% \end{quote}
% \end{minipage} \\
% \multirow{2}{=}{\begin{minipage}[t]{\linewidth}\raggedright
% \begin{quote}
% \emph{\ul{Presente}}
% \end{quote}
% \end{minipage}} & \begin{minipage}[t]{\linewidth}\raggedright
% \begin{quote}
% \emph{3. Cognição}

% \emph{4. Mente e Corpo}

% \emph{5. As cinco bases}

% \emph{6. Impressão}

% \emph{7. Sensação}
% \end{quote}
% \end{minipage} & \begin{minipage}[t]{\linewidth}\raggedright
% \begin{quote}
% \emph{Renascer -- processo}

% \emph{(}uppatti-bhava\emph{)}

% \emph{5 resultados do karma:}

% \emph{3-7}
% \end{quote}
% \end{minipage} & \begin{minipage}[t]{\linewidth}\raggedright
% \begin{quote}
% \emph{e um fruto quíntuplo no agora}
% \end{quote}
% \end{minipage} \\
% & \begin{minipage}[t]{\linewidth}\raggedright
% \begin{quote}
% \emph{8. Anseio}

% \emph{9. Apego}

% \emph{10. Processo do}

% \emph{Renascer}
% \end{quote}
% \end{minipage} & \begin{minipage}[t]{\linewidth}\raggedright
% \begin{quote}
% \emph{Karma -- processo}

% \emph{(}kamma-bhava\emph{)}

% \emph{5 causas do karma:}

% \emph{1,2,8,9,10}
% \end{quote}
% \end{minipage} & \begin{minipage}[t]{\linewidth}\raggedright
% \begin{quote}
% \emph{Cinco causas no agora,}
% \end{quote}
% \end{minipage} \\
% \begin{minipage}[t]{\linewidth}\raggedright
% \begin{quote}
% \emph{\ul{Futuro}}
% \end{quote}
% \end{minipage} & \begin{minipage}[t]{\linewidth}\raggedright
% \begin{quote}
% \emph{11. Renascer}

% \emph{12. Degradação e morte}
% \end{quote}
% \end{minipage} & \begin{minipage}[t]{\linewidth}\raggedright
% \begin{quote}
% \emph{Renascer -- processo}

% \emph{(}uppatti-bhava\emph{)}
% \end{quote}
% \end{minipage} & \begin{minipage}[t]{\linewidth}\raggedright
% \begin{quote}
% \emph{e um fruto quíntuplo no devir.}
% \end{quote}
% \end{minipage} \\
% \end{longtable}

\begin{quote}
\emph{As ligações 1-2, conjuntamente com as 8-10, representam o processo kármico, compreendendo as cinco causas kármicas do renascer.}

\emph{As ligações 3-7, conjuntamente com as 11-12, representam o processo do renascer, compreendendo os cinco resultados kármicos.}

\emph{Paṭicca-samuppāda, lit. A génese dependente, é a doutrina do condicionamento de todos os fenómenos físicos e}

\emph{mentais. Esta doutrina, juntamente com a da impersonalidade ``não-eu'' (anattā), providencia a condição indispensável para o verdadeiro entendimento e realização do ensinamento do Buddha. Mostra que os vários processos físicos e mentais, aquilo a que por convenção se chama personalidade, homem, animal, etc., não são um mero jogo de cegas coincidências, mas sim o resultado de causas e condições. Acima de tudo, o Paṭicca-samuppāda explica como o despontar do renascer e do sofrimento depende de condições; e, na sua segunda parte, demonstra como é que através da remoção destas condições, todo o sofrimento deverá desaparecer. Assim, o Paṭiccasamuppāda serve para elucidar a segunda e a terceira nobre verdades, explicando-as desde a base das suas fundações, no sentido ascendente, e dando-lhes uma forma filosófica fixa.}

\emph{Da mesma forma, está mencionado no Paṭisambhidāmagga:}

Cinco causas existiram no passado,

Cinco frutos encontram-se na vida presente. Cinco causas produzimos no agora,

Cinco frutos colheremos na vida futura.

Vism XVII, 291

\emph{Para mais pormenores, ver Fund. III e B. Dict.}

\textbf{PENSAMENTO CORRECTO}

\emph{(Sammā-saṅkappa)}

O Segundo Factor

Ora, o que é pensamento correcto?

1. Pensamento livre de luxúria (\emph{nekkamma-saṅkappa}).

2. Pensamento livre de má-fé (\emph{avyāpāda- saṅkappa}).

3. Pensamento livre de crueldade (\emph{avihiṁsā-saṅkappa}).

A isto se chama pensamento correcto.

DN 22

PENSAMENTO CORRECTO MUNDANO E SUPRA MUNDANO

Ora, em relação ao pensamento correcto afirmo, existem

dois tipos:

1. Pensamento livre de luxúria, livre de má-fé e livre de crueldade -- isto é o chamado pensamento correcto mundano (\emph{lokiya sammāsaṅkappa}) que frutifica no mundo, e traz bons resultados.

2. Mas, seja lá o que houver a pensar, a considerar, a

reflectir, de pensamento, de raciocínio, de aplicação

-- se a mente estiver purificada, afastada do mundo e unida ao caminho, no encalço do caminho sagrado -- chama-se a estas ``operações verbais'' da mente (\emph{vaci- saṅkappa}) de pensamento correcto supra mundano (\emph{lokuttara sammā-saṅkappa}) que não é do mundo, mas supra mundano e conjunto ao caminho.

MN 117

EM CONJUGAÇÃO COM OUTROS FACTORES

Ora, quando se entende o pensamento errado como errado, e o pensamento correcto como correcto, pratica-se o \emph{entendimento correcto} (1º factor); quando se esforça para vencer o pensa- mento errado e desenvolver o pensamento correcto, pratica-se o \emph{empenho correcto} (6º factor); e quando se vence o pensamento errado com consciência correcta permanecendo na posse de pensamento correcto, pratica-se a \emph{Consciência correcta} (7º factor). Daí existirem três coisas que acompanham o pensamento correcto e que lhe procedem, respectivamente: entendimento correcto, empenho correcto e consciência correcta.

MN 117

\textbf{PALAVRA CORRECTA}

\emph{(Sammā-vācā)}

O Terceiro Factor

Ora, o que é a palavra correcta? É abster-se de mentir, abster-se de intriga, de palavra rude, de palavra presunçosa.

D 22

1. ABSTER-SE DE MENTIR

88 Neste caso uma pessoa evita mentir e abstém-se de mentir. Fala a verdade e devota-se à verdade, é honesta, merecedora de confiança, que não engana os outros. Seja numa reunião, entre pessoas, entre os seus familiares, numa sociedade, ou na corte do rei, sempre que a sua presença seja solicitada, pedindo-se-lhe que se manifeste como testemunha, dizendo o que sabe, responde, no caso em que não saiba nada: ``Eu não sei nada'', e no caso em que saiba ``Eu sei''; se não viu nada, responderá: ``Não vi nada'', e se viu, responderá: ``Eu vi''.

Desta forma nunca profere conscientemente uma mentira, seja para seu próprio bem, seja pelo bem de outro, ou pelo bem seja de quem for.

2. ABSTER-SE DE INTRIGA

Aqui uma pessoa evita a intriga e abstém-se disso. O que ouviu aqui, não repete ali, para não causar dissensão ali; e o que ouviu ali, não repete aqui, para não causar dissensão aqui. Desta forma une os que estão divididos e inspira os que estão unidos. Delicia-se na concórdia e aí sente alegria; e espalha a concórdia com a sua palavra.

3. ABSTER-SE DE PALAVRA RUDE

Aqui a pessoa evita falar com rudeza e abstém-se disso. Profere palavras que são gentis e agradáveis ao ouvido, atenciosas, palavras que tocam o coração, com cortesia, palavras amigáveis e agradáveis a muitos.

A 10:176

\emph{Em M 21, o Buddha diz: ``Ó monges, mesmo se ladrões e assassinos serrassem pelos vossos membros e articulações adentro, quem quer que se enfurecesse por causa disso não estaria a seguir o meu conselho. Por isso vós tendes de vos treinar a vós próprios: ``Imperturbável deverá manter-se a nossa mente, jamais alguma palavra maliciosa deverá escapar de nossos lábios; deveremos viver amistosos e cheios de simpatia, com o coração cheio de amor, livre de qualquer malícia escondida; e deveremos infundir essa pessoa com pensamentos de amor, pensamentos amplos, profundos, sem fronteiras, livres de cólera e de ódio''.}

4. ABSTER-SE DE PALAVRA PRESUNÇOSA

Aqui a pessoa evita falar com presunção e abstém-se disso. Fala no momento certo, de acordo com os factos, sobre aquilo que é útil; fala na Doutrina e na Disciplina; o seu discurso é como um tesouro, proferido no momento certo, acompanhado de argumentos, moderado e cheio de sentido.

A isto chama-se palavra correcta.

A 10:176

PALAVRA CORRECTA MUNDANA E SUPRA MUNDANA

Ora, eu afirmo que, o pensamento correcto é de dois tipos:

1. Abster-se de mentir, abster-se de intriga, abster-se da palavra rude e abster-se de palavra presunçosa: a isto se chama palavra correcta mundana \emph{(lokiya sammā-vācā}), que frutifica no mundo e traz bons resultados.

2. Mas evitar a prática desta palavra errada nos seus quatro aspectos, o abster-se, o desistir, o refrear-se disso -- a mente estando purificada, retirada do mundo e em conjunção com o caminho, no encalço do caminho sagrado -- a isto se chama a palavra correcta supramundana (\emph{lokuttara sammā-vācā}), que não é do mundo, mas é supramundana e conjunta ao caminho.

EM CONJUGAÇÃO COM OUTROS FACTORES

Ora, quando se entende a palavra errada como errada, e a palavra correcta como correcta, pratica-se o \emph{entendimento correcto} (1º factor); quando se efectua um esforço para vencer a palavra errada e desenvolver a palavra correcta, pratica-se o \emph{empenho correcto} (6º factor); e quando se vence a palavra errada com consciência correcta, permanecendo na posse da palavra correcta, pratica-se a \emph{consciência correcta} (7º factor). Daí existirem três coisas que acompanham a palavra correcta e que lhe procedem, respectivamente: entendimento correcto, empenho correcto e consciência correcta.

M 117

\textbf{ACÇÃO CORRECTA}

\emph{(Sammā-kammanta)}

O Quarto Factor

Ora, e o que é a acção correcta? É abster-se de matar, abster-se de roubar, abster-se de relação sexual imprópria.

DN 22

1. ABSTER-SE DE MATAR

Aqui uma pessoa evita matar seres vivos, e abstém-se disso. Ele deseja o bem de todos os seres vivos conscientemente, sem flecha ou espada, pleno de compaixão.

2. ABSTER-SE DE ROUBAR

Aqui a pessoa evita roubar e abstém-se disso; seja o que for que os outros possuam de bens e móveis na vila ou na floresta, a pessoa refreia-se da intenção de roubar.

3. ABSTER-SE DE RELAÇÃO SEXUAL IMPRÓPRIA

Aqui a pessoa evita a relação sexual imprópria e abstém-se disso. A pessoa não mantém relações sexuais com pessoas que ainda estejam sobre a protecção de pai, mãe, irmão, irmã ou familiares, nem com mulheres casadas, mulheres condenadas, e por último, nem com mulheres prometidas a outrem.

A isto chama-se acção correcta.

AN 10:176

ACÇÃO CORRECTA MUNDANA E SUPRA MUNDANA

Ora, a acção correcta, digo-vos, é de dois tipos:

1. Abster-se de matar, abster-se de roubar, e abster-se de relação sexual imprópria: a isto se chama acção correcta mundana (\emph{lokiya Sammā-kammanta}), que frutifica no mundo e traz bons resultados.

2. Mas evitar a prática desta acção errada nas suas quatro vertentes, o abster-se, o desistir, o refrear-se disso -- estando a mente purificada, retirada do mundo e em conjunção com o caminho, no encalço do caminho sagrado -- a isto chama-se acção correcta supra mundana (\emph{lokuttara sammā-vācā}), que não é do mundo, mas é supra mundana e conjunta ao caminho.

EM CONJUNÇÃO COM OUTROS FACTORES

Ora, quando se compreende a acção errada como errada, e a acção correcta como correcta, pratica-se o \emph{entendimento correcto} (1º factor); quando se efectua um esforço para vencer a acção errada e desenvolver a acção correcta, pratica-se o \emph{empenho correcto} (6º factor); e quando se vence a acção errada com a consciência correcta permanecendo na posse de acção correcta, pratica-se a \emph{consciência correcta} (7º factor). Daí existirem três coisas que acompanham a \emph{acção correcta} e que lhe procedem, respectivamente: entendimento correcto, empenho correcto e consciência correcta.

M 117

\textbf{SUSTENTO CORRECTO}

\emph{(Sammā-ājiva)}

O Quinto Factor

Ora, e o que é o sustento correcto? Quando o discípulo nobre evita uma vida errada, ganhando o seu sustento através de uma vida correcta, a isto se chama sustento correcto.

D 22

\emph{Em M 117, é dito: ``praticar fraude, traição, adivinhação,}

\emph{magia, usura: isto é sustento errado''.}

\emph{E em AN 5:177, diz: ``Há cinco negócios que devem ser evitados por um discípulo: comércio de armas, comércio de seres humanos, comércio de carne, comércio de bebidas intoxicantes e alcoólicas, e comércio de drogas ou venenos''.}

SUSTENTO CORRECTO MUNDANO E SUPRA MUNDANO

Ora, o sustento correcto, digo-vos, é de dois tipos:

1. Quando o discípulo nobre adquire o seu sustento através de uma vida correcta, evitando uma vida errada, a isto se chama sustento correcto mundano (\emph{lokiya Sammā-ājiva}), que frutifica no mundo e traz bons resultados.

2. Mas evitar o sustento errado, o abster-se, o desistir, o refrear-se disso -- estando a mente purificada, retirada do mundo e em conjunção com o caminho, no encalço do caminho sagrado -- a isto chama-se sustento correcto supra mundano (\emph{lokuttara sammā-ājiva}), que não é do mundo, mas é supra mundano e conjunto ao caminho.

EM CONJUGAÇÃO COM OUTROS FACTORES

Ora, quando se compreende o sustento errado como errado, e o sustento correcto como correcto, pratica-se o \emph{entendimento correcto} (1º factor); quando se efectua um esforço para vencer o sustento errado e desenvolver o sustento correcto, pratica-se o \emph{empenho correcto} (6º factor); e quando se vence o sustento errado com consciência correcta, vivendo na posse de sustento correcto, pratica-se a \emph{consciência correcta} (7º factor). Daí existirem três coisas que acompanham o sustento correcto e que lhe procedem, respectivamente: entendimento correcto, empenho correcto e consciência correcta.

M 117

\textbf{EMPENHO CORRECTO}

\emph{(Sammā-vāyāma)}

O Sexto Factor

Ora, e o que é o empenho correcto? Existem quatro grandes empenhos: o esforço para restringir, o esforço para abandonar, o esforço para desenvolver, e o esforço para manter.

1. O ESFORÇO PARA RESTRINGIR

\emph{(saṁvara-padhāna)}

Ora, e o que é o esforço para \emph{restringir?} Aqui o discípulo activa a sua vontade para evitar que surjam estados negativos prejudiciais; e esforça-se, põe a sua energia em acção, aplica a sua mente, e empenha-se.

Assim, quando capta uma forma com o olho, um som com o ouvido, um odor com o nariz, um sabor com a língua, uma impressão com o corpo, ou um objecto com a mente, ele nem se apega ao todo nem a nenhuma de suas partes. E luta para se defender daquilo que poderia fazer surgir estados negativos prejudiciais, cobiça e sofrimento, caso os seus sentidos não estivessem resguardados; e toma conta dos seus sentidos, domina os seus sentidos.

Em posse deste nobre domínio sobre os sentidos, vive interiormente um sentimento de alegria em que nenhum estado negativo poderá entrar.

A isto chama-se o esforço para restringir.

2. O ESFORÇO PARA ABANDONAR

\emph{(pahāna-padhāna)}

Ora, e o que é o esforço para \emph{abandonar?} Neste caso o discípulo activa a sua vontade para abandonar estados negativos prejudiciais que já tenham surgido; e esforça-se, põe a sua energia em acção, aplica a sua mente, e empenha-se.

Não retém qualquer pensamento de luxúria, má fé, ou crueldade, ou quaisquer outros estados negativos que possam ter surgido; ele abandona-os, afasta-os, destrói-os, faz com que desapareçam.

A 4:13, 14

Se surgirem pensamentos negativos no discípulo relacionados com cobiça, ódio ou ilusão, o discípulo deverá então (1) afastar-se desses objectos e dar atenção a outros relacionados com o que é mais saudável; ou, (2) reflectir na miséria desses pensamentos -- ``Estes pensamentos são na realidade doentios!

Vergonhosos! Provocam dor!''; ou, (3) não deverá prestar atenção a esses pensamentos; ou, (4) ele deverá considerar a natureza composta desses pensamentos; ou, (5) com os dentes cerrados e a língua a pressionar as gengivas, deverá restringir, suprimir, e desenraizar esses pensamentos com a sua mente; e ao proceder assim, esses pensamentos negativos de cobiça, ódio e ilusão se desvanecerão e desaparecerão; e a mente ficará interiormente mais calma e descansada, serena e concentrada.

A isto chama-se o esforço para abandonar.

M 20

3. O ESFORÇO PARA DESENVOLVER

\emph{(bhāvanā-padhāna)}

Ora, e o que é o esforço para \emph{desenvolver?} Neste caso o discípulo activa a sua vontade para promover estados saudáveis que ainda não surgiram; e ele esforça-se, põe a sua energia em acção, aplica a sua mente, e empenha-se.

Desta forma ele desenvolve os factores da iluminação (\emph{bojjhaṅga}) baseados no recolhimento solitário, no desapego e na extinção, que resultam na libertação, nomeadamente: Consciência (\emph{sati}), investigação de fenómenos (\emph{dhamma-vicaya}), energia (\emph{viriya}), êxtase-arrebatamento (\emph{piti}), tranquilidade (\emph{passaddhi}), concentração (\emph{samādhi}), e equanimidade (\emph{upekkhā}).

A isto chama-se o esforço para desenvolver.

4. O ESFORÇO PARA MANTER

\emph{(anurakkhaṇa-padhāna)}

Ora, e o que é o esforço para manter? Neste caso o discípulo activa a sua vontade para manter estados saudáveis que já tenham surgido, não permitindo que desapareçam, mas que continuem a crescer, amadurecendo até à perfeição plena do desenvolvimento (\emph{bhāvanā}); e esforça-se, põe a sua energia em acção, aplica a sua mente, e empenha-se.

Assim, por exemplo, ele mantém firmemente na sua mente um objecto favorito de concentração que tenha surgido, tal como a imagem mental de um esqueleto, um corpo infestado de larvas, um corpo azul-escuro, um corpo em putrefacção, um corpo cheio de buracos, ou um corpo inchado.

A isto chama-se o esforço para manter.

A 4:13, 14

Na verdade, para um discípulo com fé e que penetrou no Ensinamento do Mestre, é justo reflectir: ``Apesar da pele, dos tendões e dos ossos definharem completamente, apesar da carne e o sangue do corpo secarem, eu não vacilarei nos meus esforços até alcançar o que quer que seja de alcançável por empenho, perseverança e energia humanas''.

A isto chama-se o empenho correcto.

M 70

O esforço para restringir, para abandonar, Para desenvolver e para manter:

Estes quatro grandes empenhos foram demonstrados por Ele, o descendente do Sol. E aquele que firmemente se apoia neles,

Porá fim ao sofrimento.

A 6:14

\textbf{CONSCIÊNCIA CORRECTA}

\emph{(sammā-sati)}

O Sétimo Factor

Ora, e o que é a consciência correcta?

AS QUATRO FUNDAÇÕES DA CONSCIÊNCIA

\emph{(satipaṭṭhāna)}

A única forma que conduz à realização da pureza, à vitória sobre a tristeza e a lamentação, ao fim da dor e da angústia, à entrada no caminho correcto e à realização do \emph{Nibbāna}, é através das Quatro Fundações da Consciência. E quais são essas quatro fundações?

Neste caso o discípulo, depois de ter posto de lado a cobiça e a angústia, detém-se com fervor na contemplação do corpo (1), na contemplação da sensação (2), na contemplação da mente (3), e na contemplação dos objectos-mente (4), atento, compreendendo-os conscientemente.

1. CONTEMPLAÇÃO DO CORPO

\emph{(kāyānupassanā)}

Mas como é que o discípulo se detém na contemplação do corpo?

\textbf{A Consciência da Respiração}

\emph{(ānāpāna-sati)}

Neste caso o discípulo retira-se para a floresta, para o pé de uma árvore, ou para um sítio recolhido, senta-se de pernas cruzadas, dorso recto, e com atenção introspectiva, simplesmente inspira e expira atentamente. Ao efectuar uma longa inspiração, está ciente: ``Faço uma longa inspiração''; ao efectuar uma longa expiração, está ciente: ``Faço uma longa expiração.'' Ao efectuar uma curta inspiração, está ciente: ``Faço uma curta inspiração''; ao efectuar uma curta expiração, está ciente: ``Faço uma curta expiração''.

``Apercebendo-me claramente do corpo (na respiração) inteiro, eu inspiro'': assim se treina: ``Apercebendo-me claramente do corpo (na respiração) inteiro, eu expiro'': assim se treina. ``Acalmando esta função física \emph{(kāya-saṅkhāra}), eu inspiro'': as- sim se treina; ``Acalmando esta função física (\emph{kāya-saṅkhāra}), eu expiro'': assim se treina.

Assim se detém na contemplação do corpo, seja em relação à sua própria pessoa, seja em relação a outras pessoas, ou ambos os casos. Ele observa como o corpo aparece; observa como o corpo desaparece; observa o aparecer e o desaparecer do corpo. Está ali um corpo. {[}\ldots{]}

\emph{``Está ali um corpo, mas nenhum ser vivo, nenhum indivíduo, nenhuma mulher, nenhum homem, nenhum eu, e nada que pertença a um eu; nem uma pessoa, nem nada pertencendo a uma pessoa''. (Comentário do autor).}

{[}\ldots{]} Esta consciência límpida faz-se nele presente, ao ponto de permitir o conhecimento e a consciência, e vive independente, desapegado de qualquer coisa no mundo. Assim se detém o discípulo na contemplação do corpo.

D 22

\emph{``A Consciência da respiração'' (ānāpāna-sati) é um dos exercícios de meditação mais importantes. Pode ser utilizado para o desenvolvimento da tranquilidade (samatha-bhāvanā), i.e., para alcançar as quatro absorções (jhāna; ver pp. 109-113 f.); para o desenvolvimento da visão introspectiva (vipassanā-bhāvanā); ou para uma combinação das duas práticas. Dentro do contexto do Satipaṭṭhāna, a tranquilização e a concentração preparatórias para a prática da visão introspectiva, são aqui o propósito principal, podendo ser realizadas da seguinte forma:}

\emph{Depois de se ter alcançado um certo grau de calma e concentração ou uma das absorções, através da prática regular da consciência da respiração, o discípulo procede ao exame da origem da respiração. Ele vê que as inspirações e expirações são condicionadas pelo corpo que é composto pelos quatro elementos materiais e variados fenómenos corporais derivados destes, por ex., os cinco órgãos dos sentidos, etc.}

\emph{Condicionada pela impressão quíntupla dos sentidos, surge então a cognição e juntamente com esta, os outros ``três agregados da existência'', i.e., sensação, percepção, e formações mentais. Desta forma quem medita percebe claramente: ``Não existe nenhuma entidade-ego ou ``eu'' nesta chamada personalidade, mas simplesmente um processo corporal e mental condicionado por vários factores''. A este fenómeno então, ele aplica as três características, compreendendo-o ao mínimo pormenor, como impermanente, sujeito ao sofrimento e como não-eu.}

\emph{Para mais pormenores sobre ānāpāna-sati, ver MN 62, MN 118, e Vism. VIII. 145ff.}

AS QUATRO POSTURAS

Além disso, seja a andar, permanecendo de pé, sentado, ou deitado, o discípulo compreende (de acordo com a realidade) as expressões: ``vou''; ``estou em pé''; ``sento-me''; ``deito-me''; ele compreende qualquer posição do corpo.

\emph{«O discípulo compreende que não existe nenhum ser vivo, nenhum ego real que anda, que se mantém de pé, etc., mas que é por mera função figurativa da linguagem que diz: ``ando'', ``estou em pé'' e, por aí adiante».}

COMPREENSÃO LÚCIDA

\emph{(sati-sampajañña)}

Além disso, o discípulo actua com compreensão lúcida no ir e vir; ele actua com compreensão lúcida ao olhar para a frente e para trás; actua com compreensão lúcida a curvar-se ou esticar-se (qualquer parte do corpo); actua com compreensão lúcida a transportar a sua tigela de mendicante e as suas próprias vestes (\emph{hábito}); actua com compreensão lúcida a comer, a beber, a mastigar e a saborear; actua com compreensão lúcida a defecar e a urinar; actua com compreensão lúcida a caminhar, estando em pé, sentado, ao adormecer, a acordar; actua com compreensão lúcida ao falar e ao guardar silêncio.

D 22

\emph{``Em tudo o que o discípulo faz, ele compreende correctamente: (1) a sua intenção; (2) a sua vantagem; (3) o seu dever; (4) a realidade''.}

CONTEMPLAÇÃO DA AVERSÃO

\emph{(paṭikkūla-sañña)}

Continuando, o discípulo contempla este corpo coberto de pele, desde as plantas dos pés para acima, e da ponta dos cabelos para baixo, impregnado de várias impurezas: ``Este corpo tem cabelos na cabeça, pêlos no corpo, unhas, dentes, pele, carne, tendões, ossos, tutano, rins, coração, fígado, diafragma, baço, pulmões, estômago, intestinos, mesentério e excremento; bílis, muco, pus, sangue, suor, linfa, lágrimas, gordura da pele, saliva, muco nasal, líquido das articulações e urina''.

Tal como um saco aberto em ambas as extremidades, cheio de vários tipos de grãos -- incluindo arroz integral, feijão, sésamo e arroz branco -- que tivesse sido aberto por um homem que não fosse cego, para assim examinar o seu conteúdo: ``Aquilo é arroz integral, estes são feijões, isto é sésamo, isto é arroz branco'', dessa mesma forma o discípulo investiga o seu corpo.

A ANÁLISE DOS QUATRO ELEMENTOS

\emph{(dhātu)}

E continuando, o discípulo contempla este corpo em relação aos movimentos, estando em pé ou movendo-se: ``Este corpo consiste em elemento sólido, em elemento líquido, em elemento térmico e em elemento vibrante''. Tal como um açougueiro experiente ou o seu aprendiz, sentados à beira da estrada de um grande cruzamento, após terem procedido à matança de uma vaca, a dividem em diferentes partes: também, assim, da mesma forma, o discípulo contempla o seu corpo em relação aos seus elementos.

D 22

\emph{Em Vism. XI, 30, este símile é explicado da seguinte forma: Quando um açougueiro conduzindo a vaca, trá-la para o matadouro, amarra-a a um poste, mantém-na de pé, mata-a e olha para a vaca morta, ao longo de todo este tempo permanece nele a noção de ``vaca''. Mas quando já cortou a vaca morta, quando a dividiu em pedaços, e se senta ao lado para vender a carne, a noção de ``vaca'' termina na sua mente, e surge a noção de ``carne''. Não pensa que está a vender uma vaca ou que as pessoas estão a comprar uma vaca, mas antes que é a carne que é vendida e comprada. Da mesma forma, numa pessoa normal, seja monge ou leigo, os conceitos de ``ser'', ``homem'', ``personalidade'', etc., não cessarão enquanto a pessoa não dissecar mentalmente o seu corpo, seja de pé ou ao mover-se, ou enquanto não o tenha contemplado de acordo com os elementos que o compõem. Mas assim que faça este discernimento, a noção ``personalidade'', etc., desaparecerá, e a sua mente estabelecer-se-á firmemente na contemplação dos elementos.}

MEDITAÇÕES NO CEMITÉRIO

1. E continuando, tal como se um discípulo estivesse a olhar para um corpo lançado numa vala, já morto há um, dois, ou três dias, inchado, azul escuro, em plena decomposição -- da mesma forma ele considera o seu próprio corpo: ``Este meu corpo também tem esta natureza, este destino, e não lhe poderá escapar''.

2. E continuando, tal como se um discípulo estivesse a olhar para um corpo lançado numa vala, comido por corvos, falcões ou abutres, por cães ou chacais, ou devorado por todo o tipo de vermes -- assim ele considera o seu próprio corpo: ``Este meu corpo também tem esta natureza, este destino, e não lhe poderá escapar''.

3. E continuando, tal como se um discípulo estivesse a olhar para um corpo lançado numa vala, uma estrutura de ossos, a carne pendendo destes salpicada de sangue, segura entre si pelos tendões -- assim ele considera o seu próprio corpo: ``Este meu corpo também tem esta natureza, este destino, e não lhe poderá escapar''.

4. Uma estrutura de ossos, despidos de carne, salpicados de sangue, seguros entre si pelos tendões -- assim ele considera o seu próprio corpo: ``Este meu corpo também tem esta natureza, este destino, e não lhe poderá escapar''.

5. Uma estrutura de ossos, sem carne nem sangue, mas ainda seguros entre si pelos tendões -- assim ele considera o seu próprio corpo: ``Este meu corpo também tem esta natureza, este destino, e não lhe poderá escapar''.

6. Ossos, separados e espalhados por toda a parte, aqui um osso da mão, ali um osso do pé, acolá um osso do queixo, além um fémur, acolá uma pélvis, além uma coluna vertebral, ali um crânio -- assim ele considera o seu próprio corpo: ``Este meu corpo também tem esta natureza, este destino, e não lhe poderá escapar''.

7. E continuando, tal como se um discípulo estivesse a olhar para ossos numa vala de corpos, descorados como conchas -- assim ele considera o seu próprio corpo: ``Este meu corpo também tem esta natureza, este destino, e não lhe poderá escapar''.

8. Com o passar dos anos, os ossos empilhados todos juntos

-- assim ele considera o seu próprio corpo: ``Este meu corpo também tem esta natureza, este destino, e não lhe poderá escapar''.

9. Ossos gastos e desfeitos em poeira -- assim ele considera o seu próprio corpo: ``Este meu corpo também tem esta natureza, este destino, e não lhe poderá escapar''.

Desta forma se detém na contemplação do corpo, seja em relação à sua própria pessoa, ou às outras, ou ambas. Ele observa como o corpo aparece; observa como o corpo desaparece; observa o aparecer e o desaparecer do corpo.

``Ali está um corpo'': esta consciência lúcida faz-se nele presente ao ponto de permitir o conhecimento e a consciência; e vive independente, desapegado de qualquer coisa no mundo. É assim que o discípulo permanece na contemplação do corpo.

D 22

CONFIANTE NOS DEZ BENEFÍCIOS

Assim que a contemplação do corpo é posta em prática, desenvolvida, repetida frequentemente, tornando-se o seu hábito, a sua fundação, firmemente estabelecida e aperfeiçoada, o discípulo poderá entrever dez benefícios.

1. Desenvolve a maestria sobre o contentamento e o descontentamento; não sucumbe ao descontentamento e subjuga-o as- sim que este aparece.

2. Conquista o medo e a ansiedade; não sucumbe nem ao medo nem à ansiedade e subjuga-os assim que estes aparecem.

3. Aguenta o frio e o calor, a fome e a sede, o vento e o sol, ataques de moscardos, mosquitos e répteis; com paciência aguenta o discurso malévolo e malicioso que lhe seja dirigido, assim como dores que o possam assolar, mesmo que penetrantes, agudas, amargas, desagradáveis e perigosas para a vida.

4. Poderá desfrutar à vontade, sem dificuldade, sem esforço, as quatro absorções (\emph{jhāna}) que purificam a mente e conferem felicidade ainda nesta vida.

SEIS PODERES PSÍQUICOS

\emph{(abhiññā)}

5. Poderá desfrutar os diferentes poderes mágicos (\emph{iddhi-vidhā}).

6. Com o ouvido divino (\emph{dibba-sota}) purificado e supra-hu- mano, poderá ouvir ambos os tipos de som, o divino e o terreno, o distante e o próximo.

7. Com a mente poderá desenvolver a visão que pressente o coração dos outros seres (\emph{parassa-cetopariya-ñāṇa}) de outras pessoas.

8. Poderá obter a lembrança de muitos nascimentos passados (\emph{pubbenivāsānussati-ñāṇa}).

9. Com o olho divino (\emph{dibba-cakkhu}), purificado e supra- humano, poderá ver os seres a desaparecerem e a reaparecerem, o vulgo e o nobre, as pessoas bonitas e as pessoas feias, felizes e infelizes; perceberá como os seres renascem de acordo com as suas obras.

10. Através da extinção das paixões (\emph{āsavakkhaya}), poderá vir a conhecer por si próprio, mesmo nesta vida, a imaculada libertação da mente, a libertação pela sabedoria.

M 119

\emph{Os últimos seis benefícios (5-10) são os poderes psíquicos (abhiññā). Os primeiros cinco são condições mundanas (lokiya), e podem, portanto, ser realizados mesmo por uma pessoa comum (puthujjana), enquanto o último abhiññā é supramundano (lokuttara) e exclusivamente característica de um arahant, Ser Puro (Santo). Só a seguir à realização de todas as quatro absorções (jhāna) é que uma pessoa poderá alcançar com sucesso os cinco poderes psíquicos mundanos. Existem quatro iddhipāda, ou ``bases para se obterem poderes mágicos'', nomeadamente: concentração da vontade, concentração da energia, concentração da mente e concentração da investigação.}

2. CONTEMPLAÇÃO DAS SENSAÇÕES

\emph{(vedanānupassanā)}

Mas como é que o discípulo se detém na contemplação das sensações?

Ao viver sensações, o discípulo está ciente: ``tenho uma sensação agradável''; ou: ``tenho uma sensação desagradável''; ou: ``tenho uma sensação indiferente''; ou: ``tenho uma sensação mundana agradável''; ou: ``tenho uma sensação mais espiritual agradável''; ou: ``tenho uma sensação mundana desagradável''; ou: ``tenho uma sensação mais espiritual desagradável''; ou: ``tenho uma sensação mundana indiferente''; ou: ``tenho uma sensação mais espiritual indiferente''.

Desta forma ele permanece na contemplação das sensações, seja em relação a si próprio, ou a outras pessoas, ou a ambas as situações. Ele observa como as sensações aparecem; observa como elas desaparecem; observa o aparecer e o desaparecer das sensações. ``Ali há sensações'': esta consciência lúcida faz-se nele presente ao ponto de permitir o conhecimento e a consciência; e ele vive independente, desapegado de qualquer coisa no mundo. É assim que o discípulo se detém na contemplação das sensações.

D 22

\emph{O discípulo compreende que a expressão ``Eu sinto'' não tem validade a não ser como expressão convencional (vohāra-vacana); compreende que, no sentido absoluto (paramatha), só existem sentimentos, e que não existe um eu, ninguém a viver os sentimentos.}

3. CONTEMPLAÇÃO DA MENTE

\emph{(cittānupassanā)}

Mas como é que o discípulo se detém na contemplação da mente?

Neste caso o discípulo reconhece a mente gananciosa como gananciosa e a mente não gananciosa como não gananciosa; reconhece a mente que odeia como odiosa e a mente que não odeia como não odiosa; reconhece a mente iludida como iludida e a mente não iludida como não iludida. Reconhece a mente constrangida como constrangida e a mente dispersa como dispersa; reconhece a mente desenvolvida como desenvolvida e a mente não desenvolvida como não desenvolvida; reconhece a mente superável como superável e a mente insuperável como insuperável; reconhece a mente concentrada como concentrada a mente desconcentrada como desconcentrada; reconhece a mente livre como livre e a mente não livre como não livre.

D 22

\emph{Citta (mente-consciência-coração) é aqui usado como termo colectivo para os cittas, ou momentos de cognição. Citta, sendo idêntico a viññāṇa ou cognição, não devia ser traduzido como ``pensamento''.}

\emph{``Pensamento'' e ``pensar'' correspondem mais às ``operações verbais da mente'' -- vitakka (concepção-pensamento) e vicāra (pensamento discursivo), que pertencem aos saṅkhārakkhandha.}

Assim se detém na contemplação da mente, seja em relação à sua pessoa, seja em relação a outras pessoas, ou ambas. Ele observa como a mente aparece; observa como a mente desaparece; observa o aparecer e o desaparecer da mente.

``A mente está lá'': esta consciência lúcida faz-se nele presente ao ponto de permitir o conhecimento e a consciência; e ele vive independente, desapegado de qualquer coisa no mundo. É assim que o discípulo se detém na contemplação da mente.

4. CONTEMPLAÇÃO DOS OBJECTOS DA MENTE

\emph{(dhammānupassanā)}

Mas como é que o discípulo se detém na contemplação dos objectos-mente?

Neste caso o discípulo detém-se na contemplação dos objectos-mente, nomeadamente, dos cinco obstáculos.

OS CINCO OBSTÁCULOS

\emph{(pañca nīvaraṇa)}

1. Ele está ciente quando existe em si luxúria (\emph{kāmacchanda}): ``Existe em mim luxúria''; 2. Está ciente quando existe em si má-fé (\emph{vyāpāda}): ``Existe em mim má-fé''; 3. Está ciente quando existe em si torpor e preguiça (\emph{thīnamiddha}): ``Existe em mim torpor e preguiça''; 4. Está ciente quando existe em si inquietação e preocupação mental (\emph{uddhacca-kukkucca}): ``Existe em mim inquietação e preocupação mental''; 5. Está ciente quando existe em si dúvida (\emph{vicikicchā}): ``Existe em mim dúvida''.

Ele está ciente quando em si estes obstáculos estão ausentes: ``Estes obstáculos estão em mim ausentes.'' Está ciente como eles surgem; sabe como vencê-los ao surgirem; e sabe como não surgirão de novo no futuro.

D 22

\emph{Por exemplo, a luxúria surge pelo pensamento imprudente sobre o que é agradável e delicioso. Poderá ser refreado pelos seguintes seis métodos: fixando a mente numa ideia que provoca repugnância; contemplando a repugnância do corpo; dominando os próprios seis sentidos; moderando-se na comida; cultivar amizade com pessoas boas e sábias e instrução correcta. A luxúria e a má-fé são extintas definitivamente quando se alcança o estágio de anāgāmī; a inquietação é extinta quando se alcança o estágio de arahant; a preocupação mental é extinta quando se alcança o estágio de sotāpanna (entrada-na-corrente).}

OS CINCO AGREGADOS DA EXISTÊNCIA

\emph{(pañcakhanda)}

E continuando: o discípulo detém-se na contemplação dos objectos-mente, nomeadamente, dos cinco agregados da existência. Ele está ciente do que é a corporalidade (\emph{rūpa}), como aparece, como desaparece; está ciente do que é a sensação (\emph{vedanā}), como aparece, como desaparece; está ciente do que é a percepção (\emph{saññā}), como aparece, como desaparece; está ciente do que são as formações mentais (\emph{saṅkhārā}), como aparecem, como desaparecem; está ciente do que é a cognição (\emph{viññāṇa}), como aparece, como desaparece.

AS BASES DOS SENTIDOS

\emph{(saḷāyatana)}

E continuando: o discípulo detém-se na contemplação dos objectos-mente, nomeadamente, das seis bases subjectivas e objectivas dos sentidos. Ele conhece os olhos e os objectos visuais, os ouvidos e os sons, o nariz e os odores, a língua e os sabores, o corpo e as impressões físicas, a mente e os objectos-mente; e também está ciente da prisão que resulta da dependência deles. Está ciente de como a prisão aparece; está ciente de como a prisão é vencida; e está ciente de como a prisão abandonada não surgirá de novo.

OS SETE FACTORES DA ILUMINAÇÃO

\emph{(satta bojjhaṅgā)}

E continuando: o discípulo detém-se na contemplação dos objectos-mente, nomeadamente, dos sete factores da iluminação. Está ciente quando existe em si consciência (\emph{sati}), investigação dos fenómenos (\emph{dhammavicaya}), energia (\emph{viriya}), êxtase-arrebatamento (\emph{pīti}), tranquilidade (\emph{passaddhi}), concentração (\emph{samādhi}) e equanimidade (\emph{upekkhā}). Está ciente quando em si estes factores não estão estabelecidos; está ciente como eles surgem; está ciente como são plenamente desenvolvidos.

AS QUATRO NOBRES VERDADES

\emph{(cattāri ariya-saccāni)}

E continuando: o discípulo detém-se na contemplação dos objectos-mente, nomeadamente, das ``Quatro Nobres Verdades''. Está ciente, de acordo com a realidade, do que é o sofrimento; está ciente, de acordo com a realidade, do que é a origem do sofrimento; está ciente, de acordo com a realidade, do que é a extinção do sofrimento; está ciente, de acordo com a realidade, do que é o caminho que conduz à extinção do sofrimento.

Assim o discípulo detém-se na contemplação dos objectos- mente, seja com respeito a si próprio, ou com respeito a outras pessoas, ou ambas. Ele observa como os objectos-mente aparecem, observa como desaparecem, observa o aparecer e o desaparecer dos objectos-mente. ``Ali estão os objectos-mente'': esta consciência faz-se nele presente ao ponto de permitir o conhecimento e a consciência; e ele vive independente, desapegado de qualquer coisa no mundo. É assim que o discípulo se detém na contemplação dos objectos-mente.

A única forma que conduz à realização da pureza, à superação da tristeza e da lamentação, ao fim da dor e da angústia, à entrada no caminho correcto e à realização do \emph{Nibbāna}, é através destas quatro fundações da Consciência.

D 22

\emph{Estas quatro contemplações do Satipaṭṭhāna relacionam-se com todos os cinco agregados da existência, nomeadamente: (1) a contemplação do corpo relaciona-se com o agregado da corporalidade; (2) a contemplação da sensação, com o agregado da sensação; (3) a contemplação da mente, com o agregado da cognição; (4) a contemplação dos objectos-mente, com os agregados da percepção e das formações mentais.}

\emph{Para mais pormenores relacionados com o Satipaṭṭhāna, ver o comentário sobre o discurso com esse nome, traduzido em ``The Way of Mindfulness'', pelo Bhikkhu Soma (Buddhist Publication Society, 1967).}

\emph{NIBBĀNA} ATRAVÉS DE \emph{ĀNĀPĀNA-SATI}

A consciência da respiração, na inspiração e na expiração (\emph{ānāpānasati}), praticada e desenvolvida, leva as quatro fundações da consciência à perfeição; as quatro fundações da consciência, praticadas e desenvolvidas, levam os sete factores da iluminação à perfeição; os sete factores da iluminação, praticados e desenvolvidos, levam a sabedoria e a libertação à perfeição.

Mas como é que a consciência da respiração, na inspiração e na expiração, praticada e desenvolvida, leva as quatro fundações da consciência à perfeição?

I. E sempre que o discípulo (1) inspire ou expire de forma consciente e prolongada, ou (2) inspire ou expire de forma consciente e curta, ou (3) se treine inspirando e expirando enquanto se consciencializa de todo o corpo, ou (4) acalme esta função física que é a respiração -- em tal momento o discípulo detém-se na contemplação do corpo, cheio de energia, compreendendo-o conscientemente, após subjugar a cobiça mundana e a angústia. Assim, considero a inspiração e expiração como um dos fenómenos corporais.

II. E sempre que o discípulo se treine a inspirar e a expirar durante (1) o êxtase-arrebatamento (\emph{pīti}), ou (2) a alegria (\emph{sukha}), ou (3) as funções mentais (\emph{cittasaṅkhāra}), ou (4) enquanto acalma as funções mentais -- em tal momento detém-se na contemplação das sensações, cheio de energia, compreendendo-as conscientemente, após subjugar a cobiça mundana e a angústia. Assim, considero a plena consciência da inspiração e expiração como uma das sensações.

III. E sempre que o discípulo se treine a inspirar e a expirar (1) enquanto experimenta a mente, ou (2) agrada à mente, ou (3) concentra a mente, ou (4) liberta a mente -- em tal momento ele detém-se na contemplação da mente, cheio de energia, compreendendo-a conscientemente, após subjugar a cobiça mundana e a angústia. Assim, sem a compreensão lúcida, afirmo, não existe consciência alguma na inspiração ou na expiração.

IV. E sempre que o discípulo se treine a inspirar e a expirar enquanto contempla (1) a impermanência, ou (2) o desaparecimento gradual da paixão, ou (3) o extinguir, ou (4) o desapego -- em tal momento detém-se na contemplação dos objectos-mente, cheio de energia, compreendendo-os lucidamente, após subjugar a cobiça mundana e a angústia. Sabendo através da compreensão, o que é o abandono da cobiça e da angústia, ele olha em frente com plena equanimidade.

A consciência na inspiração e expiração, assim praticada e desenvolvida, conduz as quatro fundações da consciência à perfeição.

Mas como é que as quatro fundações da consciência, praticadas e desenvolvidas, levam os sete factores da iluminação à plena perfeição?

1. E sempre que o discípulo se detenha na contemplação do corpo, das sensações, da mente, e dos objectos-mente, persistente, atento, compreendendo-os lucidamente, após subjugar a cobiça mundana e a angústia, em tal momento a sua consciência é imperturbável; e sempre que a consciência esteja presente e imperturbável, em tal momento conquistou e desenvolveu o factor consciência da iluminação (\emph{sati-sambojjhaṅga}); e assim este factor da iluminação alcança a máxima perfeição.

2. E sempre que, ao deter-se com consciência, sabiamente investigue, examine, e pense sobre a doutrina (\emph{dhamma}) -- em tal momento conquistou e desenvolveu o factor investigação dos fenómenos da iluminação (\emph{dhamma vicaya-sambojjhaṅga}); e assim este factor da iluminação alcança a máxima perfeição.

3. E sempre que, ao investigar sabiamente, examinando e reflectindo sobre a doutrina, a sua energia fique firme e inabalável -- em tal momento conquistou e desenvolveu o factor energia da iluminação (\emph{viriya-sambojjhaṅga}); e assim este factor da iluminação alcança a máxima perfeição.

4. E sempre que, estando firme na sua energia, surgir dentro de si êxtase supra-sensual -- em tal momento conquistou e desenvolveu o factor êxtase da iluminação (\emph{pīti-sambojjhaṅga}); e assim este factor da iluminação alcança a máxima perfeição.

5. E sempre que, na sua compostura espiritual a sua mente fique tranquila em êxtase mental -- em tal momento conquistou e desenvolveu o factor tranquilidade da iluminação (\emph{passaddhi- sambojjhaṅga}); e assim este factor da iluminação alcança a máxima perfeição.

6. E sempre que, tranquilo na sua compostura espiritual e feliz, a sua mente fique concentrada -- em tal momento conquistou e desenvolveu o factor concentração da iluminação (\emph{samādhi-sambojjhaṅga}); e assim este factor da iluminação alcança a máxima perfeição.

7. E sempre que observe com total indiferença a sua mente assim concentrada -- em tal momento conquistou e desenvolveu o factor equanimidade da iluminação (\emph{upekkhā-sambojjhaṅga}); e assim este factor da iluminação alcança a máxima perfeição.

As quatro fundações da consciência, assim praticadas e desenvolvidas, levam os sete factores da iluminação à perfeição plena.

E como é que os sete factores da iluminação, praticados e desenvolvidos, levam a sabedoria e a libertação (\emph{vijjā-vimutti}) à perfeição plena?

Desta forma, o discípulo desenvolve os factores da iluminação: a consciência, a investigação dos fenómenos, a energia, o êxtase, a tranquilidade, a concentração e a equanimidade, baseando-se no desapego, na ausência de desejo sensual e na extinção, culminando na renúncia.

Os sete factores da iluminação, assim praticados e desenvolvidos, levam a sabedoria e a libertação à perfeição plena.

M 118

Tal como o caçador de elefantes espeta uma enorme esta- ca de madeira no chão, aí prendendo o elefante selvagem pelo pescoço, de maneira a retirar-lhe os modos e desejos habituais da selva, o desregramento da selva, a obstinação, a violência e acostumá-lo ao ambiente da vila, ensinando-lhe o bom compor- tamento como é apropriado entre os homens -- da mesma forma o discípulo nobre deverá fixar firmemente a sua mente nestas quatro fundações da consciência, de forma a expulsar de si pró- prio os costumes e desejos mundanos habituais, o desregramento habitual mundano, a obstinação, a violência, e entrar no caminho certo, realizando o \emph{Nibbāna}.

M 125

\textbf{CONCENTRAÇÃO CORRECTA}

\emph{(sammā-samādhi)}

O Oitavo Factor

Ora, e o que é a concentração correcta?

A SUA DEFINIÇÃO

Ter a mente fixa num único objecto (\emph{citteggatā}, lit. ``con-

vergência mental''): isto é concentração.

OS SEUS OBJECTIVOS

As quatro fundações da consciência (7º factor): são os objec- tivos da concentração.

OS SEUS REQUISITOS

Os quatro grandes empenhos (6º factor): são os requisitos

para a concentração.

O SEU DESENVOLVIMENTO

O praticar, o desenvolver e o cultivar destas coisas: isto é, o

desenvolvimento (\emph{bhāvanā}) da concentração.

M 44

\emph{A concentração correcta tem dois níveis de desenvolvimento: (1) ``concentração aproximada'' (upacāra-samādhi), que se aproxima da primeira absorção, sem no entanto a atingir; (2) ``concentração atingida'' (appanā-samādhi), que é a concentração presente nas quatro absorções (jhāna). Estas absorções são estados mentais que não ficam ao alcance da actividade quíntupla dos sentidos, alcançáveis somente em recolhimento e por incessante perseverança na prática da concentração. Nestes estados, é suspensa toda a actividade dos cinco sentidos. Não há nenhuma impressão auditiva ou visual que surja em tal momento, nenhuma sensação física que seja sentida. Mas apesar de parar toda a impressão exterior dos sentidos, a mente mantém-se activa, perfeitamente alerta, totalmente desperta.}

\emph{No entanto, a realização destas absorções, não é um requisito para a realização dos quatro caminhos supramundanos da sublimação; e nem a concentração aproximada nem a concentração atingida, tal como são, possuem o poder de conduzir aos quatro caminhos supramundanos; consequentemente, não possuem na realidade qualquer poder para libertar uma pessoa definitivamente do sofrimento. A realização dos quatro caminhos supramundanos só é possível no momento de profunda visão introspectiva (vipassanā) na impermanência (aniccatā), no sofrimento (dukkhatā), e na natureza desprovida de eu (anattatā) de todo este processo fenomenológico da existência. Esta introspecção, mais uma vez, só é atingida durante a concentração aproximada, não durante a concentração atingida.}

\emph{Quem tenha realizado um ou outro dos quatro caminhos supramundanos, sem alguma vez ter realizado as absorções, é chamado de sukkha-vipassaka, ou suddhavipassanā-yānika, i.e., ``aquele que somente assumiu a introspecção (vipassanā) como seu veículo (yāna)''. No entanto, aquele que alcançou um dos caminhos supramundanos após ter cultivado as absorções, é chamado de samathayānika, ou ``aquele que ganhou a tranquilidade (samatha) como seu veículo (yāna)''.}

\emph{Para samatha e vipassanā, ver Fund. IV e B.Dict.}

AS QUATRO ABSORÇÕES

\emph{(jhāna)}

Desligado de prazeres sensuais, desligado de estados negativos, o discípulo entra na primeira absorção, que é acompanhada por concepção de pensamento e pensamento discursivo, que nasce do desapego, e que é permeada de êxtase e de alegria.

D 22

\emph{Esta é a primeira das absorções que se integram na esfe- ra matéria-subtil (rūpāvacarajjhāna). É atingida quando, pelo esforço da concentração, a actividade quíntupla dos sentidos fica temporariamente suspensa e os cinco obstá- culos são igualmente eliminados.}

\emph{Ver B. Dict.: kasiṇa, nimitta, samādhi.}

107 Esta primeira absorção está livre de cinco coisas, e cinco coisas estão presentes. Quando o discípulo entra na primeira ab- sorção, desaparecem (os cinco obstáculos): luxúria, má-fé, torpor, preguiça, inquietação e dúvida; e ficam presentes: concepção de pensamento (\emph{vitaka}), pensamento discursivo (\emph{vicāra}), êxtase (\emph{pīti}), alegria (\emph{sukha}) e concentração (\emph{citt'ekaggatā = samādhi}).

M 43

\emph{Estes cinco factores mentais, presentes na primeira absorção, são chamados de factores (ou constituintes) de absorção (jhānaṅga). Vitakka (formação inicial de um pensamento abstracto) e vicāra (pensamento discursivo, congeminação) são chamados de ``funções verbais'' (vacisaṅkhāra) da mente; daí serem algo secundário à consciência sensorial. No Visuddhimagga, vitakka é comparado ao segurar-se um pote, e vicāra ao limpá-lo. Na primeira absorção, ambos estão presentes, mas exclusivamente centrados no tema da meditação; aqui, vicāra não é discursivo, mas sim de uma natureza ``exploratória''. Estão ambos totalmente ausentes nas absorções seguintes.}

Adiante: depois de diminuir a concepção de pensamento e pensamento discursivo, e ao ganhar tranquilidade interior e compostura mental, ele entra num estado livre de concepção de pensamento e de pensamento discursivo, na segunda absorção que nasce da concentração (\emph{samādhi}) e que é permeada de êxtase (\emph{pīti}) e alegria (\emph{sukha}).

\emph{Na segunda absorção, há três factores de absorção: êxtase, alegria e concentração.}

E mais: após o esmorecer do êxtase, vive em equanimidade, atento, com consciência límpida: e vive em si, aquele sentimento que os nobres costumam referir: ``Feliz é aquele que vive com equanimidade e consciente''. E assim entra na terceira absorção.

\emph{Na terceira absorção existem dois factores de absorção: alegria-equanimidade (upekkhā-sukha) e concentração (citt'ekaggatā).}

E mais: depois de abandonar o prazer e a dor, através do desvanecer da alegria e da angústia anteriores, entra num estado para além do prazer e da dor, na quarta absorção, que é purificada pela equanimidade e pela consciência.

M 43

\emph{Na quarta absorção existem dois factores de absorção: a concentração e a equanimidade (upekkhā). No Visudhimagga, são enumerados e tratados em pormenor quarenta temas de meditação (kammaṭṭhāna).}

\emph{Através da sua prática bem sucedida, as absorções a seguir referidas poderão ser realizadas. Todas as quatro: pela consciência da respiração (ver Vism. VIII, 145ff.) e pelos dez exercícios-kasiṇa (Vism. IV, V, e B.Dict.), sendo a contemplação da equanimidade (upekkhā) a prática do quarto brahma-vihāra (Vism. IX, 88-90).}

\emph{As três primeiras absorções: o desenvolvimento da gentileza amabilidade (mettā), compaixão (karuṇā), e alegria-simpatia (muditā), constituem a prática dos três primeiros brahma-vihāras (Vism. IX, 1ff.).}

\emph{A primeira absorção: através das dez contemplações na impureza (asubha-bhāvanā; i.e., as contemplações em cemitérios, que são dez conforme a enumeração no Vism. VI); e a contemplação do corpo (i.e., as trinta e duas partes do corpo; ver Vism. VIII, 42ff.).}

\emph{``Concentração aproximada'' (upacarā-samādhi): através do relembrar do Buddha, Dhamma, e Sangha; sobre a moralidade, a libertação, os seres celestiais, a paz (= Nibbāna), e a morte (Vism. VII, VIII, 1ff.); a contemplação da repugnância para com a comida (Vism. XI, 27ff.); a análise dos quatro elementos. (Vism. XI, 27ff).}

\emph{As quatro absorções imateriais (arūpa-jhāna ou āruppa), baseadas na quarta absorção, realizam-se meditando nos respectivos objectos, de onde derivam os seus nomes: as esferas do espaço sem fronteiras, da consciência sem fronteiras, do nada, e da nem-percepção-nem-não-percepção (Vism. X).}

\emph{Todos os objectos e métodos de concentração são tratados no Vism. IIIXIII; ver também Fund. IV.}

Desenvolve a tua concentração; pois quem tem concentração compreende as coisas de acordo com a realidade. E quais são? O aparecer e o desaparecer da corporalidade, da sensação, da percepção, das formações mentais e da cognição.

S 22:5

Assim, estes cinco agregados da existência devem ser sabiamente investigados; a ignorância e a ânsia devem ser sabiamente abandonadas; a tranquilidade (\emph{samatha}) e a visão introspectiva (\emph{vipassanā}) devem ser sabiamente desenvolvidas.

M 149

Este é o ``Caminho do Meio'' que o ``Ser Perfeito'' descobriu, que nos faz simultaneamente ver e saber, e que conduz à paz, ao discernimento, à iluminação, ao \emph{Nibbāna}.

S 56:11

E seguindo este caminho, porás um fim ao sofrimento.

Dhp 275
\end{quote}

\subsubsection{O Desenvolvimento Gradual do Nobre Óctuplo Caminho no Progresso do Discípulo}\label{o-desenvolvimento-gradual-do-nobre-uxf3ctuplo-caminho-no-progresso-do-discuxedpulo}

CONFIANÇA E PENSAMENTO CORRECTO

(Segundo Factor)

\begin{quote}
Suponhamos que um chefe de família, ou o seu filho, ou alguém nascido numa boa família, ouve a Doutrina; e depois de ouvir a Doutrina enche-se de confiança pelo ``Ser Perfeito''. E pleno de confiança, ele pensa: ``A vida do lar em família é cheia de dificuldades, um montão de detritos; mas a vida mendicante sem lar (de um renunciante - monge) é como o ar livre. Quando se vive a vida em família, não é fácil satisfazer todos os aspectos das regras de uma vida pura.

Deixai-me agora cortar o cabelo e a barba, envergar o hábito amarelo, e partir da vida do lar para a vida mendicante sem lar.'' E num curto espaço de tempo, desfazendo-se das suas posses, se- jam elas grandes ou pequenas, tendo abandonado um grande ou pequeno grupo de relações, ele corta o cabelo e a barba, enverga o hábito amarelo, e parte da vida do lar para a vida mendicante sem lar.

MORALIDADE

(Terceiro, Quarto e Quinto Factores)

Tendo assim deixado o mundo, o discípulo segue as regras dos monges renunciantes. Evita matar seres vivos e abstém-se disso. Sem lança nem espada, consciente, cheio de compaixão, deseja o bem de todos os seres vivos. Evita roubar, e abstém-se de tirar aquilo que não lhe é dado. Esperando até lhe ser dado, toma somente o que é dado, e vive com um coração honesto e puro. Evita a lascívia, vivendo casto, em celibato.

Evita mentir e abstém-se disso. Fala a verdade, devota-se à verdade, honesto, merecedor de confiança, não enganando as pessoas. Evita a intriga e abstém-se desta. O que ouviu aqui, não repete ali com intenção de provocar dissensão; e o que ouviu ali, não repete aqui com intenção de provocar dissensão.

Assim faz por unir aqueles que estão divididos e inspirar aqueles que estão unidos; a concórdia fá-lo feliz, delicia-se e tem pleno prazer na concórdia; e é a concórdia que ele espalha com as suas palavras. Evita a palavra rude e abstém-se desta. Diz palavras gentis, tranquilizadoras ao ouvido, amáveis, palavras que se dirigem ao coração com cortesia, amigáveis e agradáveis a todos. Evita conversa fútil, e abstém-se desta. Fala no momento certo, de acordo com os factos, fala o que é útil, fala da doutrina e da disciplina; o seu discurso é como um tesouro, proferido no momento certo, acompanhado por argumentos, moderado e pleno de sentido.

Toma a refeição uma só vez ao dia (de manhã), abstém-se de comida ao fim do dia, e não come a horas impróprias. Mantém-\/-se afastado da dança, das cantigas, da música, e de assistir a espectáculos; rejeita flores, perfumes, unguentos, assim como qualquer tipo de adorno e embelezamento. Não usa camas altas nem luxuosas. Não aceita nem ouro nem prata. Não aceita milho cru nem carne crua, mulheres nem moças, escravos nem escravas, ou cabras, ovelhas, galinhas, porcos, elefantes, vacas ou cavalos, nem terra nem bens. Não se desvia para fazer tarefas de mensageiro. Afasta-se da compra e venda de coisas. Nada tem a ver com falsas medidas, metais e pesos. Evita os esquemas tortuosos do suborno, da vigarice e da fraude. Não participa em traições, violência, prisões, ataques, saques nem em opressões.

Contenta-se com o hábito que protege o seu corpo e com a gamela com a qual sobrevive. Aonde quer que vá, está provido destas duas coisas, tal como um pássaro alado ao voar, carrega consigo as suas asas. Ao preencher este nobre domínio de moralidade (\emph{sīlakkhandha}) ele sente no seu coração uma felicidade irrepreensível.

O CONTROLO DOS SENTIDOS

(Sexto Factor)

Ora, ao perceber uma forma com o olho, um som com o ouvido ... um odor com o nariz ... um sabor com a língua uma impressão com o corpo ... um objecto com a mente, não se apega nem ao todo nem aos pormenores. E tenta evitar aquilo que, caso os sentidos estivessem desprotegidos, poderia originar o mal e estados menos saudáveis de cobiça e tristeza; ele vigia os seus sentidos, mantém os seus sentidos sob controlo. Ao praticar este nobre controlo dos sentidos (\emph{indriya-saṃvara}), sente no seu coração uma felicidade irrepreensível.

COMPREENSÃO CORRECTA

(Sétimo Factor)

Ele está desperto e age com compreensão correcta ao ir e vir; ao olhar para diante e para trás; ao dobrar e a esticar os membros; ao vestir os hábitos e a transportar a gamela; ao comer, beber, mastigar e saborear; ao defecar e urinar; ao andar e permanecer de pé, sentado, ao adormecer e acordar; ao falar e ao manter silêncio.

Ora, equipado com esta nobre moralidade (\emph{sīla}), equipado com este nobre controlo dos sentidos \emph{(indriya-saṃvara)}, e permeado desta nobre compreensão lúcida (\emph{sati-sampajañña}), escolhe um local retirado na floresta, no sopé de uma árvore, numa montanha, num rochedo, numa gruta, num cemitério, num planalto arborizado, ao ar livre, ou num monte de palha. Depois da ronda a esmolar alimento, após a refeição, senta-se de pernas cruzadas, de corpo aprumado, concentrado em si.

ABANDONANDO OS CINCO OBSTÁCULOS

Livrou-se da luxúria (\emph{kāmacchanda}); mantém-se com um

coração livre de luxúria; retira do seu coração a luxúria.

Livrou-se da má-fé (\emph{vyāpāda}); mantém-se com um coração livre de má vontade; acalentando amor e compaixão para com todos os seres vivos, retira do seu coração a má vontade.

Livrou-se do torpor e da preguiça (\emph{thīnamiddha}); mantém-se livre do torpor e da preguiça; amando a luz, de mente vigilante, com compreensão consciente, retira da sua mente o torpor e a preguiça.

Livrou-se da inquietação e preocupação (\emph{uddhacca-kukkucca}); mantendo-se com a mente imperturbada, com o coração cheio de paz, retira da sua mente a inquietação e a preocupação.

Livrou-se da dúvida (\emph{vicikicchā}); mantendo-se livre de dúvida, cheio de confiança no bem, retira do seu coração a dúvida.

AS ABSORÇÕES

(Oitavo Factor)

Pôs de lado estes cinco obstáculos, as corrupções da mente que paralisam a sabedoria. Desligado de prazeres sensuais, desligado de estados malignos, ele entra nas quatro absorções (\emph{jhāna}).

M 38

INTROSPECÇÃO

(Primeiro Factor)

Mas seja o que quer que exista de corporalidade, sensação, percepção, formações mentais, ou cognição, todos estes fenómenos reconhece-os como impermanentes (\emph{anicca}), sujeitos à dor (\emph{dukkha}), como enfermidade, como uma úlcera, um espinho, uma miséria, um fardo, um inimigo, uma perturbação, como vazio, e não-eu (\emph{anattā}); e afastando-se destas coisas, dirige a sua mente para o que é Imortal da seguinte forma: ``Isto, verdadeiramente, é a paz, isto é o mais elevado, nomeadamente, o fim de todas as formações kármicas, o abandono de todos os estágios do renascer, o desaparecer da ânsia, o desapego, a extinção, \emph{Nibbāna}.'' E alcança a extinção das paixões neste estado (\emph{āsavakkhaya}).

A 9:36

NIBBĀNA

E o seu coração torna-se livre da paixão sensual (\emph{kāmāsava}), livre da paixão pela existência (\emph{bhavāsava}), livre da paixão da ignorância (\emph{avijjāsava}). ``Estou livre!'' -- este reconhecimento surge naquele que se libertou; e sabe: ``O renascer está esgotado; a purificação realizada; o que tinha que ser feito foi feito; nada mais resta fazer neste mundo''.

M 39

Estou livre para sempre,

Este é o meu último nascimento,

Nenhuma outra existência me espera.

M 26

Isto é, sem dúvida, a paz mais pura e elevada: a destruição da cobiça, do ódio, e da ilusão.

O PENSADOR SILENCIOSO

``Eu sou'' é um pensamento presunçoso; ``Eu sou isto'' é um pensamento presunçoso; ``Eu serei'' é um pensamento presunçoso; ``Eu não serei'' é um pensamento presunçoso. Pensamentos presunçosos são uma doença, uma úlcera, um espinho. Mas de- pois de vencer todos os pensamentos presunçosos, o ser é chamado de pensador silencioso. E o pensador, o Silencioso, não aparece mais, não desaparece, não é abalado, nem anseia mais. E ao não aparecer mais, como poderá envelhecer? E ao não envelhecer mais, como poderá morrer de novo? E ao não morrer mais, como poderá ser abalado? E ao não ser mais abalado, como poderá sentir anseio?

M 140

O VERDADEIRO OBJECTIVO

Assim, o objectivo da vida pura não consiste em adquirir riqueza, honra, ou fama, nem em ganhar moralidade, concentração, ou a visão do conhecimento. Sem dúvida, o objectivo da vida pura, a sua essência, o seu propósito: é aquela inabalável libertação do coração.

M 29

E aqueles que no passado foram Puros e Iluminados, esses Abençoados também indicaram aos seus discípulos este mesmo objectivo interior, assim como foi indicado por mim aos meus discípulos. E aqueles que no futuro se purificarem e Iluminarem, esses Abençoados também indicarão aos seus discípulos este mesmo objectivo interior, assim como foi indicado por mim aos meus discípulos.

M 51

No entanto, discípulos, poderá acontecer que depois de eu partir possais pensar: ``Perdida está a doutrina do nosso mestre. Já não temos mais mestre''.

Mas não devereis pensar dessa forma; porque a Doutrina (\emph{dhamma}) e a Disciplina (\emph{vinaya}) que vos ensinei, será o vosso mestre após a minha morte.

Que o \emph{Dhamma} seja a vossa ilha!

Que o \emph{Dhamma} seja o vosso refúgio!

Não procureis outro refúgio!

Assim discípulos, as doutrinas que vos ensinei, depois de eu próprio as ter penetrado, devereis vós preservá-las e guardá-las bem, para que esta vida sagrada possa tomar o seu curso e continuar por eras, para o bem e prosperidade de muitos, como uma consolação para o mundo, pela felicidade, pelo bem e prosperidade dos seres celestiais e humanos.

D 16
\end{quote}


\backmatter

\chapter{Glossário}

\subsection{A}

\begin{glossarydescription}

\item[Anattā] Não \emph{``attan''}, no sentido de adjectivar o que não é
\emph{``attan''}, ou o que não consiste de \emph{``attan''}. Não significa a
negação em si da realidade de \emph{``attan''}, mas sim como adjectivo
aplicado ao que não o é. Doutrina budista de que toda a composição substancial
dos cinco agregados da existência e do Universo manifestado tal como o
conhecemos, está votada à desintegração total, sendo por isso impermanente,
razão pela qual o Buddha chamou a toda a realidade temporária manifestada, de
\emph{anattā}, por ser isenta de qualquer qualidade intrínseca de
\emph{``attan''}. Querendo isto dizer que a natureza de \emph{``attan''} não
se encontra no Cosmos manifestado, mas na Realidade Imortal Imanifestada.

\item[Anicca] Impermanente, inconstante, evanescente; instável.

\item[Arahant (arahat)] Digno, merecedor, respeitável, honrado, nobre; adoptado
pelos budistas para indicar aquele que alcançou o \emph{Summum Bonum} da
aspiração espiritual (\emph{Nibbāna}). (o mesmo que \emph{``ārya''}- nobre ou
\emph{``ariya"-puggala''}).

\item[Āsava] Vício, mancha, obsessão, tendências intoxicantes, corrupções,
ulceração.

\item[Asubha] Impureza, asquerosidade, sujidade, putrefacção, imundice.

\item[Attan (attā)] (Sânscrito: \emph{ātman}). (1) A alma/espírito. (2) a si
mesmo, o próprio, a própria, ele próprio, ela própria, a ti mesmo; tu mesmo; tu
próprio/a.

\item[Ahiṁsā] \emph{(avihiṁsā)} Não"-violência, ausência de crueldade, não
injuriar.

\item[Avijjā] Ignorância, desconhecimento néscio, não conhecer; sinónimo de
ilusão. No contexto budista, ignorância define"-se principalmente pelo
desconhecimento das ``Quatro Nobres Verdade'' respectivamente ``o sofrimento'',
``a causa do sofrimento'', ``a extinção do sofrimento'' e ``o caminho que conduz
à extinção do sofrimento''.

\end{glossarydescription}

\subsection{B}

\begin{glossarydescription}

\item[Bhava] Voltar a existir; o processo da existência envolvendo os três planos, nomeadamente da existência sensual, da existência matéria"-subtil e, da existência imaterial.

\item[Bhāvanā] desenvolver, cultivar mentalmente.

\item[Bojjhaṅga] Os sete factores da iluminação.

\item[Brahma"-Vihāra] As quatro ``Residências Sublimes'' ou ``Divinas'', também
chamadas de ``os quatro Estados Incondicionados'' (\emph{appamañña}), que são:
Amor Incondicional"-Gentileza"-Compreensão (\emph{mettā}), Compaixão
(\emph{karuṇā}), Alegria empática e altruísta (\emph{muditā}), Equanimidade
(\emph{upekkhā}).

\item[Buddha] O ``Desperto''. Aquele que atingiu a iluminação; que se elevou da
esfera humana pelo conhecimento e prática da verdade, um Buddha. A palavra
Buddha é um apelativo e não um nome próprio.

\end{glossarydescription}

\subsection{C}

\begin{glossarydescription}

\item[Cetanā] Volição, vontade. É um dos sete factores mentais.

\item[Citta] Mente, consciência, estado de consciência.

\end{glossarydescription}

\subsection{D}

\begin{glossarydescription}

\item[Dhamma] (Sânscrito: \emph{Dharma}) Lei, doutrina, dever. A reflexão da
lei cósmica e da qualidade intrínseca de toda a fenomenologia. Integridade,
justiça, probidade. A doutrina patente nas escrituras Budistas e sua
instrução.

\item[Diṭṭhi] (Sânscrito: \emph{dṛṣṭi}) ponto de vista, credo, dogma, teoria,
especulação, ideologia -- teoria falsa e injustificada, opinião infundada.

\item[Dosa] Ódio.

\item[Dukkha] (Sânscrito: \emph{duḥkha}) (aplica"-se tanto ao mental, como ao
físico) Sofrimento, desagradável, doloroso, que causa miséria, dificuldade,
infelicidade, defeito, prisão.

\end{glossarydescription}

\subsection{J}

\begin{glossarydescription}

\item[Jarā] Velhice, caducidade, decadência.

\item[Jāti] Nascimento.

\item[Jhāna] (Sânscrito: \emph{dhyāna}) lit. Absorção. Meditação. Refere"-se
principalmente às quatro absorções, ou abstracções meditativas. Estados subtis,
supra materiais, alcançados através da concentração na meditação, com diminuição
e suspensão da actividade sensual dos cinco sentidos e dos cinco obstáculos
(\emph{nīvaraṇa}), sendo a vitalidade elevada para um estado de plena lucidez e
vigília. Em DN I.76 lê"-se: ``com o seu coração sereno, tornado puro, translúcido,
composto, vazio de malícia, dócil, pronto para agir, firme e imperturbável''.
Vitalidade sublimada e engrandecida. \emph{Jhānas} são somente meios e não um
fim. Foi por terem apontado \emph{jhānas} como objectivo último do seu
ensinamento, que Gautama, o Buddha, rejeitou as doutrinas dos seus dois
professores. No entanto, mais tarde, o próprio Buddha confirmou a importância
dos \emph{jhānas} como meio e fase fundamental da realização do desapego do
mundo em direcção ao \emph{Nibbāna}.

\item[Jīva] A Alma. Princípio vital.

\end{glossarydescription}

\subsection{K}

\begin{glossarydescription}

\item[Kāma] Desejo dos sentidos, sensualidade subjectiva, no sentido mais
amplo do desejo ao nível dos cinco objectos dos sentidos, não exclusivamente
sexual, mas também; \emph{-chanda} (impulso): luxúria, semelhante a
\emph{kāma} no sentido amplo, mas com impulsividade adicional; \emph{-rāga}
(paixão, excitação): volúpia lasciva, aqui mais especificamente ao nível do
deleite e da indulgência sensual.

\item[Kamma] (Sânscrito: \emph{Karma}) Acção (saudável, ou prejudicial). A causa
e sua consequência, a semente e sua germinação conforme a intenção, o acto e o
resultado perante a Lei Cósmica.

\item[Kammaṭṭhāna] lit. ``O terreno de trabalho'', i.e., para meditação.

\item[Karuṇā] Compaixão. Um dos quatro \emph{Brahma"-Vihāras} (Residências Sublimes).

\item[Khandha] Agregado, substância: os cinco agregados, grupos, categorias,
substâncias, corpos -- da existência.

\item[Kilesa] Mácula, corrupção, fraquezas, qualidades prejudiciais.

\item[Kukkucca] Escrúpulo, remorso ou preocupação.

\item[Kusala] Karmicamente benéfico, saudável ou salutar.

\end{glossarydescription}

\subsection{L}

\begin{glossarydescription}

\item[Lobha] Ganância, cobiça, avareza.

\item[Lokiya] Mundano. Tudo o relacionado com o que é mundano, inclusive a
consciência e os factores mentais ainda não associados ao que é supramundano.

\item[Lokuttara] Supramundano.

\end{glossarydescription}

\subsection{M}

\begin{glossarydescription}

\item[Magga] O Caminho, ex: o ``O Nobre Óctuplo Caminho''
(\emph{aṭṭhangika"-magga}).

\item[Māna] Presunção, orgulho.

\item[Mano] Mente.

\item[Mettā] Amor, Gentileza.

\item[Moha] Ilusão. Correspondendo a ignorância, imbecilidade, desorientação e
engano.

\item[Muditā] Alegria empática e altruísta.

\end{glossarydescription}

\subsection{N}

\begin{glossarydescription}

\item[Nibbāna] (Sânscrito: \emph{Nirvāṇa}) lit. Soprar, extinguir, apagar,
``pulverizar'' o apego e o desejo aprisionante. O último e mais elevado
objectivo de todas as aspirações budistas = Libertação do desejo sensual pelo
desapego, com a ``aniquilação'' total da afirmação de vida normalmente
manifestada como cobiça, ódio e ilusão (\emph{lobha, dosa, moha}).

\end{glossarydescription}

\subsection{P}

\begin{glossarydescription}

\item[Pahāna] Abandonar, vencer.

\item[Padhāna] Esforço.

\item[Paññā] Sabedoria, conhecimento.

\item[Paramattha] A verdade mais elevada.

\item[Phala] Fruto, resultado, efeito, benefício.

\item[Phassa] Impressão dos sentidos, contacto.

\item[Pīti] Êxtase, arrebatamento, delícia, júbilo.

\item[Puthujjana] Pessoa mundana, o vulgo, o comum dos homens, o leigo ou o
monge que ainda estão aprisionados pelos dez obstáculos.

\end{glossarydescription}

\subsection{R}

\begin{glossarydescription}

\item[Rāga] Paixão, excitação.

\item[Rūpa] Corporalidade, matéria"-subtil.

\end{glossarydescription}

\subsection{S}

\begin{glossarydescription}

\item[Sakkāya] (Sânscrito: \emph{satkāya}) Individualidade, personalidade, ego.

\item[Samādhi] concentração, recolhimento, estado da mente unificada com o
objecto de meditação, integração imperturbável num único ponto.

\item[Samatha] Tranquilidade, serenidade.

\item[Sammā] Excelente, certo, correcto.

\item[Sampajañña] Compreensão, discriminação, circunspecção.

\item[Saṁsāra] lit. ``Ciclo perpétuo''. A roda dos renascimentos.

\item[Saṁvara] Restringir, evitar, dominar.

\item[Saṁyojana] Grilhão, laço, prisão, cadeia. Que prende, que amarra.

\item[Sangha] Congregação. Comunidade monástica budista.

\item[Saṅkhāra] refere"-se ao potencial formativo e criativo, ao formar ou ao
estado passivo de ``já se ter formado''. O terreno preparativo, tanto físico
como principalmente subtil, da génese ao nível da consciência, mente e
pensamento.

\item[Saṅkhata] O formado ou criado. Tudo o que seja originado ou condicionado.

\item[Saññā] Percepção.

\item[Sati] Consciência, lembrança, memória.

\item[Sīla] Moral, virtude, ética.

\item[Sīlabbata"-Parāmāsa] Apego a meras regras e rituais.

\item[Sukha] Prazer, agradável, felicidade, bênção.

\item[Suñña] Vazio.

\end{glossarydescription}

\subsection{T}

\begin{glossarydescription}

\item[Taṇhā] Sede, anseio, secura, carência, desejo.

\item[Tejo"-Dhātu] Elemento fogo, calor interior.

\item[Thīna"-Middha] Torpor, preguiça, indolência.

\end{glossarydescription}

\subsection{U}

\begin{glossarydescription}

\item[Uddhacca] Inquietação.

\item[Upekkhā] Equanimidade.

\end{glossarydescription}

\subsection{V}

\begin{glossarydescription}

\item[Vāyāma] Empenho.

\item[Vedanā] Sentimento, sensação.

\item[Vibhava] Poder, riqueza, prosperidade.

\item[Vicāra] Pensamento discursivo.

\item[Vicikicchā] Dúvida, cepticismo.

\item[Viññāṇa] Qualidade mental como constituinte da individualidade.
Consciência física, sensorial e percepção.

\item[Vipassanā] Introspecção. Visão e compreensão introspectiva.

\item[Viriya] Vigor, energia, virilidade.

\item[Vitakka] Pensamento. Concepção"-pensamento.

\item[Vyāpāda] Má"-fé, maledicência, perfídia, malícia.

\end{glossarydescription}

\chapter{Outras obras do Autor}

{\centering
  \emph{(Disponibilizadas pela Buddhist Publication Society)}
\par}

\bigskip

\textbf{The Buddha's Path to Deliverance}

Esta antologia clássica do Cânone Pāli, traça esquematicamente o curso inteiro de desenvolvimento meditativo, assim como prescrito nos textos budistas mais antigos. O autor procedeu à compilação deste trabalho, quase na sua totalidade a partir dos \emph{suttas}, delineando um manual de meditação compreensivo pelas próprias palavras do Buddha, clarificando com breves mas esclarecedoras explicações.

216 + XII páginas\hfill Ordem nº. BP 202S

\bigskip

\textbf{Buddhist Dictionary}

Desde a sua primeira publicação em 1952, este manual de termos e doutrinas
budistas tem sido um fiel companheiro e fonte de ajuda no estudo da literatura
budista. O Ven. Nyanatiloka oferece explicações autênticas e lúcidas de termos
Pāli budistas, com referência cruzada e igualmente fontes referenciadas. Este
livro servirá de ajuda tanto para o estudante sério, como para o profissional
académico de Budismo.

258 + X páginas\hfill Ordem nº BP 601S

\clearpage

\textbf{Fundamentals of Buddhism}

Neste opúsculo da edição `Wheel', o autor explica ``As Quatro Nobres Verdades'',
o \emph{karma}, o renascer, a génese dependente, e a meditação, de uma forma
bastante esclarecedora, não só para aqueles ainda pouco familiarizados com o
Budismo, mas também para aqueles que já estudam o Dhamma há bastante
tempo, mas que se sentem insatisfeitos com as versões melancólicas e
superficiais sobre o ensinamento, tão em voga nos dias de hoje.

80 páginas\hfill Ordem nº Wh 394/396


\chapter{The Buddhist Publication Society}

(Sociedade Budista de Publicações)

A BPS é uma ``\emph{Charity}'' (Instituição de Caridade) reconhecida, dedicada à divulgação do Ensinamento do Buddha, ensinamento este tão importante para as populações de todos os credos.

Fundada em 1958, a BPS publicou já um vasto número de livros, cobrindo uma grande amplitude de tópicos. As suas publicações incluem traduções dos Discursos do Buddha com anotações precisas, trabalhos de referência padrão, bem como exposições originais contemporâneas do pensamento e prática budistas. Os trabalhos apresentam o Budismo como realmente é -- uma força dinâmica que tem influenciado muitos seres nos últimos 2500 anos, e que é tão relevante hoje como o era quando surgiu. Uma lista completa das nossas publicações será enviada mediante pedido.

The Hony. Secretary

BUDDHIST PUBLICATION SOCIETY

P.O. Box 61

54, Sangharaja Mawatha

Kandy • Sri Lanka

\href{mailto:bps@ids.lk}{E-mail:} bps@ids.lk

W\href{http://www.bps.lk/}{ebsite -- http://www.bps.lk/}


\input{./manuscript/tex/copyright-details.tex}

\emptyUntilEven

\end{document}
