\chapterNote{A Nobre Verdade da Extinção do Sofrimento}

\chapter{A Terceira Nobre Verdade}

\tocChapterNote{A Nobre Verdade da Extinção do Sofrimento}

Afinal, o que significa a nobre verdade da extinção do sofrimento? É o total
desvanecimento e fim deste anseio, a sua renúncia e abandono, o seu desapego e a
sua libertação.

Mas onde pode este anseio desaparecer, onde é que poderá ser extinto?

Onde quer que existam coisas belas e agradáveis no mundo, aí se poderá desvanecer este anseio, aí poderá ser extinto.

\quoteRef{DN 22}

Seja no passado, presente ou futuro, quem de entre os monges ou sacerdotes
encarar as coisas belas e agradáveis como impermanentes (\emph{anicca}),
portadoras de infelicidade (\emph{dukkha}), e vazias de um eu (\emph{anattā}),
como doenças ou cancros, esses são os que superam o anseio.

\quoteRef{SN 12.66}

\section{A Dependência da Extinção\\ de Todos os Fenómenos}

É através do total desvanecimento e da extinção do anseio (\emph{taṇhā}), que se
extingue o apego (\emph{upādāna}); através da extinção do apego, extingue-se o
processo de retorno (\emph{bhava}); através da extinção do processo (kármico) do
retorno, extingue-se o renascer (\emph{jāti}); e através da extinção do
renascer, extingue-se a decadência e a morte, a tristeza, a lamentação, o
sofrimento, a angústia e o desespero. Assim se realiza a extinção de toda esta
carga de sofrimento.

\quoteRef{SN 12.43}

Daí o aniquilar, o cessar e o superar da corporalidade, da sensação, da
percepção, das formações mentais e da cognição -- isto é o fim do sofrimento, o
fim da doença, a vitória sobre a idade avançada e a morte.

\quoteRef{SN 22.30}

\begin{quote}
  O movimento ondulatório a que chamamos onda -- e que no observador ignorante
  gera a ilusão de uma e mesma massa de água movendo-se à superfície do lago --
  é produzido e insuflado pelo vento e mantido pelas energias acumuladas. Ora,
  depois do vento parar e não mais agitar a água do lago, as energias acumuladas
  serão gradualmente consumi- das e, consequentemente, todo o movimento
  ondulatório chegará ao fim. Da mesma forma, se não for adicionado ao fogo novo
  combustível, este extinguir-se-á, depois de consumir todo o combustível
  existente.

  Assim também, este processo dos cinco khandhas -- que cria a ilusão de uma
  entidade-ego na pessoa mundana ignorante -- é gerado e insuflado pelo anseio
  (\emph{ta\d{n}hā}) de afirmação da vida, e mantido durante certo tempo, pelas energias de
  vida acumuladas. Ora, após o combustível (\emph{upādāna}), i.e., o anseio e
  apego à vida, cessar, se nenhum anseio impulsionar de novo este processo dos
  cinco khandhas, a vida continuará enquanto ainda houver energias de vida
  acumuladas, mas com a sua destruição pela morte, o processo dos cinco khandhas
  alcançará então a extinção final.

  Assim, Nibbāna, ou ``extinção'' (Sânscrito: \emph{nirvā\d{n}a}; derivado de nir +
  vā -- parar de soprar, apagar-se) poderá ser considerado sob dois aspectos a
  citar:

  \begin{enumerate}

    \item ``Extinção das Impurezas'' (\emph{kilesa-parinibbāna}), que se
          alcança ao realizar o nível de Arahant, ou Purificação Nobre, o que
          geralmente ocorre durante o período de vida; nos Suttas é referido
          como \emph{saupādisesa-nibbāna}, i.e., ``Nibbāna com os agregados da
          existência ainda remanescentes''.

    \item ``Extinção do processo dos cinco khandhas''
          (\emph{khandha-parinibbāna}), que ocorre à morte do Arahant, referida
          nos Suttas como: \emph{anupādisesa-nibbāna}, i.e., ``Nibbāna já sem os
          agregados da existência remanescentes''.

  \end{enumerate}
\end{quote}

\section{Nibbāna}

Isto é na verdade, a paz, o mais elevado, nomeadamente o fim de todas as
formações kármicas, o renunciar de toda a forma de renascimento, o
desvanecimento do anseio, do desapego, a extinção, \emph{Nibbāna}.

\quoteRef{AN 3.32}

Extasiado na sensualidade, enfurecido pela raiva, cego pela ilusão, avassalado,
com a mente enredada, dirige-se o homem à sua própria ruína, à ruína dos outros,
à ruína de ambos, e acaba por experimentar a dor e a angústia mental. Mas, se
abandonar a sensualidade, a raiva e a ilusão, o homem não se dirige à sua
própria ruína, nem à ruína dos outros, nem à ruína de ambos, e acaba então por
não experimentar nem dor, nem angústia mental. Assim é o \emph{Nibbāna}
imediato, visível nesta vida, convidativo, cativante e compreensível aos olhos
dos sábios.

\quoteRef{AN 3.55}

A extinção da cobiça, a extinção do ódio, a extinção da ilusão, isto é, na
verdade, chamado de \emph{Nibbāna}.

\quoteRef{SN 38.1}

\section{O Arahant, o Puro (Santo)}

E para um discípulo assim liberto, em cujo coração mora a paz, nada mais há a
acrescentar ao que já foi feito, e nada mais resta ser feito. Tal como uma rocha
de massa sólida inabalável ao vento, assim também nem formas, nem sons, nem
odores, nem sabores, nem contactos de género algum, nem o desejável ou o
indesejável conseguirão perturbar tal discípulo. Firme na sua mente, ele
conquista a libertação.

\quoteRef{AN 6.55}

E aquele que reflectiu sobre todos os contrastes nesta terra, que já não se
deixa perturbar por mais nada no mundo, ``O Pacífico'', livre da raiva, da
tristeza e da saudade, esse transcendeu o nascimento e a decadência.

\quoteRef{Snp 1048}

\section{O Incondicionado}

Na verdade, existe uma dimensão, onde nem sequer existe o sólido, nem o fluido,
nem calor, nem movimento, nem este mundo, nem qualquer outro, nem sol, nem lua.

A isto eu chamo nem surgir, nem passar, nem permanecer quieto, nem nascer, nem
morrer. Não existe sequer um ponto de apoio, nem desenvolvimento, nem qualquer
base. Isto é o fim do sofrimento.

\quoteRef{Ud 8.1}

Existe um Não-nascido, Não-originado, Não-creado, Não-formado. Se não existisse
este Não-nascido, Não-originado, Não-creado, Não-formado, então a saída do mundo
do nascido, do originado, do creado e do formado, não seria possível.

Mas uma vez que existe este Não-nascido, Não-originado, Não-creado, Não-formado,
é possível sair do mundo do nascido, do originado, do creado e do formado.

\quoteRef{Ud 8.3}
