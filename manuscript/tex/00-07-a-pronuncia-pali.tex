\chapter{A PRONÚNCIA PĀḶI}

\textbf{AS VOGAIS}

% FIXME: add example words. See chanting book guide.

a - Em Português é pronunciado como `\emph{a}' mudo em `\emph{para}' ou `\emph{caneca}'. Em Português do Brasil é melhor exemplificar em inglês: O `\emph{a}' funciona como o `\emph{u}' na palavra inglesa `\emph{shut}'; nunca aberto como em `\emph{cat}', e nunca como em `\emph{take}'.

\emph{ā} - Como `á' de `já'ou `a' de `tomar'.

\emph{e} - É pronunciado como `ê' longo.

\emph{i} - Como `\emph{i}'.

\emph{ī} - Como `\emph{i}' longo.

\emph{o} - Como `\emph{ô}' longo.

\emph{u} - Como `\emph{u}'.

\emph{ū} - Como `\emph{u}' longo.

\textbf{AS CONSOANTES}

\emph{c} - É pronunciado como `\emph{tch}', assim como o `\emph{ch}' inglês em `\emph{chair}'; nunca como `\emph{c}' em `\emph{cavalo}' ou `\emph{ch}' em `\emph{cheirar}', `\emph{chover}'.

\emph{g} - Como em `\emph{gamo}'.

\emph{h} - Mesmo que colocado imediatamente a seguir às consoantes ou consoantes duplas, o `\emph{h}' é sempre aspirado como sopro em suspiro gutural, típico na língua inglesa; exemplo como no inglês: `\emph{bh}' `\emph{cabhorse}'; `\emph{ch}' como

`\emph{chh}' em `\emph{ranch-house}'; `\emph{dh}' como em `\emph{handhold}'; `\emph{gh}' como em `\emph{bag-handle}'; `\emph{jh}' como `\emph{dgeh}' em `\emph{sledgehammer}', etc.

\emph{j} - Não como em `\emph{jarra}', mas como `\emph{dj}' de `\emph{djarra}'; como na palavra inglesa `\emph{joy}'.

\emph{ṁ} - O chamado `nasal' é como o `m' em `\emph{amparo}' ou, `\emph{ambiente}'.

\emph{s} - Sempre como em `\emph{sublimar}' ou em `\emph{se}'; nunca como `\emph{z}', ex: em `\emph{causar}' ou em `\emph{físico}'.

\emph{ñ} - Como o `\emph{nh}' normal na língua portuguesa; ex: `\emph{manhã}', `\emph{minha}', `\emph{apanhar}', etc.

\emph{ph} - Como `\emph{f}' seguido de suspiro gutural como no inglês, assim como o `\emph{ph}' da palavra inglesa `\emph{haphazard}'.

\emph{th} - Como `\emph{t}' seguido de suspiro gutural típico do `\emph{h}' em inglês.

\emph{y} - Como o `\emph{i}' normal.

\emph{ṭ, ṭh, ḍ, ḍh} - São sons de língua, ditos cerebrais; ao pronunciá-los deve-se pressionar a língua contra o céu da boca.

\emph{Consoantes duplas} - Cada uma deve ser pronunciada, como `\emph{bb}' em `\emph{subbase}'.
