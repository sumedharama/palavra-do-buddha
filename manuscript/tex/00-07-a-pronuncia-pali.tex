\chapter{A Pronúncia Pāli}
\label{pron-pali}

\subsection{As Vogais}

\begin{tabular}{@{} L{5mm} L{\linewidth-5mm}}
\emph{a} & Em Português é pronunciado como \emph{a} mudo em \emph{par\prul{a}} ou \emph{c\prul{a}nec\prul{a}}.\\

\emph{ā} & Longo, como em \emph{tom\prul{a}r, c\prul{a}sa, m\prul{a}la, \prul{á}gua}.\\

\emph{e} & Como em \emph{p\prul{ê}ra, m\prul{e}sa, m\prul{ê}s}.\\

\emph{i} & Curto, como em \emph{v\prul{i}da, l\prul{i}vro, f\prul{i}lho}.\\

\emph{ī} & Longo, como em \emph{Mar\prul{i}a, melanc\prul{i}a}.\\

\emph{o} & Como em \emph{av\prul{ô}, b\prul{o}lo, fl\prul{o}r}.\\

\emph{u} & Curto, como em \emph{p\prul{u}reza, m\prul{u}ndo}.\\

\emph{ū} & Longo, como em \emph{m\prul{ú}sica, d\prul{ú}vida}.\\
\end{tabular}

\subsection{As Consoantes}

\enlargethispage{\baselineskip}

\begin{tabular}{@{} L{5mm} L{\linewidth-5mm}}
\emph{c} & É pronunciado como \emph{tch}, assim como o \emph{ch} inglês em \emph{chair}; nunca como \emph{c} em \emph{cavalo} ou \emph{ch} em \emph{cheirar}, \emph{chover}.\\

\emph{g} & Como em \emph{gamo}.\\

\emph{h} & Mesmo que colocado imediatamente a seguir às consoantes ou consoantes duplas, o \emph{h} é sempre aspirado como sopro em suspiro gutural, típico na língua inglesa; exemplo como no inglês:\\
\end{tabular}

\begin{tabular}{@{} L{5mm} L{5mm} L{\linewidth-10mm}}
& \emph{bh} & \emph{ca\prul{bh}orse}\\

& \emph{ch} & \emph{ran\prul{ch-h}ouse}\\

& \emph{dh} & \emph{han\prul{dh}old}\\

& \emph{gh} & \emph{ba\prul{g-h}andle}\\

& \emph{jh} & \emph{sle\prul{dgeh}ammer}\\
\end{tabular}

\clearpage

\begin{tabular}{@{} L{5mm} L{\linewidth-5mm}}
\emph{j} & Não como em \emph{jarra}, mas como na palavra inglesa \emph{\prul{j}oy}.\\

\emph{ṁ} & O chamado ‘nasal’ é como \emph{e\prul{ng}anar}, ou \emph{so\prul{ng}} em inglês.\\

\emph{s} & Sempre como em \emph{\prul{s}ublimar} ou em \emph{\prul{s}e}; nunca como \emph{z} em \emph{causar} ou em \emph{físico}.\\

\emph{ñ} & Como o \emph{nh} normal na língua portuguesa; ex: \emph{ma\prul{nh}ã}, \emph{mi\prul{nh}a}, \emph{apa\prul{nh}ar}, etc.\\

\emph{ph} & Como \emph{p} seguido de suspiro gutural, como nas palavras \emph{ha\prul{ph}azard, a\prul{pp}ear}.\\

\emph{th} & Como \emph{t} seguido de suspiro gutural nas palavras \emph{\prul{th}at, \prul{th}ese}.\\

\emph{y} & Como o \emph{i} em \emph{ma\prul{i}o, sa\prul{i}a, pra\prul{i}a}.\\
\end{tabular}

\bigskip

{\raggedright

\emph{ṭ, ṭh, ḍ, ḍh:} São sons produzidos com a língua, ditos cerebrais; ao pronunciá-los deve-se pressionar a língua contra o céu da boca.

\emph{Consoantes duplas:} Cada uma deve ser pronunciada, como \emph{bb} em \emph{subbase}.

}
