\chapterNote{A Nobre Verdade da Origem do Sofrimento}

\chapter{A Segunda Nobre Verdade}

\tocChapterNote{A Nobre Verdade da Origem do Sofrimento}

Afinal, o que significa a nobre verdade da origem do sofrimento? É o anseio que enlaçado pelo prazer e pela sensualidade, provoca o renascimento, logo encontrando renovado deleite, ora aqui, ora acolá.

\section{O Triplo Anseio}

Existe o anseio sensual (\emph{kāma"-taṇhā}); o anseio pela existência (eterna) (\emph{bhava"-taṇhā}); o anseio pela auto"-aniquilação (\emph{vibhava"-taṇhā}).

\quoteRef{DN 22}

\begin{quote}
Anseio sensual (\emph{kāma"-ta\d{n}hā}) é o anseio por desfrutar do
prazer dos objectos dos cinco sentidos.

Anseio pela existência (\emph{bhava"-ta\d{n}hā}) é o anseio pela vida eterna ou
contínua, mais particularmente a vida naqueles mundos superiores chamados
existências de matéria subtil e existências imateriais (\emph{rūpa"-bhava}, e
\emph{arūpa"-bhava}). Está estreitamente relacionada com a chamada crença na
eternidade (\emph{bhava"-di\d{t}\d{t}hi} ou \emph{sassata"-di\d{t}\d{t}hi}), i.e.,
a crença num eu absoluto e eterno, que persiste independentemente do corpo.

O anseio pela auto"-aniquilação (lit., “pela não existência”,
\emph{vibhava"-ta\d{n}hā}) é o resultado do crer na aniquilação
(\emph{vibhava"-di\d{t}\d{t}hi} ou \emph{uccheda"-di\d{t}\d{t}hi}), i.e., a noção
materialista ilusória de um “eu” mais ou menos real que se aniquila no momento
da morte, não permanecendo nenhuma relação casual com o tempo, antes e depois da
morte.
\end{quote}

\section{A Origem do Anseio}

Mas onde nasce e ganha raiz este anseio? Onde quer que no mundo existam coisas
adoráveis e agradáveis, este anseio surge e ganha raiz. Os olhos, os ouvidos, o
nariz, a língua, o corpo e a mente, transmitem prazer e agrado: aí este anseio
surge e ganha raiz.

Os objectos visuais, os sons, os cheiros, os sabores, as impressões corporais e
os objectos da mente são belos e agradáveis: aí este anseio surge e ganha raiz.

A consciência, a impressão sensorial e a sensação nascida da impressão
sensorial, da percepção, da vontade, do anseio, do pensamento e da reflexão, são
belas e agradáveis: aí este anseio surge e ganha raiz.

\enlargethispage{\baselineskip}

Esta é chamada a nobre verdade da origem do sofrimento.

\quoteRef{DN 22}

\section{A Génese Dependente de Todos os Fenómenos}

Sempre que alguém percepcione um objecto visual, som, odor, sabor, impressão
corporal, ou um objecto mental, se o objecto for agradável, sentirá atracção; se
o objecto for desagradável, sentirá repulsa.

Assim, qualquer tipo de sensação (\emph{vedanā}) que seja experimentada por
alguém -- agradável, desagradável ou indiferente -- se a pessoa aprovar,
acalentar e se apegar a essa sensação, ao fazê"-lo, surge o anseio; mas o anseio
por sensações significa apego (\emph{upādāna}), e é do apego que depende o
(presente) processo de retorno; por sua vez é do processo de retorno
(\emph{bhava}; neste caso \emph{kamma"-bhava}, processo kármico) que depende o
(futuro) nasci- mento (\emph{jāti}); e a decadência, a morte, a tristeza, a
lamentação, a dor, a angústia e o desespero, assentam no nascimento. Desta forma
surge toda esta carga de sofrimento.

\quoteRef{MN 38}

\begin{quote}
  A fórmula da génese dependente (\emph{pa\d{t}icca"-samuppāda}) da qual só algumas das
  doze correspondências foram mencionadas na passagem anterior, pode ser
  entendida como uma explicação minuciosa da segunda nobre verdade.
\end{quote}

\section{Os Resultados-Kármicos Presentes}

Na realidade, devido ao anseio dos sentidos, condicionados pelo anseio dos
sentidos, impelidos pelo anseio dos sentidos, completamente movidos pelo anseio
dos sentidos, reis lutam contra reis, príncipes contra príncipes, padres contra
padres, cidadãos contra cidadãos; a mãe discute com o filho, o filho com o pai;
o irmão discute com o irmão, o irmão com a irmã, amigo com o amigo.

Assim, entregues à dissensão, à implicância e ao desacato, atiram"-se uns aos
outros com punhos cerrados, paus e armas. E, por conseguinte, acabam por sofrer
dor mortal ou morte.

E mais ainda, devido ao anseio dos sentidos, condicionadas pelo anseio dos
sentidos, impelidas pelo anseio dos sentidos, completamente movidas pelo anseio
dos sentidos, as pessoas arrombam casas, roubam, saqueiam, e cometem sérios
assaltos na rua e na estrada e seduzem as mulheres do alheio. Então, os
governantes mandam prender estas pessoas, infligindo"-lhes várias medidas de
punição. E, por isso, acabam por incorrer na dor mortal e na morte. Ora, isto é
a miséria do anseio dos sentidos, o acumular do sofrimento nesta vida presente
devido ao anseio dos sentidos, condicionado pelo anseio dos sentidos, gerado
pelo anseio dos sentidos, totalmente dependente do anseio dos sentidos.

\quoteRef{MN 13}

\section{Os Resultados-Kármicos Futuros}

E nesta sequência, as pessoas seguem o caminho do mal por acções, palavras e
pensamentos; e ao irem pelo caminho do mal por acções, palavras e pensamentos,
no momento da desintegração do corpo, após a morte, caem num estado involutivo
de existência, num estado de sofrimento, num destino infeliz, nos abismos do
inferno. Mas esta é a miséria do anseio dos sentidos, o acumular de sofrimento
na vida futura devido ao anseio dos sentidos, condicionado pelo anseio dos
sentidos, gerado pelo anseio dos sentidos, totalmente dependente do anseio dos
sentidos.

\quoteRef{MN 13}

\begin{verse}
  Nem no ar, nem no meio do oceano,\\
  Nem escondido nas frestas da montanha,\\
  Em sítio algum se encontra um lugar na Terra,\\
  Onde o homem esteja livre de más acções.

  \quoteRef{Dhp 127}
\end{verse}

\section{O Karma Como Volição}

É à volição (\emph{cetanā}) que chamo \emph{“Karma”} (acção). Por se ter
desejado, age"-se com o corpo, com a fala, com a mente.

Há acções (\emph{kamma}) a amadurecer nos infernos\ldots{} a amadurecer no reino
animal\ldots{} a amadurecer no domínio dos espíritos\ldots{} a amadurecer entre
os homens\ldots{} a amadurecer em mundos celestiais.

O resultado das acções (\emph{vipāka}) é de três tipos: amadurecimento na vida
presente; na próxima; ou em vidas futuras.

\quoteRef{AN 6.63}

\section{A Herança das Acções}

Todos os seres são os responsáveis pelas suas acções (\emph{kamma, Skr: karma}),
herdeiros das suas acções: as suas acções são o útero de onde brotam, eles
aprisionam"-se com as suas acções, as suas acções são o seu refúgio.

Quaisquer acções que façam -- boas ou más -- eles serão os seus herdeiros.

\quoteRef{AN 10.206}

E onde quer que surjam os seres na existência, é aí que as suas acções
amadurecerão; e onde quer que amadureçam as suas acções, é aí que ganharão os
frutos dessas acções, nesta vida e nas futuras.

\quoteRef{AN 3.33}

Virá um tempo, em que o poderoso oceano secará, desaparecerá, e não mais
existirá. Virá um tempo em que a poderosa Terra será devorada pelo fogo,
perecerá, e não mais existirá. Mas mesmo assim, não haverá fim para o sofrimento
dos seres que, obstruídos pela ignorância, e enganados pelo anseio, se apressam
e se precipitam ao longo deste ciclo de renascimentos.

\quoteRef{SN 22.99}

\clearpage

\begin{quote}
  O anseio (\emph{ta\d{n}hā}), não é, no entanto, a única causa da má acção, nem por
  conseguinte, de todo o sofrimento e miséria produzidos desta forma, nesta e na
  próxima vida; mas onde quer que haja anseio, é aí que, na dependência desse
  anseio, poderá surgir inveja, raiva, ódio, e muitos outros males que geram a
  infelicidade e a miséria. E todos estes impulsos e acções egoístas de
  afirmação da vida, juntamente com os diversos tipos de miséria gerados, agora
  ou posteriormente, e mesmo todos os cinco agregados de fenómenos que
  constituem a vida -- está tudo basicamente enraizado na cegueira e na
  ignorância (\emph{avijjā}).
\end{quote}

\section{Karma}

\begin{quote}
  A segunda nobre verdade também ajuda a explicar as causas das aparentes
  injustiças na natureza, ensinando que nada no mundo se pode manifestar sem
  razão ou causa e que, não só as nossas tendências latentes, mas todo o nosso
  destino, toda a boa e má sorte, provêm de causas que devemos procurar, em
  parte, nesta vida e, em parte, em vidas passadas.

  Estas causas são as actividades de afirmação da vida (\emph{kamma, Skr:
    karma}) produzidas pelo corpo, pela fala e pela mente. O carácter e o
  destino de todos os seres é assim determinado por esta tripla acção.

  O \emph{karma}, definido com exactidão manifesta essas volições boas e más
  (\emph{kusala"-akusala"-cetanā}), causando o renascer. Assim sendo, a
  existência, ou melhor, o proceder do retorno (\emph{bhava}) consiste num
  processo kármico activo e condicionante (\emph{kamma"-bhava}) e no seu
  resultado, o processo do renascer (\emph{upapatti"-bhava}).

  Igualmente, ao considerarmos o karma, não nos devemos esquecer da natureza
  impessoal (\emph{anattatā}) da existência. No caso de um maremoto por exemplo,
  não é a mesma onda que se apressa à superfície do oceano, mas sim o movimento
  de diferentes massas de água consideráveis. Da mesma forma se deve compreender
  que não existem entidades"-ego reais precipitando"-se através do oceano do
  renascimento, mas simplesmente ondas"-vida, que, de acordo com a sua natureza e
  actividades (boas ou más), se manifestam ora aqui como seres humanos, ora
  acolá como animais, e noutros lugares como seres invisíveis.

  De novo se deve enfatizar o facto de que, correctamente falando, o termo
  “\emph{karma}” significa simplesmente os tipos de acção em si, já
  anteriormente referidos e não significa nem inclui os seus resultados.

  Para mais pormenores sobre o \emph{karma} ver \emph{Fund.} e \emph{B. Dict}.
\end{quote}
