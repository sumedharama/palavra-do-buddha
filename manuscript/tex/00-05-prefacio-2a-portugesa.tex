\chapter[Prefácio à Segunda Edição Portuguesa (2013)]{Prefácio\\ à Segunda Edição Portuguesa\\ (2013)}

Uma tradução é sempre delicada no que respeita a preservar o sentido e
significado original que o autor quis transmitir, principalmente quando envolve
um Ensinamento milenar, como o do Buddha. Este livro em particular, inclui
transcrições directas das Escrituras Budistas Theravada, sendo esta uma tradução
para a língua portuguesa da versão inglesa anteriormente traduzida do original
alemão. Por conseguinte, exigiu um cuidado extra com relação aos princípios
fundamentais do Dhamma, no sentido de evitar deturpações do seu significado
original, uma vez que contém precisamente transcrições directas parciais de
alguns dos principais \emph{Suttas} do Cânone Pāli.

Após várias considerações, procedeu"-se ao ajuste de alguns dos termos chave do
Budismo, como o caso mais pertinente do termo \emph{Sati} na língua Pāli,
traduzido normalmente para o inglês como “mindfulness” e para o português
directamente do inglês como “plena atenção”, não fazendo a justiça devida ao
significado original mais amplo que o termo \emph{Sati} comporta. Neste trabalho
decidiu"-se então, para traduzir melhor esse significado mais amplo de
\emph{Sati}, adoptar"-se a palavra “consciência” em vez de meramente “plena
atenção”, evitando assim a deturpação e o reducionismo do seu significado. De
uma forma mais abrangente, o termo português “consciência”, abarca
simultaneamente os quatro termos em inglês, “consciousness”, “conscience”,
“mindfulness” e “awareness”, que traduzem mais completamente as noções
ausentes nos termos “plena atenção” de uma consciência já espiritual, não só
de atenção mesmo que plena no seu limite cognitivo em termos de memória,
inteligência e consciência - a presença e lembrança correcta do que é importante
e se deve fazer e como se deve fazer, saudável e correctamente, diligentemente,
com responsabilidade humana e espiritual. Significa a forma cuidada com que se
usa, aplica e presta atenção, mais do que a atenção propriamente dita, que por
si só, na óptica da Palavra do Buddha, pode ser aplicada e usada tanto sábia e 
conscientemente como néscia e inconscientemente, até de uma forma criminosa. 
Daí os termos \emph{yoniso manasikāra} (atenção correcta, sabiamente aplicada) 
versus \emph{ayoniso manasikāra} (atenção incorrecta, mal aplicada).

Por sua vez na cultura e língua inglesa, o uso dos termos “Consciousness”
versus “Conscience” apresentam uma distinção de significado em que o
significado o primeiro se circunscreve mais aos sentidos físicos humanos, a
função cognitiva física e sensorial do corpo, que em Pāli se traduz por
\emph{Viññāṇa}, termo que neste livro se traduziu para o português como
“cognição”, fazendo aqui também jus à raiz sânscrita do Pāli \emph{Viññāṇa}.
“Conscience” na cultura e língua inglesa, traduz mais o aspecto ético que se
aproxima mais do significado de \emph{sati} no que respeita ao princípio do
cuidado, da diligência e da consciência, num todo de princípios éticos e
qualidades fundamentais que no Ensinamento do Buddha ajudam a conduzir"-nos ao
caminho da purificação e realização espiritual.

Sendo este só um exemplo, entre vários termos, tentou"-se assim ajustar da melhor
forma as diferentes expressões usadas, no sentido de respeitar tanto o Pāli
original, como o próprio autor, o Venerável Nyanatiloka Mahāthera.

Os trechos em itálico Sans Serif encontrados ao longo do livro, são os comentários
do autor, à excepção dos termos Pāli também em itálico.

A fundamentação dos significados Pāli, foi apoiada com a colaboração de outros
monásticos versados em Pāli e com o suporte adicional dos dicionários
‘Pāli"-English Dictionary’ de T. W. Rhys Davids e William Stede, bem como
‘Buddhist Dictionary’ de Nyanatiloka Mahāthera.

\bigskip

{\raggedleft
  Dhammiko Bhikkhu\\
  2013
\par}
