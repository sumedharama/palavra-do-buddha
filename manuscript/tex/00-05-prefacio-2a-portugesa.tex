\chapter[Prefácio à Segunda Edição Portuguesa (2013)]{Prefácio\\ à Segunda Edição Portuguesa\\ (2013)}

Uma tradução é sempre delicada no que respeita a preservar o sentido e
significado original que o autor quis transmitir, principalmente quando envolve
um Ensinamento milenar, como o do Buddha. Este livro em particular, inclui
transcrições directas das Escrituras Budistas Theravada, sendo esta uma tradução
para a língua portuguesa da versão inglesa anteriormente traduzida do original
alemão. Por conseguinte, exigiu um cuidado extra com relação aos princípios
fundamentais do Dhamma, no sentido de evitar deturpações do seu significado
original, uma vez que contém precisamente transcrições directas parciais de
alguns dos principais \emph{Suttas} do Cânone Pāli.

Após várias considerações, procedeu"-se ao ajuste de alguns dos termos chave do
Budismo, como o caso mais pertinente do termo \emph{Sati} na língua Pāli,
traduzido normalmente para o inglês como “mindfulness” e para o português
directamente do inglês como “plena atenção”, não fazendo a justiça devida ao
significado original mais amplo que o termo \emph{Sati} comporta, que se caracteriza
pela descriminação e memória do que se aprendeu, do bem e do mal, do que é saudável e
nefasto (kusala e akusala), caracterísca esta ausente da matriz puramente cognitiva da 
“plena atenção”. Por isso, adoptou"-se aqui a palavra “consciência” em vez de meramente
“plena atenção”, evitando-se assim a deturpação e o reducionismo do seu significado. 
“Consciência” em português, corresponde simultaneamente aos quatro termos em inglês
“consciousness”, “conscience”, “mindfulness” e “awareness”, traduzindo de uma forma 
abrangente, o significado da consciência diligente no sentido humano e espiritual, 
imbuída da memória e inteligência fundamentais à correcta lembrança daquilo que é 
importante decidir e fazer com responsabilidade humana, como o Buddha disse, para
cultivar mais e mais \emph{Sati} (consciência). Por \emph{Sati} enfatiza-se a forma
cuidada com que se usa, aplica e presta atenção, mais do que a atenção propriamente
dita que por si só, na óptica da Palavra do Buddha, pode ser aplicada e usada tanto 
conscientemente como inconscientemente, até de forma criminosa. Daí os termos 
\emph{yoniso manasikāra} (atenção correcta, sabiamente aplicada) 
versus \emph{ayoniso manasikāra} (atenção incorrecta, mal aplicada).

Por sua vez na cultura e língua inglesa, o uso dos termos “Consciousness” versus “Conscience”
apresentam uma distinção de significado em que o primeiro se circunscreve mais aos sentidos 
físicos humanos, a função cognitiva física e sensorial do corpo, que em Pāli se refere por
\emph{Viññāṇa}, neste livro traduzido para o português como “cognição”, 
aqui fazendo jus também à raiz sânscrita do Pāli \emph{Viññāṇa} - \emph{vijñāna}.
“Conscience” por sua vez, na cultura e língua inglesa, já traduz mais o aspecto
ético que se aproxima do significado de \emph{sati} com respeito ao princípio do
cuidado, da diligência e da consciência, congregando qualidades fundamentais que no 
Ensinamento do Buddha ajudam a conduzir"-nos ao caminho da purificação e realização espiritual.

Sendo este o exemplo mais pertinente, entre vários termos, tentou"-se assim ajustar da melhor
forma as diferentes expressões usadas, no sentido de respeitar tanto o Pāli
original, como o próprio autor, o Venerável Nyanatiloka Mahāthera.

Os trechos em itálico Sans Serif encontrados ao longo do livro, são os comentários
do autor, à excepção dos termos Pāli também em itálico.

A fundamentação dos significados Pāli, foi apoiada com a colaboração de outros
monásticos versados em Pāli e com o suporte adicional dos dicionários
‘Pāli"-English Dictionary’ de T. W. Rhys Davids e William Stede, bem como
‘Buddhist Dictionary’ de Nyanatiloka Mahāthera.

\bigskip

{\raggedleft
  Dhammiko Bhikkhu\\
  2013
\par}
