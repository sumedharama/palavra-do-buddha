\chapter{ABREVIATURAS}

D. - \emph{Dīgha Nikāya} (o número refere-se ao \emph{Sutta}).

M. - \emph{Majjhima-Nikāya} (o número refere-se ao \emph{Sutta}).

A. - \emph{Aṅguttara-Nikāya} (o primeiro número refere-se à divisão principal em partes ou Nipātas; o segundo número, ao \emph{Sutta}).

S. - \emph{Saṁyutta-Nikāya}. (o primeiro número refere-se à divisão em subgrupos (\emph{Saṁyutta}), e.g. \emph{Devatā-Saṁyutta} = I, etc.; o segundo número refere-se ao \emph{Sutta}).

Dhp. - \emph{Dhammapada} (o número refere-se ao verso).

Ud. - \emph{Udāna} (o primeiro número refere-se aos capítulos, o segundo número ao \emph{Sutta}).

Snp.- \emph{Sutta-Nipāta} (o número refere-se ao verso).

VisM. - \emph{Visuddhi-Magga (A Senda da Purificação)} (por capítulo e secção).

B. Dict. - \emph{Buddhist Dictionary (Dicionário Budista)}, por

Nyanatiloka Mahāthera.

Fund. - \emph{Fundamentals of Buddhism (Fundamentos do Budismo)},

por Nyanatiloka Mahāthera.
