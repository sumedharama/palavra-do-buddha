\chapter{INTRODUÇÃO}

I. O BUDDHA

\begin{quote}
O BUDDHA ou Iluminado -- lit. Aquele que sabe ou o Desperto -- é o nome honorífico conferido ao Sábio indiano, Gotama, que desvendou e proclamou ao mundo a lei da libertação, conhecida no Ocidente pelo nome de Budismo.

Nasceu no Século VI a.C., em Kapilavatthu, filho do rei que na época regia o País Sakya, um principado situado na zona de fronteira com o actual Nepal. O seu nome próprio era Siddhattha e seu nome de clã Gotama (Sânscrito: \emph{Siddhārtha Gautama}). Aos 29 anos de idade, renunciou ao esplendor da sua vida principesca como herdeiro real, e tornou-se um asceta mendicante, com o propósito de descobrir uma solução para aquilo que antes havia reconhecido como um mundo de sofrimento. Depois de uma busca de seis anos sob a orientação de vários instrutores religiosos e de um período de auto-mortificação infrutífera, Siddhattha finalmente alcançou a Iluminação Perfeita (\emph{sammāsambodhi}), debaixo da árvore \emph{Bodhi} em Gayā (actualmente Boddh-Gayā). Seguiram-se quarenta e cinco anos de incansável ensinamento e pregação, e finalmente, no seu octogésimo ano de vida, morre em Kusinara ``aquele ser não iludido que surgiu para a bênção e alegria do mundo''.

O Buddha não é nem um deus nem um profeta, nem a encarnação de um deus, mas um ser humano supremo que, através do seu próprio empenho, alcançou a redenção final, a sabedoria perfeita, tornando-se ``o mestre sem par de deuses e homens''. É um ``Salvador'' unicamente no sentido em que mostra aos homens como se salvarem a si próprios, seguindo até ao fim, na prática, o caminho percorrido e mostrado por ele. O Buddha, na sua consumada harmonia de sabedoria e compaixão, encarna o ideal universal e intemporal do homem Aperfeiçoado.
\end{quote}

II. O DHAMMA

\begin{quote}
O DHAMMA -- é o Ensinamento da Libertação total, tal como foi desvendado, realizado e proclamado pelo Buddha. Tem sido transmitido na antiga língua Pāḷi e preservado em três grandes colecções de livros, chamados \emph{Ti-Piṭaka}, os ``Três Cestos'', nomeadamente: (I) o \emph{Vinaya-piṭaka}, ou a Colecção da Disciplina, contendo as regras da ordem monástica; (II) o \emph{Sutta- piṭaka}, ou a Colecção dos Discursos, consistindo em vários livros de discursos, diálogos, versos, histórias, etc., tratando da doutrina em si, tal como foi resumida nas ``Quatro Nobres Verdades''; (III) o \emph{Abhidhamma-piṭaka}, ou a Colecção Filosófica, apresentando os ensinamentos do \emph{Sutta-piṭaka} de uma forma sistemática e filosófica.

O \emph{Dhamma} não é uma doutrina de revelação, mas o ensinamento da Iluminação baseado na compreensão lúcida da realidade. É o ensinamento da \emph{Quádrupla Verdade} que trata dos factos fundamentais da vida e da libertação realizada através do próprio esforço do homem, em direcção à introspecção e purificação. O \emph{Dhamma} oferece um sistema ético superior, mas realista, uma análise penetrante da vida, uma filosofia profunda, métodos práticos para o treino da mente -- resumidamente, uma orientação no seu todo, perfeita e acessível no Caminho para a Libertação. Ao responder ao clamor tanto do coração como da razão, e ao mostrar o libertador ``Caminho do Meio'' que nos conduz para além de todos os extremos fúteis e destruidores da mente e da conduta individual, o \emph{Dhamma} tem e terá sempre um apelo intemporal e universal onde quer que existam corações e mentes suficientemente maduras para valorizar a sua Mensagem.
\end{quote}

III. O SANGHA

\begin{quote}
O SANGHA -- lit. a assembleia, ou comunidade -- é a Ordem dos \emph{Bhikkhus} ou Monges Mendicantes, fundada pelo Buddha e ainda existente na sua forma original em Myanmar (Birmânia), Tailândia, Sri Lanka (Ceilão), Camboja, Laos e Chittagong (Bengala). Juntamente com a Ordem dos monges Jainas, é uma das ordens monásticas mais antigas do mundo. Entre os mais famosos discípulos no tempo do Buddha, encontravam-se: Sāriputta que, a seguir ao próprio Mestre, tinha a mais profunda compreensão no \emph{Dhamma}; Moggallāna, dotado com os maiores poderes sobrenaturais; Ānanda, o devotado discípulo e constante companheiro do Buddha; Mahā-Kassapa, o Presidente do Conselho que se reuniu em Rājagaha imediatamente a seguir à morte do Buddha; Anuruddha, o mestre de visão divina e da consciência pura e Rāhula, o filho do próprio Buddha.

O \emph{Sangha} providencia o veículo externo e as condições favoráveis para todos aqueles que, livres das distracções mundanas, desejem seriamente devotar toda a sua vida à realização do mais elevado objectivo que é a libertação. Assim, o \emph{Sangha} também possui um significado universal e intemporal, onde quer que o desenvolvimento religioso alcance a maturidade.
\end{quote}

O TRIPLO REFÚGIO

\begin{quote}
O Buddha, o \emph{Dhamma} e o \emph{Sangha}, são designados ``As Três Jóias'' (\emph{tiratana}) pela sua pureza inigualável e por serem, para o budista, aquilo que há de mais precioso no mundo. Estas ``Três Jóias'' constituem também o ``Triplo Refúgio'' (\emph{ti-saraṇa}) que o praticante assume, ao proferir as palavras com as quais o declara ou reafirma, ao adoptá-las como guias da sua vida e do seu pensamento.

A fórmula Pāḷi do Refúgio é ainda a mesma aquando do tempo do Buddha:

\emph{Buddhaṃ saraṇaṃ gacchāmi}. \emph{Dhammaṃ saraṇaṃ gacchāmi. Sanghaṃ saraṇaṃ gacchāmi.}

Eu busco o refúgio no \emph{Buddha} Eu busco o refúgio no \emph{Dhamma} Eu busco o refúgio no \emph{Sangha}

É através do simples acto de recitar esta fórmula três vezes, que uma pessoa se considera budista. Na segunda e terceira repetições, acrescentam-se as palavras correspondentes \emph{Dutiyampi} e \emph{Tatiyampi} no início das frases:

\emph{Dutiyampi Buddhaṃ saraṇaṃ gacchāmi}.

\emph{Dutiyampi Dhammaṃ saraṇaṃ gacchāmi.}

\emph{Dutiyampi Sanghaṃ saraṇaṃ gacchāmi.}

Pela 2ª vez eu busco o refúgio no \emph{Buddha}

Pela 2ª vez eu busco o refúgio no \emph{Dhamma}

Pela 2ª vez eu busco o refúgio no \emph{Sangha}

\emph{Tatiyampi Buddhaṃ saraṇaṃ gacchāmi}.

\emph{Tatiyampi Dhammaṃ saraṇaṃ gacchāmi.}

\emph{Tatiyampi Sanghaṃ saraṇaṃ gacchāmi.}

Pela 2ª vez eu busco o refúgio no \emph{Buddha}

Pela 2ª vez eu busco o refúgio no \emph{Dhamma}

Pela 2ª vez eu busco o refúgio no \emph{Sangha}

OS CINCO PRECEITOS

A seguir à fórmula do Triplo Refúgio, normalmente assumem-se os Cinco Preceitos Morais \emph{(pañca-sīla)}. A sua obser- vância é o requisito de base para uma vida íntegra e consequente progresso em direcção à Libertação.

1. \emph{Pāṇātipātā veramaṇī-sikkhāpadaṃ samādiyāmi.}

Eu assumo o preceito de me abster de matar seres vivos.

2. \emph{Adinnādānā veramaṇī-sikkhāpadaṃ samādiyāmi.}

Eu assumo o preceito de me abster de tirar o que não me é oferecido.

3. \emph{Kāmesu micchācārā veramaṇī-sikkhāpadaṃ samādiyāmi.}

Eu assumo o preceito de me abster de sexualidade imprópria.

4. \emph{Musāvādā veramaṇī-sikkhāpadaṃ samādiyāmi.}

Eu assumo o preceito de me abster de discurso desonesto.

5. \emph{Surāmeraya - majja - pamādaṭṭhānā- veramaṇī - sikkhāpadaṃ samādiyāmi.}

Eu assumo o preceito de me abster de bebidas e

drogas intoxicantes que conduzem à falta de consciência.
\end{quote}
