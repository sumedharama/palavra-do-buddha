\chapterNote{A Nobre Verdade do Sofrimento}

\chapter{A Primeira Nobre Verdade}

\tocChapterNote{A Nobre Verdade do Sofrimento}

O que é afinal a nobre verdade do sofrimento?

Nascer é sofrimento; envelhecer é sofrimento; morrer é sofrimento; a tristeza, a
lamentação, a dor, a angústia e o desespero são sofrimento; não conseguir o que
se deseja, é sofrimento; resumindo: os cinco agregados da existência são
sofrimento.

E afinal, o que é nascer? É o aparecer de seres pertencentes a determinada
ordem, a sua concepção e o acto de nascerem, o virem à existência, a
manifestação dos agregados da existência, o começo da actividade sensitiva -- a
isto chama"-se nascer.

E o que é envelhecer? É a degradação de seres pertencentes a determinada ordem,
o acumular de mais idade, o debilitar, o ficar grisalho, o enrugar; a diminuição
da força vital, a exaustão dos sentidos -- a isto chama"-se envelhecer.

E o que é morrer? É a partida e o desfalecer de seres de determinada ordem, a
sua destruição, o desaparecimento, o término do seu período de vida, a
dissolução dos agregados da existência, o descartar do corpo -- a isto chama"-se
morte.

E o que é a tristeza? A tristeza surge por qualquer tipo de perda ou infortúnio
com que a pessoa se depara, pela preocupação, susto, aflição e lamento -- a isto
chama"-se tristeza.

E o que é o lamento? É toda a lamúria e queixume por qualquer tipo de perda,
infortúnio sofrido, o facto de se lamentar e recriminar, o estado de aflição e
deploração -- a isto chama"-se lamentação.

E o que é a dor? É a sensação dolorosa e desagradável produzida pela impressão
física -- a isto chama"-se dor.

E o que é a angústia? É a dor e o desagrado mental, o sentimento doloroso e
desagradável produzido pela impressão mental -- a isto chama"-se angústia.

E o que é o desespero? É o estado aflitivo e angustiante que surge de qualquer
tipo de perda ou infortúnio com que a pessoa se depara, a desolação e a
exasperação -- a isto chama"-se desespero.

E o que é o sofrimento por não se conseguir o que se deseja? Aos seres que estão
sujeitos a nascer, surge o desejo: “Ah, pudéssemos não estar sujeitos a nascer!
Pudéssemos não ter pela frente mais nenhum nascimento!”. Sujeitos ao
envelhecer, à doença, à morte, à tristeza, à lamentação, à dor, à angústia e ao
desespero, surge"-lhes o desejo: “Ah, pudéssemos não estar sujeitos a estas
coisas! Pudéssemos não ter de nos sujeitar a isto de novo!” Mas tal não se
obtém por mero desejo; e não obter o que se deseja, é sofrimento.

\quoteRef{\href{https://suttacentral.net/dn22/en/sujato}{DN 22}}

\clearpage

\section{Os Cinco Khandhas, ou Agregados da Existência}

\sectionSubtitle{(pañcupādānakkhandhā)}

E o que são os cinco agregados da existência? Eles são a corporalidade, a
sensação, a percepção, as formações mentais e a cognição.

\quoteRef{\href{https://suttacentral.net/dn22/en/sujato}{DN 22}}

Todos os fenómenos físicos, sejam eles do passado, do presente ou futuro, de uma
pessoa ou exteriores a ela, grosseiros ou subtis, superiores ou inferiores,
distantes ou próximos, pertencem ao agregado da corporalidade; todas as
sensações pertencem ao agregado da sensação; todas as percepções pertencem ao
agregado da percepção; todas as formações mentais pertencem ao agregado das
formações mentais; toda a cognição pertence ao agregado da cognição.

\quoteRef{\href{https://suttacentral.net/mn109/en/sujato}{MN 109}}

\begin{quote}
  Estes agregados estão reunidos numa classificação quíntupla, na qual o Buddha
  resumiu todos os fenómenos físicos e mentais da existência, principalmente
  aqueles que parecem ao homem ignorante como sendo o seu ego e sua
  personalidade. Assim, o nascimento, o envelhecimento, a morte, etc., estão
  também incluídos nestes cinco agregados que na realidade englobam o mundo
  inteiro.
\end{quote}

\clearpage

\subsection{O Agregado da Corporalidade}

\sectionSubtitle{(rūpa-khandha)}

Afinal, o que é o agregado da corporalidade? São os quatro elementos primários e
a corporalidade que deles deriva.

\subsubsection{Os Quatro Elementos}

E o que são os quatro elementos primários? São o elemento sólido, o elemento
fluido, o elemento térmico e o elemento vibrante (ventoso).

\quoteRef{\href{https://suttacentral.net/mn28/en/bodhi}{MN 28}}

\begin{quote}
  Os quatro elementos (\emph{dhātu} ou \emph{mahā"-bhūta}), popularmente chamados
  de Terra, Água, Fogo e Ar, deverão ser entendidos como qualidades elementares
  da matéria. São chamados em Pāli, \emph{pa\d{t}havī"-dhātu}, \emph{āpo"-dhātu},
  \emph{tejo"-dhātu}, \emph{vāyo"-dhātu}, e podem ser traduzidos como inércia,
  coesão, radiação e vibração. Todos os quatro estão presentes em qualquer
  objecto material, variando, no entanto, em grau de força. Se, por exemplo, o
  elemento Terra predomina, o objecto material é chamado “sólido”, etc.

  A “corporalidade derivada dos quatro elementos” (\emph{upādāya rūpa} ou
  \emph{upādā rūpa}) consiste, conforme o Abhidhamma, nos seguintes vinte e
  quatro fenómenos e qualidades materiais: olho, ouvido, nariz, língua, corpo,
  forma visível, som, cheiro, sabor, masculinidade, feminilidade, vitalidade, a
  base física da mente (\emph{hadaya"-vatthu}; ver B. Dic.), gesto, fala, espaço
  (cavidades do ouvido, nariz, etc.), envelhecimento, mudança e nutrição.

  As impressões corporais (\emph{pho\d{t}\d{t}habba} -- o tacto) não são propriamente
  mencionadas entre estes vinte e quatro fenómenos, uma vez que são idênticas
  aos elementos, sólido, térmico e vibrante, que são apreendidas pelas sensações
  de pressão, frio, calor, dor, etc.
\end{quote}

(1.) O que é afinal o elemento sólido (\emph{paṭhavī"-dhātu})? O elemento sólido
pode ser do próprio indivíduo, ou pode ser exterior. E o que é o elemento sólido
do indivíduo? O que quer que exista na própria pessoa ou no corpo de dureza e
firmeza adquiridas karmicamente, tais como os cabelos da cabeça e do corpo,
unhas, dentes, pele, carne, tendões, ossos, medula, rins, coração, fígado,
diafragma, baço, pulmões, estômago, intestinos, mesentério, excremento e por aí
fora -- a isto chama"-se elemento sólido do indivíduo. Quer o elemento sólido
seja do indivíduo ou do exterior, é meramente o elemento sólido.

E há que compreender que, de acordo com a realidade e a verdadeira sabedoria --
“Isto não me pertence; eu não sou isto; isto não é o que eu sou”.

(2.) O que é afinal o elemento fluido (\emph{āpo"-dhātu})? O elemento fluido pode
ser do indivíduo ou do exterior. E o que é o elemento fluido do indivíduo? O que
quer que exista na própria pessoa ou no corpo, de liquidez ou fluidez adquiridas
karmicamente, tais como a bílis, mucosidades, pus, sangue, suor, gordura,
lágrimas, gordura da pele, saliva, muco nasal, líquido nas articulações, urina e
por aí fora -- a isto chama"-se elemento fluido do indivíduo. Quer o elemento
fluido seja do indivíduo ou do exterior, é mera- mente o elemento fluido.

E uma pessoa devia compreender, conforme a realidade e a verdadeira sabedoria,
que: - “Isto não me pertence; isto não sou eu; isto não é o que eu sou”.

(3.) O que é afinal o elemento térmico (\emph{tejo"-dhātu})? O elemento térmico
pode ser do indivíduo, ou do exterior. E o que é o elemento térmico do
indivíduo? O que quer que exista na própria pessoa ou no corpo, de calor ou
ardor adquiridos karmicamente, tal como tudo com que um indivíduo se aquece,
consome, queima, e tudo através do qual digere o que foi ingerido, bebido,
mastigado ou degustado e por aí adiante -- a isto chama"-se o elemento térmico do
indivíduo. Quer o elemento térmico seja do indivíduo ou do exterior, é meramente
o elemento térmico.

Há que compreender, que de acordo com a realidade e a verdadeira sabedoria --
“Isto não me pertence; eu não sou isto; isto não é o que eu sou”.

\enlargethispage{\baselineskip}

(4.) O que é afinal o elemento vibrante (ventoso) \emph{(vāyo"-dhātu)}? O
elemento vibrante pode ser do indivíduo, ou do exterior. E o que é o elemento
vibrante do indivíduo? O que quer que exista na própria pessoa ou no corpo, de
vento ou ventosidade adquiridos karmicamente, tal como os ventos que sobem e
descem, os ventos do estômago e intestinos, o vento que permeia todos os
membros, a inspiração e a expiração, etc. -- a isto chama"-se o elemento vibrante
do indivíduo. Quer o elemento vibrante seja do indivíduo ou do exterior, é
meramente o elemento vibrante.

E há que compreender, conforme a realidade e a verdadeira sabedoria -- “Isto
não me pertence; eu não sou isto; isto não é o que eu sou”.

Assim como se chama “cabana” ao espaço circunscrito formado com madeira e
juncos, canas e barro, da mesma forma se chama “corpo” ao espaço circunscrito
formado com ossos, tendões, carne e pele.

\quoteRef{\href{https://suttacentral.net/mn28/en/bodhi}{MN 28}}

\subsection{O Agregado da Sensação}

\sectionSubtitle{(vedanā"-khandha)}

Existem três tipos de sensação: agradável, desagradável e nem agradável nem
desagradável.

\quoteRef{SN 36.1}

\subsection{O Agregado da Percepção}

\sectionSubtitle{(saññā"-khandha)}

O que é afinal a percepção? Existem seis classes de percepção: percepção das
formas, dos sons, odores, sabores, sensações físicas, e dos objectos mentais.

\subsection{O Agregado das Formações Mentais}

\sectionSubtitle{(sa\.{n}khāra"-khandha)}

O que são afinal as formações mentais? Existem seis classes de volições
(\emph{cetanā}): vontade projectada nas formas (\emph{rūpa"-sañcetanā}), nos
sons, odores, sabores, nas sensações físicas, e nos objectos mentais.

\quoteRef{SN 22.56}

\clearpage

\begin{quote}
  O “agregado das formações mentais” (\emph{sa\.{n}khāra"-khandha}), é um termo
  colectivo para representar inúmeras funções ou aspectos da actividade mental
  que, acrescidos à sensação e à percepção, estão presentes num só momento da
  consciência. No Abhidhamma, são distinguidas cinquenta formações mentais, sete
  das quais são factores constantes da mente. O número e a composição do
  restante, varia consoante o carácter da respectiva classe de cognição (ver
  quadro no \emph{B. Dic.}). No Discurso sobre a Visão Correcta (\href{https://suttacentral.net/mn9/en/bodhi}{MN 9}),
  são mencionados três factores principais representativos do agregado das
  formações mentais: volição (\emph{cetanā}), impressão sensual (\emph{phassa})
  e atenção (\emph{manasikāra}). Destes, uma vez mais, é a volição que sendo um
  factor principal “formativo”, é particularmente característico do agregado
  das formações, tendo sido assim utilizado para exemplificá"-lo na passagem
  acima referida.

  Para outras aplicações do termo \emph{sa\.{n}khāra}, ver \emph{B. Dic.}
\end{quote}

\subsection{O Agregado da Cognição}

\sectionSubtitle{(viññā\d{n}a-khandha)}

O que é afinal a cognição? Há seis classes de cognição: cognição das formas, dos
sons, dos odores, sabores, sensações físicas, e dos objectos mentais (lit.:
cognição"-visão, cognição"-audição, etc.).

\quoteRef{SN 22.56}

\section{A Génese Dependente da Cognição}

Mesmo que uma pessoa veja bem, se no entanto, as formas externas não estiverem
dentro do seu campo de visão, e não suceder qualquer ligação correspondente (de
vista e formas), não ocorrerá nesse caso a respectiva génese do aspecto da
cognição. Ou, mesmo que uma pessoa tenha boa visão e as formas externas estejam
dentro do seu campo de visão, mas mesmo assim não suceder qualquer ligação
correspondente, igualmente aí não ocorrerá a respectiva génese do aspecto da
cognição. Se, porém, a pessoa tiver uma boa visão, as formas externas estiverem
dentro do seu campo de visão e a ligação correspondente suceder, nesse caso
ocorrerá a respectiva génese do aspecto da cognição.

\quoteRef{\href{https://suttacentral.net/mn28/en/bodhi}{MN 28}}

Por isso afirmo: a génese da cognição depende das condições e, sem estas
condições, não se gera cognição alguma. E sejam quais forem as condições de que
a cognição depende, esta denomina"-se segundo as respectivas condições.

Quando a génese da cognição depende da vista e das formas, denomina"-se
“cognição visual” (\emph{cakkhu"-viññāṇa}).

Quando a génese da cognição depende do ouvido e dos sons, denomina"-se “cognição
auditiva” (\emph{sota"-viññāṇa}).

Quando a génese da cognição depende do órgão do olfacto e dos odores,
denomina"-se “cognição olfactiva” (\emph{ghāna"-viññāṇa}).

Quando a génese da cognição depende da língua e dos sabores, denomina"-se
“cognição"-palatal” (\emph{jivhā"-viññāṇa}).

Quando a génese da cognição depende do corpo e das sensações físicas,
denomina"-se “cognição"-corporal” (\emph{kāya"-viññāṇa}).

Quando a génese da cognição depende da mente e dos objectos da mente,
denomina"-se “cognição"-mental” (\emph{mano"-viññāṇa}).

\quoteRef{MN 38}

O que quer que exista de corporalidade (\emph{rūpa}), nesse momento pertence ao
agregado da corporalidade. O que quer que exista de “sensação”
(\emph{vedanā}), pertence ao agregado da sensação. O que quer que exista de
“percepção” (\emph{saññā}), pertence ao agregado da percepção. O que quer que
exista de “formações mentais” (\emph{saṅkhāra}), pertence ao agregado das
formações men- tais. O que quer que exista de “cognição” (\emph{viññāṇa})
pertence ao agregado da cognição.

\quoteRef{\href{https://suttacentral.net/mn28/en/bodhi}{MN 28}}

\section{A Dependência da Cognição\\ dos Outros Quatro Khandhas}

Também é impossível explicar o que é o término de uma existência e a entrada
noutra, ou o crescimento, ou o aumento e o desenvolvimento da cognição,
independentemente da corporalidade, sensação, percepção e formações mentais.

\quoteRef{SN 22.53}

\clearpage

\section{As Três Características da Existência}

\sectionSubtitle{(ti-lakkha\d{n}a)}

Todas as formações são “transitórias” (\emph{anicca}); todas as formações
estão sujeitas ao sofrimento (\emph{dukkha}); todas as coisas são desprovidas de
um eu (\emph{anattā}).

\quoteRef{AN 3.134}

A corporalidade é transitória, a sensação é transitória, a percepção é
transitória, as formações mentais são transitórias, a cognição é transitória.

E o que é transitório, está sujeito ao sofrimento; e é incorrecto dizer --
“Isto pertence"-me; isto sou eu; isto é o que eu sou” - daquilo que é
transitório e sujeito ao sofrimento e à mudança.

Assim, o que quer que exista de corporalidade, de sensação, percepção, formações
mentais, cognição, seja do passado, presente ou do futuro, do nosso interior ou
exterior, grosseiro ou subtil, elevado ou inferior, distante ou próximo, deve"-se
compreender segundo a realidade e a verdadeira sabedoria -- “Isto não me
pertence; eu não sou isto; isto não é o que eu sou”.

\quoteRef{SN 22.59}

\section{A Doutrina Anattā}

\begin{quote}
  A existência individual, bem como a de todo o mundo, não é, na realidade, mais
  do que um processo de fenómenos em constante mutação, todos incluídos nos
  cinco agregados da existência. Este processo tem decorrido desde antes do
  nosso nascimento, há tempos imemoriais, e assim continuará também depois da
  nossa morte, por tempos sem fim, enquanto e até onde existirem condições para
  tal. Como referido nos textos anteriores, os cinco agregados da existência --
  sejam eles considerados separadamente ou combinados -- de forma alguma
  constituem uma verdadeira entidade ego ou personalidade subsistente, e da
  mesma forma nenhum “eu”, alma ou substância se poderá encontrar como seu
  proprietário fora destes agregados. Por outras palavras, os cinco agregados da
  existência são “não"-eu” (\emph{anattā}), nem tão pouco pertencem a um “eu”
  (\emph{anattaniya}). Tendo em conta a impermanência e condicionalidade de toda
  a existência, a crença em qualquer “forma” de “eu” deverá ser vista como
  uma ilusão.

  Tal como o que designamos de “carruagem” não tem existência separada dos
  eixos, das rodas, veios, corpo e por aí adiante, assim bem como a palavra
  “casa”, que sendo apenas uma designação apropriada para indicar vários
  materiais reunidos, encerrando determinado espaço, não existe na realidade
  como entidade"-casa separada, precisamente da mesma forma, aquilo que nós
  chamamos de “ser”, “indivíduo”, “pessoa”, ou “eu”, não é senão uma
  combinação transitória de fenómenos físicos e mentais, sem existência real
  própria.

  Isto é resumidamente, a doutrina anattā do Buddha, o ensinamento de que toda a
  existência é vazia (\emph{suñña}) de um “eu” ou substância permanente.

  É a doutrina fundamental budista, que não se encontra em nenhum outro
  ensinamento religioso ou sistema filosófico. Percebê"-la plenamente, não só
  apenas de uma forma abstracta e intelectual, mas com referência constante à
  experiência real, é condição indispensável para a verdadeira compreensão do
  Buddha"-Dhamma e para a realização do seu objectivo. A doutrina"-anattā é o
  resultado indispensável da análise minuciosa da realidade, efectuada por
  exemplo na doutrina dos cinco khandhas, da qual só pode ser feita uma ligeira
  referência com os textos aqui incluídos.

  Para uma análise pormenorizada sobre os khandhas, ver \emph{B. Dic}.
\end{quote}

Imagine"-se um homem que não sendo cego, contempla as inúmeras bolhas no Ganges,
observando"-as e examinando"-as à medida que passam; após tê"-las examinado
cuidadosamente, parecem"-lhe vazias, irreais e insubstanciais.

Precisamente da mesma maneira, o monge contempla todos os fenómenos corporais,
sensações, percepções, formações mentais, e estados de cognição -- sejam eles do
passado, do presente ou do futuro, distantes ou próximos. Observa"-os e
examina"-os cuidadosamente; e após examiná"-los com cuidado, eles parecem"-lhe
vazios, sem nada e sem um eu.

\quoteRef{SN 22.95}

Quem quer que se deleite na corporalidade, ou na sensação, ou na percepção, ou
nas formações mentais, ou na consciência, deleita"-se no sofrimento; e quem se
deleita no sofrimento, não se libertará do sofrimento.

\quoteRef{SN 22.29}

\clearpage

Assim digo,

\begin{verse}
Que delícia e regozijo poderás encontrar\\
Quando tudo arde incessantemente?\\
Estás fechado na mais profunda escuridão!\\
Porque não procuras a luz?

Olha aqui para este fantoche, bem ataviado,\\
Acumulando mazelas,\\
Doente, e cheio de gula,\\
Instável, e impermanente!

Esta forma é devorada pela idade avançada,\\
Presa da doença, fraca e frágil;\\
Em pedaços se partirá este corpo pútrido,\\
A vida acabando na morte.
\end{verse}

\quoteRef{Dhp 146-148}

\section{Os Três Avisos}

Nunca viste um homem ou mulher neste mundo, com oitenta, noventa, ou cem anos de
idade, frágil, quebrado como um telhado velho, curvado, apoiado em muletas, com
passos inseguros, sem firmeza, a juventude há muito perdida, os dentes
estragados, cabelo ralo e branco ou nenhum, cheio de rugas, os seus membros com
manchas? E nunca pensaste que também tu estás sujeito à velhice e que não
conseguirás escapar"-lhe?

Nunca viste um homem ou mulher neste mundo que, estando fatigados, aflitos, e
gravemente doentes, revolvendo"-se na sua própria impureza, foram ajudados por
uns para se levantarem e postos na cama por outros? E nunca pensaste que também
tu estás sujeito a adoecer, que a isso não conseguirás escapar? Nunca viste o
corpo de um homem ou mulher neste mundo, um, dois ou três dias depois da sua
morte, inchado, de cor azul"-escuro, em pleno apodrecimento? E nunca pensaste que
também tu estás sujeito à morte, que não lhe conseguirás escapar?

\quoteRef{AN 3.35}

\section{Sa\.{m}sāra}

O começo deste \emph{Saṁsāra} é inconcebível; difícil é de se conhecer qualquer
princípio dos seres que, obstruídos pela ignorância e enredados na cobiça,
correm apressadamente através deste ciclo de renascimentos.

\quoteRef{SN 15.3}

\begin{quote}
  \emph{Sa\.{m}sāra} -- a roda da existência, lit. o “ciclo perpétuo” - é o
  nome dado, nas escrituras Pāli, ao mar da vida que se agita constantemente
  para cima e para baixo, símbolo deste contínuo processo de nascer, uma e outra
  vez, de envelhecer, sofrer e morrer. Mais precisamente: \emph{Sa\.{m}sāra} é a
  sequência ininterrupta das combinações entre os cinco khandhas que, em
  constante mutação, a cada momento se sucedem continuamente ao longo de
  períodos inconcebíveis. Deste saṁsāra, o período de uma vida constitui apenas
  uma minúscula fracção. Assim, de modo a compreender a Primeira Nobre Verdade,
  dever"-se-á meditar no saṁsāra, nesta terrível sequência de renascimentos, e
  não meramente numa só vida, a qual, como é evidente, poderá, por vezes, não
  ser assim tão dolorosa.

  Assim, o termo “sofrimento” (\emph{dukkha}), na primeira nobre verdade,
  refere"-se não só às sensações dolorosas do corpo e da mente provocadas pelas
  impressões desagradáveis, mas inclui também tudo o que produz sofrimento ou
  que seja responsável por este. A verdade do sofrimento ensina que, devido à
  lei universal da impermanência, até os estados sublimes e elevados de
  felicidade estão sujeitos à mudança e a acabar, e que todos os estados desta
  existência são assim insatisfatórios, carregando em si, sem excepção, as
  sementes do sofrimento.
\end{quote}

O que pensais ser maior: a inundação das lágrimas que, em choro e lamento haveis
derramado neste longo caminho -- nesta corrida desenfreada ao longo deste ciclo
de renascimentos unido ao que é indesejável e separado do que é desejável -- ou
as águas dos quatro oceanos?

Durante muito tempo sofrestes a morte de pai, mãe, filhos, filhas, irmãos e
irmãs. E nesse sofrimento, haveis derramado na realidade mais lágrimas neste
longo caminho do que a água dos quatro oceanos.

O que pensais ser maior: os rios de sangue que foram derramados pela vossa
decapitação, neste longo caminho\ldots{} ou as águas dos quatro oceanos?

Por eras sem fim, tendes sido apanhados como ladrões, bandidos ou adúlteros e,
na verdade, pela vossa decapitação, correu muito mais sangue neste longo caminho
do que a água dos quatro oceanos.

Mas como é isto possível?

É inconcebível o começo deste \emph{Saṁsāra}. Difícil será de conhecer qualquer
início dos seres que, obstruídos pela ignorância e enredados pela cobiça, correm
apressadamente por este ciclo de renascimentos.

\quoteRef{SN 15.13}

E assim, há muito que vós tendes vindo a sofrer, vivendo tormento, vivendo o
infortúnio, enchendo os cemitérios; na verdade, já há muito que tendes vivido o
suficiente para vos sentirdes insatisfeitos com todas as formas de existência, o
bastante para partir e libertar"-vos de todas elas.

\quoteRef{SN 15.1}
