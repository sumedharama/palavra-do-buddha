\chapter{As Quatro Nobres Verdades}

Assim foi dito pelo Buddha, o Iluminado:

Foi por não compreender, por não realizar quatro coisas, que eu, discípulos, tal
como vós, tive de vaguear tanto tempo neste ciclo de renascimentos. E quais são
essas quatro coisas? São:

\begin{enumerate}
  \item A nobre verdade do sofrimento \emph{(dukkha)};

  \item A nobre verdade da origem do sofrimento (\emph{dukkha-samudaya});

  \item A nobre verdade da extinção do sofrimento (\emph{dukkha-nirodha});

  \item A nobre verdade do caminho que conduz à extinção do sofrimento (\emph{dukkha-nirodha-gāmini-paṭipadā}).
\end{enumerate}

\quoteRef{DN 16}

Enquanto a visão introspectiva e o conhecimento perfeitamente verazes,
respeitando estas ``Quatro Nobres Verdades'', não se clarificaram em mim de
forma alguma, durante esse tempo, não tive a certeza de ter conquistado a
suprema iluminação, insuperável em todo o mundo com os seus seres celestiais,
espíritos malignos e deuses, entre todas as hostes de ascetas, sacerdotes e
homens. Mas, assim que a visão introspectiva e o conhecimento perfeitamente
verazes, com respeito a estas ``Quatro Nobres Verdades'', se clarificaram em
mim, surgiu-me interior- mente a certeza de ter conquistado aquela suprema e
insuperável iluminação.

\quoteRef{SN 56.11}

E eu descobri aquela verdade profunda, tão difícil de perceber, difícil de
compreender, tranquilizadora e sublime, que não é conquistada por mero
raciocínio e só é visível aos sábios.

\quoteRef{MN 26}

O mundo, está, no entanto, votado ao prazer, deleitado no prazer, enfeitiçado
com o prazer. Na verdade, tais seres dificilmente compreenderão a lei da
condicionalidade, a génese dependente (\emph{paṭicca-samuppāda}) de tudo;
incompreensível para eles será também o fim de todas as formações, o abandono de
tudo o que subjaz a cada renascimento, o desaparecimento da cobiça, do desapego
-- a libertação, o \emph{Nibbāna}.

No entanto, existem seres cujos olhos estão só ligeiramente cobertos de poeira:
estes compreenderão a verdade.

\quoteRef{MN 26}
