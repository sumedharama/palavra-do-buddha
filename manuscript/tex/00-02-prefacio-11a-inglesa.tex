\chapter[Prefácio à Décima Primeira Edição Inglesa]{Prefácio\\ à Décima Primeira Edição Inglesa}

\emph{A Palavra do Buddha}, cuja primeira edição foi publicada em língua alemã,
constituiu a primeira explanação sistemática das linhas mestras do Ensinamento
do Buddha, apresentada pelas palavras do próprio Mestre, tal como encontradas no
\emph{Sutta Piṭaka} do Cânone Pāli Budista.

Embora possa servir como primeira introdução para o principiante, o objectivo
principal deste livro é oferecer ao leitor que já se encontra mais ou menos
familiarizado com as ideias fundamentais do Budismo, uma síntese clara,
autêntica e concisa dos seus diversos ensinamentos, no enquadramento das
``Quatro Nobres Verdades'', respectivamente as verdades do sofrimento (inerente
a toda a existência), da origem do sofrimento, da extinção do sofrimento e do
caminho que conduz à extinção do sofrimento. Verifica"-se pelo próprio conteúdo
do livro, como os ensinamentos do Buddha, em última análise, convergem todos
para uma realização final: a Libertação do Sofrimento. Por essa razão se
encontrava impressa na capa da primeira edição em alemão, a seguinte passagem do
\emph{Aṅguttara Nikāya}, que diz:

\begin{verse}
  \emph{``Eu ensino não só a verdade do sofrimento, como também a libertação
    desse sofrimento''.}
\end{verse}

Os textos, traduzidos do Pāli original, foram seleccionados de entre as cinco
grandes colecções de discursos que formam o \emph{Sutta Piṭaka}. Foram agrupados
e explicados de modo a formarem um todo interligado. Assim, a colecção,
originalmente compilada de entre os inúmeros e volumosos livros do \emph{Sutta
  Piṭaka} para orientação do próprio autor, revela"-se um guia fidedigno para o
estudante do Budismo. Facilita o trabalho, no sentido de consultar todas as
demais secções das escrituras Pāli, permitindo obter uma visão clara no seu
todo; poderá ajudar a relacionar a parte principal da doutrina com os inúmeros
pormenores encontrados em estudos subsequentes.

Como o livro contém muitas definições e explanações de termos importantes da
doutrina, com respectiva equivalência Pāli, pode, com a ajuda da pronúncia Pāli
(ver p.\pageref{pron-pali}), servir como uma referência útil para o estudo individual
da doutrina do Buddha.

Depois da primeira edição em língua alemã em 1906, a primeira versão em língua
inglesa foi publicada em 1907 e, desde então, já se fizeram mais dez, incluindo
uma edição abreviada para estudantes (Colombo, 1948, Y.M.B.A.) e outra americana
(Santa Bárbara, Cal., 1950, J. F. Rowny Press). A obra foi também incluída na
Bíblia Budista de Dwight Goddard, publicada nos Estados Unidos da América.

Para além das edições subsequentes alemãs, já foram também editadas em francês,
italiano, checo, finlandês, russo, japonês, hindu, bengali e cingalês. O Pāli
original das passagens traduzidas foi publicado em caracteres ceilonenses
(edição do autor, sob o título \emph{Sacca"-Sangaha}, Colombo, 1914) e em escrita
devanagárica na Índia.

A 11ª edição foi totalmente revista. Foram feitas algumas adições à Introdução e
às notas explicativas, bem como acrescentados alguns textos.

\bigskip

{\raggedleft
  Nyanatiloka
\par}
