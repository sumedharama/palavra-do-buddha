\chapter{Imagem da Capa}

``Pegadas do Buddha'' (\emph{Buddhapada}) é uma das representações mais antigas
da arte e da simbologia budista na fase anti iconográfica (a ausência de
estátuas). O \emph{Buddhapada} é altamente reverenciado em países budistas,
especialmente no Sri Lanka e na Tailândia. Na Índia, os pés têm sido objecto de
respeito muito antes do Budismo, como arquétipo de ligação do ``transcendente''
à Terra.

De acordo com a lenda, o Buddha depois da sua iluminação, deixou a impressão dos
seus pés numa pedra onde caminhara em Kusinara, na Índia. As pegadas simbolizam
a \emph{presença do Buddha}, no contacto com a Terra e paradoxalmente, também a
\emph{ausência do Buddha}, aquando da sua entrada no Nirvāna, daí a memória ao
ideal budista do desapego.

As pegadas do Buddha são normalmente representadas com todos os dedos dos pés no
mesmo comprimento e com um \emph{Dharma-chakra} (Roda do Dharma) ao
centro. Outros símbolos budistas antigos aparecem também nos calcanhares e
dedos, tais como o Lótus, a \emph{Swastika} e as \emph{Triratna} (Três Jóias).

Resgatando o verdadeiro significado ancestral da cruz \emph{swastika},
independentemente das atrocidades cometidas com a sua imagem pelos nazis, a
palavra deriva do Sânscrito \emph{svastika} (em Devanagari \devanagari{स्वस्तिक}),
significando fortuna e bem-estar, um símbolo utilizado para dar boa sorte. A
palavra é composta por su-significando ``bom'', ``bem'' e \emph{asti} ``ser''
\emph{svasti} significando ``bem-estar''. O sufixo - \emph{ka} ora forma um
diminutivo ora intensifica o significado verbal, e~\emph{svastika} pode então
traduzir-se literalmente como ``aquilo que está associado com bem-estar'',
correspondendo a ``boa fortuna'' ou ``algo auspicioso''. Historicamente,
tornou-se um símbolo sagrado no Hinduísmo, Jainismo, Mitraismo e Xamanismo,
ganhando importância no Budismo durante o Império Máuria. Com a propagação do
Budismo, alcançou o Tibete e a China. Pensa-se que o seu uso pela fé indígena do
Tibete, bem como de religiões sincréticas como a \emph{Cao Dai} do Vietnam e a
\emph{Falun Gong} da China, também se originou do Budismo. O símbolo pode também
ser encontrado por toda a Coreia.

Hoje em dia é usado na arte e nas escrituras budistas, representando o
Dharma, a harmonia universal e o equilíbrio dos opostos. Pode observar-se
a \emph{swastika} nos pilares de Ashoka (304 A.C.), onde simboliza a dança
cósmica em torno de um centro fixo, funcionando como protecção contra o mal.
