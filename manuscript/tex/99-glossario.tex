\chapter{Glossário}

\subsection{A}

\begin{glossarydescription}

\item[Anattā] Termo sempre usado para adjectivar o que não é
\emph{“attan”} ou o que é desprovido de \emph{“attan”}. Não significa a
negação por si da realidade de \emph{“attan”}, senão como adjectivo qualificativo
do que não é \emph{“attan”}. Doutrina budista de que toda a composição substancial
dos cinco agregados da existência e do Universo manifestado, tal como o
conhecemos, está destinada à desintegração total, sendo impermanente e qualificada
pelo Buddha de \emph{anattā}, isenta de \emph{“attan”}. Querendo isto dizer que a
natureza de \emph{“attan”} não se encontra no Cosmos manifestado, mas na Realidade
Imortal Imanifestada, como o Buddha defendeu.

\item[Anicca] Impermanente, inconstante, evanescente; instável.

\item[Arahant (arahat)] Digno, merecedor, respeitável, honrado, nobre; adoptado
pelos budistas para indicar aquele que alcançou o \emph{Summum Bonum} da
aspiração espiritual (\emph{Nibbāna}). (o mesmo que \emph{“ārya”}- nobre ou
\emph{“ariya"-puggala”}).

\item[Āsava] Vício, mancha, obsessão, tendências intoxicantes, corrupções,
ulceração.

\item[Asubha] Impureza, asquerosidade, sujidade, putrefacção, imundice.

\item[Attan (attā)] (Sânscrito: \emph{ātman}). (1) A alma/espírito. (2) a si
mesmo, o próprio, a própria, ele próprio, ela própria, a ti mesmo; tu mesmo; tu
próprio/a. A essência.

\item[Ahiṁsā] \emph{(avihiṁsā)} Não"-violência, ausência de crueldade, não
injuriar.

\item[Avijjā] Ignorância, desconhecimento néscio, não conhecer; sinónimo de
ilusão. No contexto budista, ignorância define"-se principalmente pelo
desconhecimento das “Quatro Nobres Verdade” respectivamente “o sofrimento”,
“a causa do sofrimento”, “a extinção do sofrimento” e “o caminho que conduz
à extinção do sofrimento”.

\end{glossarydescription}

\subsection{B}

\begin{glossarydescription}

\item[Bhava] Voltar a existir; o processo da existência envolvendo os três planos, nomeadamente da existência sensual, da existência matéria"-subtil e, da existência imaterial.

\item[Bhāvanā] desenvolver, cultivar mentalmente.

\item[Bojjhaṅga] Os sete factores da iluminação.

\item[Brahma"-Vihāra] As quatro “Residências Sublimes” ou “Divinas”, também
chamadas de “os quatro Estados Incondicionados” (\emph{appamañña}), que são:
Amor Incondicional"-Gentileza"-Compreensão (\emph{mettā}), Compaixão
(\emph{karuṇā}), Alegria empática e altruísta (\emph{muditā}), Equanimidade
(\emph{upekkhā}).

\item[Buddha] O “Desperto”. Aquele que atingiu a iluminação; que se elevou da
esfera humana pelo conhecimento e prática da verdade, um Buddha. A palavra
Buddha é um apelativo e não um nome próprio.

\end{glossarydescription}

\subsection{C}

\begin{glossarydescription}

\item[Cetanā] Volição, vontade. É um dos sete factores mentais.

\item[Citta] Mente, consciência, estado de consciência.

\end{glossarydescription}

\subsection{D}

\begin{glossarydescription}

\item[Dhamma] (Sânscrito: \emph{Dharma}) Lei, doutrina, dever. A reflexão da
lei cósmica e da qualidade intrínseca de toda a fenomenologia. Integridade,
justiça, probidade. A doutrina patente nas escrituras Budistas e sua
instrução.

\item[Diṭṭhi] (Sânscrito: \emph{dṛṣṭi}) ponto de vista, credo, dogma, teoria,
especulação, ideologia -- teoria falsa e injustificada, opinião infundada.

\item[Dosa] Ódio.

\item[Dukkha] (Sânscrito: \emph{duḥkha}) (aplica"-se tanto ao mental, como ao
físico) Sofrimento, desagradável, doloroso, que causa miséria, dificuldade,
infelicidade, defeito, prisão.

\end{glossarydescription}

\subsection{J}

\begin{glossarydescription}

\item[Jarā] Velhice, caducidade, decadência.

\item[Jāti] Nascimento.

\item[Jhāna] (Sânscrito: \emph{dhyāna}) lit. Absorção. Meditação. Refere"-se
principalmente às quatro absorções, ou abstracções meditativas. Estados subtis,
supra materiais, alcançados através da concentração na meditação, com diminuição
e suspensão da actividade sensual dos cinco sentidos e dos cinco obstáculos
(\emph{nīvaraṇa}), sendo a vitalidade elevada para um estado de plena lucidez e
vigília. Em DN I.76 lê"-se: “com o seu coração sereno, tornado puro, translúcido,
composto, vazio de malícia, dócil, pronto para agir, firme e imperturbável”.
Vitalidade sublimada e engrandecida. \emph{Jhānas} são somente meios e não um
fim. Foi por terem apontado \emph{jhānas} como objectivo último do seu
ensinamento, que Gautama, o Buddha, rejeitou as doutrinas dos seus dois
professores. No entanto, mais tarde, o próprio Buddha confirmou a importância
dos \emph{jhānas} como meio e fase fundamental da realização do desapego do
mundo em direcção ao \emph{Nibbāna}.

\item[Jīva] A Alma. Princípio vital.

\end{glossarydescription}

\subsection{K}

\begin{glossarydescription}

\item[Kāma] Desejo dos sentidos, sensualidade subjectiva, no sentido mais
amplo do desejo ao nível dos cinco objectos dos sentidos, não exclusivamente
sexual, mas também; \emph{-chanda} (impulso): luxúria, semelhante a
\emph{kāma} no sentido amplo, mas com impulsividade adicional; \emph{-rāga}
(paixão, excitação): volúpia lasciva, aqui mais especificamente ao nível do
deleite e da indulgência sensual.

\item[Kamma] (Sânscrito: \emph{Karma}) Acção (saudável ou prejudicial). A causa
e sua consequência, a semente e sua germinação conforme a intenção, o acto e o
resultado perante a Lei Cósmica.

\item[Kammaṭṭhāna] lit. “O terreno de trabalho”, i.e., para meditação.

\item[Karuṇā] Compaixão. Um dos quatro \emph{Brahma"-Vihāras} (Residências Sublimes).

\item[Khandha] Agregado, substância: os cinco agregados, grupos, categorias,
substâncias, corpos -- da existência.

\item[Kilesa] Mácula, corrupção, fraquezas, qualidades prejudiciais.

\item[Kukkucca] Escrúpulo, remorso ou preocupação.

\item[Kusala] Karmicamente benéfico, saudável ou salutar.

\end{glossarydescription}

\subsection{L}

\begin{glossarydescription}

\item[Lobha] Ganância, cobiça, avareza.

\item[Lokiya] Mundano. Tudo o relacionado com o que é mundano, inclusive a
consciência e os factores mentais ainda não associados ao que é supramundano.

\item[Lokuttara] Supramundano.

\end{glossarydescription}

\subsection{M}

\begin{glossarydescription}

\item[Magga] O Caminho, ex: o “O Nobre Óctuplo Caminho”
(\emph{aṭṭhangika"-magga}).

\item[Māna] Presunção, orgulho.

\item[Mano] Mente.

\item[Mettā] Amor, Gentileza.

\item[Moha] Ilusão. Correspondendo a ignorância, imbecilidade, desorientação e
engano.

\item[Muditā] Alegria empática e altruísta.

\end{glossarydescription}

\subsection{N}

\begin{glossarydescription}

\item[Nibbāna] (Sânscrito: \emph{Nirvāṇa}) lit. Soprar, extinguir, apagar,
“pulverizar” o apego e o desejo aprisionante. O último e mais elevado
objectivo de todas as aspirações budistas = Libertação do desejo sensual pelo
desapego, com a “aniquilação” total da afirmação de vida acompanhada
de cobiça, ódio e ilusão (\emph{lobha, dosa, moha}).

\end{glossarydescription}

\subsection{P}

\begin{glossarydescription}

\item[Pahāna] Abandonar, vencer.

\item[Padhāna] Esforço.

\item[Paññā] Sabedoria, conhecimento.

\item[Paramattha] A verdade mais elevada.

\item[Phala] Fruto, resultado, efeito, benefício.

\item[Phassa] Impressão dos sentidos, contacto.

\item[Pīti] Êxtase, arrebatamento, delícia, júbilo.

\item[Puthujjana] Pessoa mundana, o vulgo, o comum dos homens, o leigo ou o
monge que ainda estão aprisionados pelos dez obstáculos.

\end{glossarydescription}

\subsection{R}

\begin{glossarydescription}

\item[Rāga] Paixão, excitação.

\item[Rūpa] Corporalidade, matéria"-subtil.

\end{glossarydescription}

\subsection{S}

\begin{glossarydescription}

\item[Sakkāya] (Sânscrito: \emph{satkāya}) Individualidade, personalidade, ego.

\item[Samādhi] concentração, recolhimento, estado da mente unificada com o
objecto de meditação, integração imperturbável num único ponto.

\item[Samatha] Tranquilidade, serenidade.

\item[Sammā] Excelente, certo, correcto.

\item[Sampajañña] Compreensão, discriminação, circunspecção.

\item[Saṁsāra] lit. “Ciclo perpétuo”. A roda dos renascimentos.

\item[Saṁvara] Restringir, evitar, dominar.

\item[Saṁyojana] Grilhão, laço, prisão, cadeia. Que prende, que amarra.

\item[Sangha] Congregação. Comunidade monástica budista.

\item[Saṅkhāra] refere"-se ao potencial formativo e criativo, ao formar ou ao
estado passivo de “já se ter formado”. O terreno preparativo, tanto físico
como principalmente subtil, da génese ao nível da consciência, mente e
pensamento.

\item[Saṅkhata] O formado ou criado. Tudo o que seja originado ou condicionado.

\item[Saññā] Percepção.

\item[Sati] Consciência, lembrança, memória.

\item[Sīla] Moral, virtude, ética.

\item[Sīlabbata"-Parāmāsa] Apego a meras regras e rituais.

\item[Sukha] Prazer, agradável, felicidade, bênção.

\item[Suñña] Vazio.

\end{glossarydescription}

\subsection{T}

\begin{glossarydescription}

\item[Taṇhā] Sede, anseio, secura, carência, desejo.

\item[Tejo"-Dhātu] Elemento fogo, calor interior.

\item[Thīna"-Middha] Torpor, preguiça, indolência.

\end{glossarydescription}

\subsection{U}

\begin{glossarydescription}

\item[Uddhacca] Inquietação.

\item[Upekkhā] Equanimidade.

\end{glossarydescription}

\subsection{V}

\begin{glossarydescription}

\item[Vāyāma] Empenho.

\item[Vedanā] Sentimento, sensação.

\item[Vibhava] Poder, riqueza, prosperidade.

\item[Vicāra] Pensamento discursivo.

\item[Vicikicchā] Dúvida, cepticismo.

\item[Viññāṇa] Cognição. Qualidade mental como constituinte da individualidade.
Consciência física, sensorial e aperceptiva.

\item[Vipassanā] Introspecção. Visão e compreensão introspectiva.

\item[Viriya] Vigor, energia, virilidade.

\item[Vitakka] Pensamento. Concepção"-pensamento.

\item[Vyāpāda] Má"-fé, maledicência, perfídia, malícia.

\end{glossarydescription}
