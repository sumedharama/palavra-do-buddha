\chapter{Glossario}

\begin{quote}
\textbf{A}

\emph{\textbf{Anattā}} -- Não \emph{``attan''}, no sentido de adjectivar o que não é \emph{``attan''}, ou o que não consiste de \emph{``attan''}. Não significa a negação em si da realidade de \emph{``attan''}, mas sim como adjectivo aplicado ao que não o é. Doutrina budista de que toda a composição substancial dos cinco agregados da existência e do Universo manifestado tal como o conhecemos, está votada à desintegração total, sendo por isso impermanente, razão pela qual o Buddha chamou a toda a realidade temporária manifestada, de \emph{anattā}, por ser isenta de qualquer qualidade intrínseca de \emph{``attan''}. Querendo isto dizer que a natureza de \emph{``attan''} não se encontra no Cosmos manifestado, mas na Realidade Imortal Imanifestada.

\emph{\textbf{Anicca}} -- Impermanente, inconstante, evanescente; instável.

\emph{\textbf{Arahant (arahat)}} -- Digno, merecedor, respeitável, honrado, nobre; adoptado pelos budistas para indicar aquele que alcançou o \emph{Summum Bonum} da aspiração espiritual (\emph{Nibbāna}). (o mesmo que \emph{``ārya''}- nobre ou \emph{``arya-puggala''}).

\emph{\textbf{Āsava}} -- Vício, mancha, obsessão, tendências intoxicantes, corrupções, ulceração.

\emph{\textbf{Asubha}} -- Impureza, asquerosidade, sujidade, putrefacção, imundice.

\emph{\textbf{Attan (attā)}} -- (\emph{Sânscrito: ātman}). \textbf{1}. A alma/espírito. \textbf{2}. a si mesmo, o próprio, a própria, ele próprio, ela própria, a ti mesmo; tu mesmo; tu próprio/a.

\emph{\textbf{Ahimsā}} -- \emph{(avihimsā)} Não - violência, ausência de crueldade, não

injuriar.

\emph{\textbf{Avijjā}} -- Ignorância, desconhecimento néscio, não conhecer; sinónimo de ilusão. No contexto budista, ignorância define-se principalmente pelo desconhecimento das ``Quatro Nobres Verdade'' respectivamente ``o sofrimento'', ``a causa do sofrimento'', ``a extinção do sofrimento'' e ``o caminho que conduz à extinção do sofrimento''.

\textbf{B}

\emph{\textbf{Bhav a}} -- Voltar a existir; o processo da existência envolvendo os três planos, nomeadamente da existência sensual, da existência matéria-subtil e, da existência imaterial.

\emph{\textbf{Bhāvanā}} -- desenvolver, cultivar mentalmente.

\emph{\textbf{Bojjhanga}} -- Os sete factores da iluminação.

\emph{\textbf{Brahma-Vihāra}} -- As quatro ``Residências Sublimes'' ou ``Divinas'', também chamadas de ``os quatro Estados Incondicionados'' (\emph{appamañña}), que são: Amor Incondicional-Gentileza-Compreensão (\emph{mettā}), Compaixão (\emph{karuṇā}), Alegria empática e altruísta (\emph{muditā}), Equanimidade (\emph{upekkhā}).

\emph{\textbf{Buddha}} -- O ``Desperto''. Aquele que atingiu a iluminação; que se elevou da esfera humana pelo conhecimento e prática da verdade, um Buddha. A palavra Buddha é um apelativo e não um nome próprio.

\textbf{C}

\emph{\textbf{Cetanā}} -- Volição, vontade. É um dos sete factores mentais.

\emph{\textbf{Citta}} -- Mente, consciência, estado de consciência.

\textbf{D}

\emph{\textbf{Dhamma}} -- (Sânscrito: \emph{Dharma}) Lei, doutrina, dever. A reflexão da lei cósmica e da qualidade intrínseca de toda a fenomenologia. Integridade, justiça, probidade. A doutrina patente nas escrituras Budistas e sua instrução.

\emph{\textbf{Diṭṭhi}} -- (Sânscrito: \emph{dṛṣṭi}) ponto de vista, credo, dogma, teoria, especulação, ideologia -- teoria falsa e injustificada, opinião infundada.

\emph{\textbf{Dosa}} -- Ódio.

\emph{\textbf{Dukkha}} -- (\emph{Sânscrito: duḥkha}) (aplica-se tanto ao mental, como ao físico) Sofrimento, desagradável, doloroso, que causa miséria, dificuldade, infelicidade, defeito, prisão.

\textbf{J}

\emph{\textbf{Jarā}} -- Velhice, caducidade, decadência.

\emph{\textbf{Jāti}} -- Nascimento.

\emph{\textbf{Jhān a}} - (Sânscrito: \emph{dhyāna}) lit. Absorção. Meditação. Refere-se principalmente às quatro absorções, ou abstracções meditativas. Estados subtis, supra materiais, alcançados através da concentração na meditação, com diminuição e suspensão da actividade sensual dos cinco sentidos e dos cinco obstáculos (\emph{nīvaraṇa}), sendo a vitalidade elevada para um estado de plena lucidez e vigília. Em D I.76 lê-se: ``com o seu coração sereno, tornado puro, translúcido, composto, vazio de malícia, dócil, pronto para agir, firme e imperturbável''. Vitalidade sublimada e engrandecida. \emph{Jhānas} são somente meios e não um fim. Foi por terem apontado \emph{jhānas} como objectivo último do seu ensinamento, que Gautama, o Buddha, rejeitou as doutrinas dos seus dois professores. No entanto, mais tarde, o próprio Buddha confirmou a importância dos \emph{jhānas} como meio e fase fundamental da realização do desapego do mundo em direcção ao \emph{Nibbāna}.

\emph{\textbf{Jīva}} -- A Alma. Princípio vital.

\textbf{K}

\emph{\textbf{Kām a}} -- Desejo dos sentidos, sensualidade subjectiva, no sentido mais amplo do desejo ao nível dos cinco objectos dos sentidos, não exclusivamente sexual, mas também; - \emph{cchanda} (impulso): luxúria, semelhante a \emph{kāma} no sentido amplo, mas com impulsividade adicional; - \emph{rāga} (paixão, excitação): volúpia lasciva, aqui mais especificamente ao nível do deleite e da indulgência sensual.

\emph{\textbf{Kamma}} -- (Sânscrito: \emph{Karma}) Acção (saudável, ou prejudicial). A

causa e sua consequência, a semente e sua germinação conforme a intenção, o acto e o resultado perante a Lei Cósmica.

\emph{\textbf{Kammaṭṭhāna}} -- lit. ``O terreno de trabalho'', i.e., para meditação.

\emph{\textbf{Karuṇā}} -- Compaixão. Um dos quatro \emph{Brahma-Vihāras} (Residências Sublimes).

\emph{\textbf{Khandha}} -- Agregado, substância: os cinco agregados, grupos,

categorias, substâncias, corpos -- da existência.

\emph{\textbf{Kilesa}} -- Mácula, corrupção, fraquezas, qualidades prejudiciais.

\emph{\textbf{Kukkucca}} -- Escrúpulo, remorso ou preocupação.

\emph{\textbf{Kusala}} -- Karmicamente benéfico, saudável ou salutar.

\textbf{L}

\emph{\textbf{Lobha}} -- Ganância, cobiça, avareza.

\emph{\textbf{Lokiya}} -- Mundano. Tudo o relacionado com o que é mundano, inclusive a consciência e os factores mentais ainda não associados ao que é supramundano.

\emph{\textbf{Lokuttara}} -- Supramundano.

\textbf{M}

\emph{\textbf{Magga}} -- O Caminho, ex: o ``O Nobre Óctuplo Caminho''

\emph{(aṭṭhangika-magga). \textbf{Māna }}-- Presunção, orgulho. \emph{\textbf{Mano}} -- Mente.

\emph{\textbf{Mettā}} -- Amor, Gentileza.

\emph{\textbf{Moha}} -- Ilusão. Correspondendo a ignorância, imbecilidade, desorientação e engano.

\emph{\textbf{Muditā}} -- Alegria empática e altruísta.

\textbf{N}

\emph{\textbf{Nibbāna}} -- \emph{{[}}Sânscrito: \emph{Nirvāṇa{]}} lit. Soprar, extinguir, apagar, ``pulverizar'' o apego e o desejo aprisionante. O último e mais elevado objectivo de todas as aspirações budistas = Libertação do desejo sensual pelo desapego, com a ``aniquilação'' total da afirmação de vida normalmente manifestada como cobiça, ódio e ilusão \emph{(lobha, dosa, moha)}.

\textbf{P}

\emph{\textbf{Pahāna}} -- Abandonar, vencer.

\emph{\textbf{Padhāna}} -- Esforço.

\emph{\textbf{Paññā}} -- Sabedoria, conhecimento. \emph{\textbf{Paramattha}} -- A verdade mais elevada. \emph{\textbf{Phala}} -- Fruto, resultado, efeito, benefício. \emph{\textbf{Phassa}} -- Impressão dos sentidos, contacto. \emph{\textbf{Pīti}} -- Êxtase, arrebatamento, delícia, júbilo.

\emph{\textbf{Puthujjana}} -- Pessoa mundana, o vulgo, o comum dos homens, o leigo ou o monge que ainda estão aprisionados pelos dez

obstáculos.

\textbf{R}

\emph{\textbf{Rāga}} -- Paixão, excitação.

\emph{\textbf{Rūpa}} -- Corporalidade, matéria-subtil.

\textbf{S}

\emph{\textbf{Sakkāya}} -- \emph{{[}}Sânscrito: \emph{satkāya{]}} Individualidade, personalidade, ego.

\emph{\textbf{Sam ādhi}} -- concentração, recolhimento, estado da mente

unificada com o objecto de meditação, integração

imperturbável num único ponto.

\emph{\textbf{Samatha}} -- Tranquilidade, serenidade.

\emph{\textbf{Sammā}} -- Excelente, certo, correcto.

\emph{\textbf{Sampajañña}} -- Compreensão, discriminação, circunspecção. \emph{\textbf{Saṁsāra}} -- lit. ``Ciclo perpétuo''. A roda dos renascimentos. \emph{\textbf{Saṁvara}} -- Restringir, evitar, dominar.

\emph{\textbf{Saṁyojana}} -- Grilhão, laço, prisão, cadeia. Que prende, que amarra.

\emph{\textbf{Sangha}} -- Congregação. Comunidade monástica budista.

\emph{\textbf{Sank hāra}} -- refere-se ao potencial formativo e criativo, ao formar ou ao estado passivo de ``já se ter formado''. O terreno preparativo, tanto físico como principalmente subtil, da génese ao nível da consciência, mente e pensamento.

\emph{\textbf{Sankhata}} -- O formado ou criado. Tudo o que seja originado ou condicionado.

\emph{\textbf{Saññā}} -- Percepção.

\emph{\textbf{Sati}} -- Consciência, lembrança, memória.

\emph{\textbf{Sīla}} -- Moral, virtude, ética.

\emph{\textbf{Sīlabbata-Parāmāsa}} -- Apego a meras regras e rituais.

\emph{\textbf{Sukha}} -- Prazer, agradável, felicidade, bênção.

\emph{\textbf{Suñña}} -- Vazio.

\textbf{T}

\emph{\textbf{Taṇhā}} -- Sede, anseio, secura, carência, desejo. \emph{\textbf{Tejo-Dhātu}} -- Elemento fogo, calor interior. \emph{\textbf{Thīna-Middha}} -- Torpor, preguiça, indolência.

\textbf{U}

\emph{\textbf{Uddhacca}} -- Inquietação.

\emph{\textbf{Upekkhā}} -- Equanimidade.

\textbf{V}

\emph{\textbf{Vāyāma}} -- Empenho.

\emph{\textbf{Vedanā}} -- Sentimento, sensação. \emph{\textbf{Vibhava}} -- Poder, riqueza, prosperidade. \emph{\textbf{Vicāra}} -- Pensamento discursivo. \emph{\textbf{Vicikicchā}} -- Dúvida, cepticismo.

\emph{\textbf{Viññ āṇa}} -- Qualidade mental como constituinte da individualidade.

Consciência física, sensorial e percepção.

\emph{\textbf{Vipassanā}} -- Introspecção. Visão e compreensão introspectiva.

\emph{\textbf{Viriya}} -- Vigor, energia, virilidade.

\emph{\textbf{Vitakka}} -- Pensamento. Concepção-pensamento.

\emph{\textbf{Vyāpāda}} -- Má-fé, maledicência, perfídia, malícia.
\end{quote}



\begin{glossarydescription}

% === A ===

\item[anicca] (Pali) Impermanence: one of the \emph{three characteristics of
    existence} along with not-self (\emph{anattā}) and unsatisfactoriness
  (\emph{dukkha}).

% === B ===

\item[borapet] (Thai) Tinospora crispa. Heart-shaped moonseed or guduchi.
  An extremely bitter vine used as a prophylactic and treatment for malaria.

% === C ===

% === D ===

% === E ===

% === F ===

% === G ===

% === H ===

% === I ===

% === J ===

% === K ===

% === L ===

% === M ===

% === N ===

% === O ===

% === P ===

% === Q ===

% === R ===

% === S ===

% === T ===

% === U ===

% === V ===

% === W ===

\end{glossarydescription}

